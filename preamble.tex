% Tipo di documento. L'uso di twoside implica che i capitoli inizino sempre con la prima pagina a sinistra, eventualmente lasciando una pagina vuota nel capitolo precedente. Se questa cosa è fastidiosa, è possibile rimuoverlo. 
\documentclass[a4paper, twoside,openright]{report}
% \documentclass[a4paper,openright]{report}

% \usepackage{fancyhdr}
% \pagestyle{fancy}
% \fancyhf{}
% \lhead{\rightmark}
% \rhead{\textbf{\thepage}}
% % \fancyfoot{}
% \setlength{\headheight}{12.5pt}
% Rimuove il numero di pagina all'inizio dei capitoli
% \fancypagestyle{plain}{
%   \fancyfoot{}
%   \fancyhead{}
%   \renewcommand{\headrulewidth}{0pt}
% }

\setcounter{secnumdepth}{4}
\setcounter{tocdepth}{4}

\usepackage{graphicx} % Required for inserting images
\setkeys{Gin}{width=0.6\columnwidth}

\usepackage[utf8]{inputenc}

\usepackage{hyperref}
\usepackage{adjustbox}

\usepackage{wrapfig}

\usepackage{enumitem}
\renewcommand{\labelitemi}{$\diamond$}
\renewcommand{\labelitemiii}{$\circ$}
\setlist[enumerate,2]{label=\roman*.}
\setlist[enumerate,3]{label=(\alph*)}

\setitemize{noitemsep}
\setenumerate{noitemsep}
\setlist{noitemsep}

\usepackage{paracol}
\usepackage{multicol}
\usepackage{booktabs}


\usepackage{geometry}

\usepackage{color}

\usepackage{listings}
% \usepackage{minted}

\usepackage{amsmath}
\usepackage{amssymb}
\usepackage{amsfonts}
\usepackage{mathtools}
\usepackage{bm}
\usepackage{nicefrac, xfrac}

\usepackage{wasysym}

% Uso dei colori
\usepackage[dvipsnames,table,xcdraw]{xcolor}
\usepackage{colortbl}
\usepackage{rotating}
\usepackage{adjustbox}

\usepackage{multirow}
\usepackage{booktabs}
\usepackage{makecell}


\usepackage{tikz}
\usetikzlibrary{automata, arrows,arrows.meta,bending}
\usetikzlibrary{positioning}
\usetikzlibrary{shapes.geometric}
\usepackage{parskip}
\usepackage{changepage}

\usepackage{soul}
\usepackage{cancel}

\usepackage{pifont} % used for tick and cross marks

% This are needed because the correct double quotes would be ``'' or ``",
% but i've always written "text"
% TODO - check whether this affects listing environment
% \usepackage [english]{babel}
% \usepackage [autostyle, english = american]{csquotes}
% \MakeOuterQuote{"}

\usepackage{minitoc}

\geometry{margin=0.6in}

\setlist[description]{itemsep=0em,topsep=0.5em,parsep=0em}
\setlist[itemize]{itemsep=0em,topsep=0pt}
\setlist[enumerate]{itemsep=0em,topsep=0pt}

\hypersetup{
    colorlinks=true,
    linkcolor=black,
    filecolor=mauve,
    urlcolor=blue,
}

\definecolor{gray}{gray}{0.3}
\definecolor{verylightgray}{gray}{0.95}
\definecolor{blue}{rgb}{0,0,1}
\definecolor{mauve}{rgb}{0.58,0,0.82}
\definecolor{darkred}{rgb}{0.3,0,0}
\definecolor{darkgreen}{rgb}{0,0.3,0}
\definecolor{darkgray}{gray}{0.15}



\newenvironment{notes}{
\par
\color{gray}
\small}

\newcommand{\note}[1]{\begin{notes}{#1}\end{notes}}
\newcommand{\nl}[0]{\parskip = \baselineskip}
\newcommand{\lst}[1]{\lstinline{#1}}
\newcommand{\ra}{\xrightarrow{\hspace*{2em}}}
\newcommand{\ns}{\setlength{\parskip}{0em}}

\newlength{\currentparindent}
\newcommand{\labelitemize}[2]{
\setlength{\currentparindent}{\parindent}
\setlength{\parindent}{0pt}

\begin{minipage}{0em} % Adjust the width as needed
    \makebox[0em][c]{\rotatebox{90}{\small #1}}
\end{minipage}
\begin{minipage}{\dimexpr\columnwidth-1cm\relax}
    #2
\end{minipage}
\setlength{\parindent}{\currentparindent}
}
\newcommand{\colfill}{\vspace{\fill}}

\newcommand{\framed}[1]{
\begin{center}
\fbox{
    \begin{minipage}{0.8\columnwidth}
        #1
    \end{minipage}
}
\end{center}}

\newcommand{\framedt}[2]{
\begin{center}
\fbox{
    \begin{minipage}{0.8\columnwidth}
        \vspace*{1em}
        \begin{center}
            \textbf{\ul{#1}}
        \end{center}
        \nl
        #2
    \end{minipage}
}
\end{center}}

\newcommand{\proscons}[4]{
    \begin{paracol}{2}
        \labelitemize{\color{darkgreen}\ul{\textit{#1}}}{
           \color{darkgreen}
           #3
        }
        \switchcolumn
        \labelitemize{\color{darkred}\ul{\textit{#2}}}{
           \color{darkred}
           #4
        }
     \end{paracol}
}

\newcommand{\coolquote}[2]{
    {
     \begin{center}
        \textit{\ul{``#1''}} --- #2
     \end{center}
	% \centering
}}

\newcommand{\textred}[1]{{\color{red}#1}}

\newcommand{\cmark}{\ding{51}}% TICK
\newcommand{\xmark}{\ding{55}}% CROSS

\newcommand\hcancel[2][black]{\setbox0=\hbox{$#2$}%
\rlap{\raisebox{.45\ht0}{\textcolor{#1}{\rule{\wd0}{1pt}}}}#2} 


\lstset{frame=false,
 showstringspaces=false,
 breaklines=true,
%  columns=flexible,
 basicstyle={\small\ttfamily},
 keywordstyle=\color{blue},
 commentstyle=\color{darkgreen},
 stringstyle=\color{mauve},
 tabsize=2
}

\newtheorem{definition}{Definition}[chapter]
\newtheorem{theorem}{Theorem}[chapter]
\newtheorem{example}{Example}[section]


\usepackage{fancyhdr}
% \usepackage{nameref,autoref}

\pagestyle{fancy}
\fancyhf{}
\fancyhead[LE,RO]{\thepage} % Page number on Outer side of header on each page
\fancyhead[LO]{\leftmark} % section title on Left side of header on Odd pages
\fancyhead[RE]{\rightmark} % subection title on Right side of header on Even pages
\fancyfoot{}
\renewcommand{\headrulewidth}{0.4pt} % Width of line under the header
% \setcounter{secnumdepth}{2} % Depth of sectioning commands to include in the table of contents

% Rimuove il numero di pagina all'inizio dei capitoli
\fancypagestyle{plain}{
  \fancyfoot{}
  \fancyhead{}
  \renewcommand{\headrulewidth}{0pt}
}

\usepackage{emptypage}

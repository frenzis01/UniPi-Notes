\chapter{Mobile Networks}
\begin{figure}[htbp]
   \centering
   \includegraphics[width=0.45\columnwidth]{images/2g_architecture.png}
   \includegraphics[width=0.45\columnwidth]{images/3g_architecture.png}\\
   \includegraphics[width=0.45\columnwidth]{images/4g_architecture.png}
   \includegraphics[width=0.45\columnwidth]{images/5g_architecture.png}

   \caption{Mobile Networks architectures}
   \label{fig:mobile_architectures}
\end{figure}

The key point in \textbf{3G} is the introduction of a data service, operating in parallel with voice network, which forced the important modifications to the architecture.

In \textbf{4G} also the voice traffic uses \textit{packet switching}, instead of circuit switching.

\framedt{Control vs Data plane}{
   \textbf{Control plane} includes routing protocols such as BGP and all the processes which handle and determine how data packets should be forwarded.
   
   \textbf{Data plane} instead handles the transport of host/application data, and performs the actually forwarding of packets.
   
   \textit{\footnotesize``Think of the control plane as being like the stoplights that operate at the intersections of a city. Meanwhile, the data plane (or the forwarding plane) is more like the cars that drive on the roads, stop at the intersections, and obey the stoplights''}
   \href{https://www.cloudflare.com/it-it/learning/network-layer/what-is-the-control-plane/}{Cloudflare Data/Control plane}
}


\begin{figure}[htbp]
   \centering
   \includegraphics{images/mobility.png}
   \caption{Mobility}
   \label{fig:mobility}
\end{figure}
To handle devices' mobility there must be home network to rely onto.
\begin{figure}[htbp]
   \centering
   \includegraphics[width=0.45\columnwidth]{images/homenetwork.png}
   \includegraphics[width=0.45\columnwidth]{images/homenetwork2.png}\\
   \includegraphics[width=0.45\columnwidth]{images/homenetwork3.png}
   \caption{Home and visited Networks}
   \label{fig:homenetwork}
\end{figure}

With respect to the questions posed in the third image in Fig. \ref{fig:homenetwork}, data being send from a device to a mobile one may be routed in three ways.
\begin{enumerate}
   \item 
   The first is the canonical routing, using IP addresses and routing tables, but it is not a feasible scenario for billions of devices.
   \item The other possibility is to rely on the edge of home and visited networks instead.
   \begin{enumerate}
      \item \textbf{Direct routing}\\
      Sender gets foreign address of mobile, send directly to mobile
      \item \textbf{Indirect routing}\\
      Communication from sender to mobile goes through home network, then forwarded to remote mobile
   \end{enumerate}
\end{enumerate}


\note{Many things are missing here. Refer to the slides or to chapter 7 of the course book (Kurose) for more information.}
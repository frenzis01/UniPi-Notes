\chapter{NFV/SDN Emulation tools}

\section{Mininet}
\href{http://mininet.org/}{Mininet} is a network emulator, it runs a collection of end-hosts, switches, routers, and links on a single Linux kernel.
It supports OpenFlow for SDN network emulation.

\note{
A Mininet host behaves just like a real machine, 
e.g can send packets with a given link speed and delay, or may ping or iperf another hosts}

\lstinline|sudo mn| initializes a default topology and launches the CLI\\
Overriding \lstinline|build()| of the \lstinline|mininet.topo.Topo| class to create a custom topology.

\begin{itemize}
   \item Topo: the base class for Mininet topologies
   \begin{itemize}
      \item \lstinline|addSwitch()|: adds a switch to a Topo and returns the switch name
      \item \lstinline|addHost()|: adds a host to a Topo and returns the host name
      \item \lstinline|addLink()|: adds a bidirectional link to Topo (and returns a link key, but this is not important).
      \ul{Links in Mininet are bidirectional unless noted otherwise.}
   \end{itemize}
   \item Mininet: main class to create and manage a network
   \begin{itemize}
      \item \lstinline|start()|: starts your network
      \item \lstinline|pingAll()|: tests connectivity by trying to have all nodes ping each other
      \item \lstinline|stop()|: stops your network
      \item \lstinline|net.hosts|: all the hosts in a network
   \end{itemize}

   \item[] Documentation: \href{http://mininet.org/api/annotated.html}{http://mininet.org/api/annotated.html}
\end{itemize}

\lstinline|veth| \textit{pairs} is a full-duplex link between two interfaces in same o separate namespace:
packets transmitted from one interface are directly forwarded to the other interface in the pair, so each inteface behaves like an ethernet interface.\\
\lstinline|ip link show type veth| or \lstinline|Mininet> links|
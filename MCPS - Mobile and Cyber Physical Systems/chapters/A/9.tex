\chapter{Embedded Programming}
The learning objectes of this chapter are embedded systems in general, and the Arduino case study.

Embedded systems are systems that are designed to perform a specific task, and they are usually part of a larger system.
Hardware and software are often designed together, aka ``hardware-software co-design''.
They are typically based on microcontrollers, optimized for controlling I/O.
They are often used in \ul{\textbf{real-time} systems, where timing is crucial}.

Many types of microcontrollers are availble on the market:
\begin{itemize}
   \item ``General purpose'', meaning that can be adapted to several embedded applications
   \item ``Application specific integrated circuit''(\texttt{ASIC}s) which are very efficient and performative, but tightly bound to a specific task
   \item ``System on a chip''(\texttt{SoC}s) which are a combination of a microcontroller and other components, like a radio module. 
   \note{The term is very broad, and it can refer to a wide range of devices.}
\end{itemize}

When programming on microcontrollers it must be taken into account the \textbf{small memory footprint}, the absence of user interfaces, file systems, OSs.
In general is not possible to program and compile code directly, or to control the program execution with a user interface.

\framedt{Programming Challenges}{
   \begin{itemize}
      \item Timing correctness
      \item High reliability 
      \item Often impossible to use debuggers
      \item Efficient use of memory
      \item Power management
   \end{itemize}
}

\section{Executable and SW organization}
When a device starts it executes the \texttt{main} program, which first initializes device peripherals using an \texttt{init} procedure and then enters an infinite \textbf{loop}, implementing the functionalities of the device.
Such loop naturally defines a duty cycle, typically consisting of:
{\ns\begin{itemize}
   \item Reading from transducers
   \item Taking a decision
   \item Controlling actuators
   \item Optionally communicating with other devices
\end{itemize}}

At low-level, the code interacts with the device hardware by means of commands and interrupts. In conventional OSs insteads commands are available in terms of system calls, which are later handled by the OS.

\section{Different Models to implement duty cycles}
\subsection{Arduino Model}
\begin{figure}[htbp]
   \centering
   \includegraphics{images/arduino_model.png}
   \caption{Sample flow of execution in Arduino.}
   \label{fig:arduino_model}
\end{figure}
The job of Arduino is defined in a single loop function, which clearly may invoke other functions anyway.
Such loop is executed repeatedly by a \ul{\textit{single} thread which is \textit{never} suspended}.
In case of an I/O operation, the program waits for the operation to complete, and then continues.

\subsection{TinyOS Model}

TinyOS offers functions (\textbf{commands}) to program and activate the hardware, it abstracts interrupts in form of \textit{events}.
It defines non-preemptive tasks that are executed sequentially, to manage different activities.

With this approach a task never waits, and can be pre-empted (only) by events. However interrupts should be handled as soon
as they arrive by means of an event handler, which should be as short as possible because other simultaneous interrupts would result in serious concurrency problems;
this is the case of the \texttt{read handler} in the figure \ref{fig:tinyos_model}, which instead of processing the data, it yields them to the ``upper-level'' task by means of a \texttt{post} command. 

\texttt{init} defines a timer which periodically triggers the \texttt{timer handler}, whose code may involve the operations to be performed periodically.

\begin{figure}[htbp]
   \centering
   \includegraphics{images/tinyos_model.png}
   \caption{Sample flow of execution in TinyOS.}
   \label{fig:tinyos_model}
\end{figure}

\section{Arduino}

Arduino is an open-source electronics prototyping platform based on flexible, easy-to-use hardware and software. Its concept comprises the actual Device, the IDE, and the online Forum.

It is now considered part of the IoT world, even though in its original design it was not meant to be connected to the internet, it was more \st{Internet-of-}``Things''.


\subsection{Interrupts}
Despite the main role of Arduino sketches is syncronous reading of sensors, it is possible to handle \textbf{interrupts}, allowing for asynchronous access to sensor data and actuators.
There are three types of interrupts in Arduino:
\begin{itemize}
   \item \textbf{External}:
   a signal outside Arduino coming from a pin
   \item \textbf{Timer}:
   internal Arduino timer 
   \item \textbf{Device}:
   internal signal coming from a device such as ADC, serial line, etc.
\end{itemize}

Internal interrupts (either Timer or Device) are managed by
the run time support of Arduino, external ones instead may be managed with an \lstinline|attachInterrupt(interrupt#, func-name,mode);| instruction.

There are only two external interrupts available on the Arduino Uno, \texttt{INT0} and \texttt{INT1}, which are mapped to pins 2 and 3.
They can be set to trigger on \texttt{RISING}, \texttt{FALLING}, \texttt{CHANGE}, \texttt{HIGH} or \texttt{LOW} level.
The trigger is interpreted by the hardware, and the interrupt is very fast.

\subsubsection*{Question on the slides}
\begin{lstlisting}
   void setup(){
      pinMode(2, INPUT);
      attachInterrupt(0, handler, CHANGE);
      return;
   }
   void handler(){}
   void loop(){
      return;
   }
\end{lstlisting}

\section{Energy Management}
\textit{Arduino Sleep Modes} to limit power consumption (aka \textit{Arduino Power Save} mode) can be activated by means of additional libraries, for example the \texttt{Low power} library.
The actual sleep mode mechanisms depend on the version of the device, and of the processor.
\lstinline|LowPower.idle()| is a function from the \texttt{Low power} library that puts the device in idle mode, which is the lowest power consumption mode available.
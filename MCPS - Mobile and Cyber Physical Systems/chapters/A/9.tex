\chapter{Embedded Programming}
The learning objectes of this chapter are embedded systems in general, and the Arduino case study.

Embedded systems are systems that are designed to perform a specific task, and they are usually part of a larger system.
Hardware and software are often designed together, aka ``hardware-software co-design''.
They are typically based on microcontrollers, optimized for controlling I/O.
They are often used in \ul{\textbf{real-time} systems, where timing is crucial}.

Many types of microcontrollers are availble on the market:
\begin{itemize}
   \item ``General purpose'', meaning that can be adapted to several embedded applications
   \item ``Application specific integrated circuit''(\texttt{ASIC}s) which are very efficient and performative, but tightly bound to a specific task
   \item ``System on a chip''(\texttt{SoC}s) which are a combination of a microcontroller and other components, like a radio module. 
   \note{The term is very broad, and it can refer to a wide range of devices.}
\end{itemize}

When programming on microcontrollers it must be taken into account the \textbf{small memory footprint}.  
\chapter{Responsible AI}
We produce an enormous amount of data while running
our daily activities.
How can we manage all these data? Can we get an added
value from them?

Data is gathered from domotics, streaming services, mobile apps, social media, audio sources, images, and so on. Such data is used in a \textit{Gather, Analyze, Gain insight, Influence, Optimize experiences,} (\textit{Sell} \smiley), \textbf{loop}.

AI is used in Healthcare, Big Data analytics, Social mining. It's the main tool for a Data Scientist to measure, understand and possibly predict \textbf{human behaviour}.\\
Data Scientist needs to take into account \textbf{ethical} and \textbf{legal}
aspects and social impact of data science and AI.

\section{European guidelines}

2019 European guidelines aim at setting the standard for AI use and development.
The ambition/goal is a \textbf{trustworthy} AI with three key components.
\begin{itemize}
   \item \textbf{Lawful AI} complying with all applicable laws and regulations
   \item \textbf{Ethical AI} ensuring adherence to ethical principles and values
   \item \textbf{Robust AI} perform in a safe, secure and reliable manner, both form technical and a social perspective, with safeguards to foresee and prevent unintentional harm
\end{itemize}

\begin{figure}[htbp]
   \centering
   \includegraphics{images/18/euAIproperties.png}
   \caption{AI requirements according to EU}
   \label{fig:18/euAIproperties}
\end{figure}

\begin{itemize}
	\item Human agency and oversight
	      \begin{itemize}
		      \item Fundamental rights
		      \item Human agency
		      \item Human oversight
	      \end{itemize}
	\item Technical robustness
	      \begin{itemize}
		      \item Resilience to attack and security
		      \item Safety
		      \item Accuracy
		      \item Reliability and reproducibility
	      \end{itemize}
	\item Privacy and data governance
	      \begin{itemize}
		      \item Privacy and data protection
		      \item Quality and integrity of data
		      \item Access to data
	      \end{itemize}
	\item Transparency
	      \begin{itemize}
		      \item Traceability
		      \item Explainability
	      \end{itemize}
	\item Diversity, non-discrimination and fairness
	      \begin{itemize}
		      \item Avoidance of unfair bias
		      \item Accessibility and universal design
		      \item Stakeholder Participation
	      \end{itemize}
	\item Societal and environmental well-being
	      \begin{itemize}
		      \item Sustainable and environmentally friendly AI
		      \item Social impact
		      \item Society and Democracy
	      \end{itemize}
	\item Accountability
	      \begin{itemize}
		      \item Minimisation and reporting of negative
		            impacts
		      \item Auditability
		      \item Minimisation and reporting of negative
		            impacts
		      \item Trade-offs
	      \end{itemize}
\end{itemize}

\section{Privacy and Data Protection}
\subsection{Types of Data}
\textbf{Personal data} is defined as any information relating to an identity or identifiable natural person.
It includes name, address, photo, email, bank account, \dots.

{\textbf{\textit{Sensitive} personal data} is a specific set of
``special categories'' that must be treated with
extra security:\ns
\begin{itemize}
	\item Racial or ethnic origin
	\item Political opinions
	\item Religious or philosophical beliefs
	\item Trade union membership
	\item Genetic data
	\item Biometric data
\end{itemize}}


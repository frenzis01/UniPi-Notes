\chapter{Decision Tree Simulation}

\begin{table}[h]
\centering
\begin{tabular}{|c|c|c|c|c|c|}
\hline
\textbf{index} & \textbf{state} & \textbf{contract} & \textbf{sex} & \textbf{calls} & \textbf{churn} \\
\hline
1 & it & C & F & >10 & Y \\
\hline
2 & ge & T & M & <=10 & N \\
\hline
3 & it & T & M & >10 & Y \\
\hline
4 & ge & Y & F & <=10 & N \\
\hline
5 & ge & T & F & >10 & N \\
\hline
6 & ge & C & M & >10 & Y \\
\hline
7 & ge & C & M & <=10 & Y \\
\hline
8 & ge & Y & F & <=10 & Y \\
\hline
9 & ge & Y & M & >10 & N \\
\hline
10 & ge & T & F & <=10 & N \\
\hline
11 & ge & Y & F & >10 & Y \\
\hline
12 & it & C & F & >10 & N \\
\hline
\end{tabular}
\caption{Customer churn dataset}
\label{tab:churn-dataset}
\end{table}

\section{Simulating DT}

Initial dataset statistics:
\begin{itemize}
   \item Y = 6
   \item N = 6 
   \item ME = 6/12
\end{itemize}

The formula for calculating the expected misclassification error after a split is given by:
\[
E = \sum_{i=1}^{k} ME(S_i) \cdot \frac{|S_i|}{|S|}
\]
where:
\begin{itemize}
	\item \(k\) is the number of subsets created by the split,
	\item \(ME(S_i)\) is the misclassification error of subset \(S_i\),
	\item \(|S_i|\) is the number of instances in subset \(S_i\),
	\item \(|S|\) is the total number of instances before the split.
\end{itemize}

The misclassification error for a subset is calculated as:
\[
ME(S_i) = 1 - \frac{\max(Y_i, N_i)}{|S_i|}
\]
where:
\begin{itemize}
   \item \(Y_i\) is the number of positive instances in subset \(S_i\),
   \item \(N_i\) is the number of negative instances in subset \(S_i\).
\end{itemize}

\subsection{Splitting by STATE}

\begin{table}[h]
\centering
\begin{tabular}{|c|c|c|}
\hline
 & \textbf{IT} & \textbf{GE} \\
\hline
\textbf{Y} & 2 & 4 \\
\hline
\textbf{N} & 1 & 5 \\
\hline
\end{tabular}
\caption{Split by STATE attribute}
\label{tab:split-state}
\end{table}

\[
\begin{aligned}
ME(IT) &= 1/3 = 1 - \max\left(\frac{2}{3}, \frac{1}{3}\right)\\
ME(GE) &= 4/9 = 1 - \max\left(\frac{5}{9}, \frac{4}{9}\right)\\
E &= ME(IT)\cdot p(IT) + ME(GE) \cdot p(GE) \\
E &= \frac{1}{3} \cdot \frac{3}{12} + \frac{4}{9} \cdot \frac{9}{12} = \frac{5}{12}
\end{aligned}
\]


\subsection{Splitting by CONTRACT}

\subsubsection{3-way split: C, T, Y}

\begin{table}[h]
\centering
\begin{tabular}{|c|c|c|c|}
\hline
 & \textbf{C} & \textbf{T} & \textbf{Y} \\
\hline
\textbf{Y} & 3 & 1 & 2 \\
\hline
\textbf{N} & 1 & 3 & 2 \\
\hline
\end{tabular}
\caption{3-way split by CONTRACT attribute}
\label{tab:split-contract-3way}
\end{table}

\[
\begin{aligned}
ME(C) &= 1/4 \\
ME(T) &= 1/4 \\
ME(Y) &= 2/4 \\
E &= \frac{1}{4} \cdot \frac{4}{12} + \frac{1}{4} \cdot \frac{4}{12} + \frac{2}{4} \cdot \frac{4}{12} = \frac{4}{12}
\end{aligned}
\]

\subsubsection{2-way split: CT, Y}

\begin{table}[h]
\centering
\begin{tabular}{|c|c|c|}
\hline
 & \textbf{CT} & \textbf{Y} \\
\hline
\textbf{Y} & 4 & 2 \\
\hline
\textbf{N} & 4 & 2 \\
\hline
\end{tabular}
\caption{2-way split by CONTRACT: CT vs Y}
\label{tab:split-contract-2way-cty}
\end{table}

\[
\begin{aligned}
ME(CT) &= 4/8 \\
ME(Y) &= 2/4 \\
E &= \frac{4}{8} \cdot \frac{8}{12} + \frac{2}{4} \cdot \frac{4}{12} = \frac{6}{12}
\end{aligned}
\]

\subsubsection{2-way split: C, TY}

\begin{table}[htbp]
\centering
\begin{tabular}{|c|c|c|}
\hline
 & \textbf{C} & \textbf{TY} \\
\hline
\textbf{Y} & 3 & 3 \\
\hline
\textbf{N} & 1 & 5 \\
\hline
\end{tabular}
\caption{2-way split by CONTRACT: C vs TY}
\label{tab:split-contract-2way-cty2}
\end{table}

\[
\begin{aligned}
ME(C) &= 1/4 \\
ME(TY) &= 3/8 \\
E &= \frac{1}{4} \cdot \frac{4}{12} + \frac{3}{8} \cdot \frac{8}{12} = \frac{4}{12}
\end{aligned}
\]

\subsubsection{2-way split: CY, T}

\begin{table}[htbp]
\centering
\begin{tabular}{|c|c|c|}
\hline
 & \textbf{CY} & \textbf{T} \\
\hline
\textbf{Y} & 5 & 1 \\
\hline
\textbf{N} & 3 & 3 \\
\hline
\end{tabular}
\caption{2-way split by CONTRACT: CY vs T}
\label{tab:split-contract-2way-cyt}
\end{table}

\[
\begin{aligned}
ME(CY) &= 3/8 \\
ME(T) &= 1/4 \\
E &= \frac{3}{8} \cdot \frac{8}{12} + \frac{1}{4} \cdot \frac{4}{12} = \frac{4}{12}
\end{aligned}
\]

\subsection{Splitting by SEX}

\begin{table}[htbp]
\centering
\begin{tabular}{|c|c|c|}
\hline
 & \textbf{F} & \textbf{M} \\
\hline
\textbf{Y} & 3 & 3 \\
\hline
\textbf{N} & 4 & 2 \\
\hline
\end{tabular}
\caption{Split by SEX attribute}
\label{tab:split-sex}
\end{table}

\[
\begin{aligned}
ME(F) &= 3/7 \\
ME(M) &= 2/5 \\
E &= \frac{3}{7} \cdot \frac{7}{12} + \frac{2}{5} \cdot \frac{5}{12} = \frac{5}{12}
\end{aligned}
\]

\subsection{Splitting by CALLS}

\begin{table}[htbp]
\centering
\begin{tabular}{|c|c|c|}
\hline
 & \textbf{>10} & \textbf{<=10} \\
\hline
\textbf{Y} & 2 & 4 \\
\hline
\textbf{N} & 3 & 3 \\
\hline
\end{tabular}
\caption{Split by CALLS attribute}
\label{tab:split-calls}
\end{table}

\[
\begin{aligned}
ME(>10) &= 2/5 \\
ME(<=10) &= 3/7 \\
E &= \frac{2}{5} \cdot \frac{5}{12} + \frac{3}{7} \cdot \frac{7}{12} = \frac{5}{12}
\end{aligned}
\]

\section{Sequential Pattern Mining}

\subsection{Exercise Setup}

Given the sequential rule: \(\{B\} \rightarrow \{CD\}\) with constraint MIN-GAP = 1.

For each of the sequences below, list all the subsequences that satisfy the constraint (if any).

\subsection{Sequence 1}

\begin{center}
\begin{tabular}{ccccc}
\{ABF\} & \{C\} & \{CDF\} & \{E\} & \{CD\} \\
$t_0$ & $t_1$ & $t_2$ & $t_3$ & $t_4$ \\
\end{tabular}
\end{center}

\textbf{No constraints:}
\begin{itemize}
   \item $\langle t_0, t_2 \rangle$
   \item $\langle t_0, t_4 \rangle$
\end{itemize}

\textbf{With min-gap = 1:}
\begin{itemize}
   \item $\langle t_0, t_2 \rangle$
   \item $\langle t_0, t_4 \rangle$
\end{itemize}

\subsection{Sequence 2}

\begin{center}
\begin{tabular}{cccc}
\{AB\} & \{C\} & \{AB\} & \{CD\} \\
$t_0$ & $t_1$ & $t_2$ & $t_3$ \\
\end{tabular}
\end{center}

\textbf{No constraints:}
\begin{itemize}
   \item $\langle t_0, t_2 \rangle$
   \item $\langle t_0, t_3 \rangle$
\end{itemize}

\textbf{With min-gap = 1:}
\begin{itemize}
   \item $\langle t_0, t_3 \rangle$
\end{itemize}

\subsection{Sequence 3}

\begin{center}
\begin{tabular}{ccccc}
\{AF\} & \{BC\} & \{AB\} & \{E\} & \{D\} \\
$t_0$ & $t_1$ & $t_2$ & $t_3$ & $t_4$ \\
\end{tabular}
\end{center}

\textbf{No constraints:}
\begin{itemize}
   \item None
\end{itemize}

\textbf{With min-gap = 1:}
\begin{itemize}
   \item None
\end{itemize}

\subsection{Sequence 4}

\begin{center}
\begin{tabular}{ccccc}
\{A,B,F\} & \{A,C\} & \{A,B,D\} & \{C\} & \{C,D\} \\
$t_0$ & $t_1$ & $t_2$ & $t_3$ & $t_4$ \\
\end{tabular}
\end{center}

\textbf{Rule:} \(\{A\} \rightarrow \{C,D\}\)

\textbf{No constraints:}
\begin{itemize}
   \item $\langle t_0, t_4 \rangle$
   \item $\langle t_1, t_4 \rangle$
   \item $\langle t_2, t_4 \rangle$
\end{itemize}

\textbf{Rule:} \(\{A\} \rightarrow \{D\}\)
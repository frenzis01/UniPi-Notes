\chapter{Time series}
% // TODO first 23 slides

\section{Sinusoids}
Fourier analysis is a method to decompose a time series into a sum of sinusoidal functions, each characterized by a specific frequency, amplitude, and phase. This decomposition allows us to analyze the frequency components of the time series, which can be useful for identifying periodic patterns, trends, and noise.

\textbf{Sinusoids} are mathematical functions that describe smooth, periodic oscillations. 
They are $sin/cos$ functions defined by a \textbf{phasor} $\psi$.
Sinusoids define \textbf{amplitude}, which is the height of the wave, and \textbf{frequency}, which is how many cycles occur in a unit of time.
At a time $t$, the phasor defines a series component with amplitude
\[
s_t = 
\underbrace{\alpha}_{\text{amplitude scaling}}
\,
\cos(
\overbrace{2\pi f_0 t}^{\text{angle } \theta_{0,t}}
\underbrace{+\phi}_{\text{shifting}}
).
\]


\chapter{Seminario - Imola informatica}
\section{Takeaway Messages}
\textit{Performance} per se is not an accurate measure,
there are many factors when developing SW systems which affect perfomance,
like usability and efficiency.

\section{Project Path}
\begin{equation}
\label{eq:old_project_path}
\textbf{Demand} \longrightarrow \textbf{Plan} \longrightarrow \textbf{Design} \longrightarrow \textbf{Develop} \longrightarrow \textbf{Release}\\
\end{equation}
This (sadly not) deprecated path \ref{eq:old_project_path} leads to \textit{situation rooms} and subsequent performance degradation, unsatisfaction and possible skyrocketing costs.

Performance should drive the whole production process, it shouldn't be treated as a post-go live concern,
otherwise it may lead to the so called \textit{situation rooms}\footnote{Often named also \textit{war rooms}}.

\section{Fitness function}
A \textbf{fitness function} provides a summarised measure of how close a given design solution is to achieving the set aims.

\section{Performance Best practices}
"Starbucks does not use two-phase commit": they aim to maximize throughput, by using an employee chain to serve customers, from ordering to delivering coffee.

Enforce business process performance with adequate fitness functions:
\begin{itemize}
    \item involve key stakeholders
    \item automatically assess and evaluate
    \item continuously review and tune
\end{itemize}

It is important to design IT architectures and solutions with real-world requirements in mind.
For example "a customer shouldn't have to wait for more than 2s to \textit{order} a coffee".\\
In distributed architectures, network's technical aspects and metrics must be taken into account: latency, available bandwidth, dedicated or shared, network billing models...\\
Aside from requirements, also costs, performance and observability should be kept in mind.\\
To measure progress fitness function must be fed periodically with real-time data.
\nl
Most of the times testing only in production is the only way to go,
since mirroring the production environment and using/managing it during development would be hugely costful.
However, precisely for this reason, production testing shouldn't be the only testing method.

\section{Bad habits}
\begin{itemize}
    \item Worrying about performance only late in development
    \item Last minute testing
    \item Focusing only on performance as a technical POV,
    not user/business pov
\end{itemize}

Enable a culture for performance across your entire value stream and
embed it in business processes as well as IT systems.\nl

\subsubsection{Takeaway message}
\textit{\textbf{Evolutionary} architectures} need \textbf{fitness functions} and it is mandatory to
continuosuly refine \textbf{fitness functions}.
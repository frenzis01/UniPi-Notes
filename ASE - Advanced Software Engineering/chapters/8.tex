\chapter{* as a Service}

In traditionally distributed software,
customers had to configure, manage updates, while producers had to mantain different product versions; 
in \textbf{Saas} instead,
a product is delivered as a \textbf{service}, thus there's no need to install anything,
and there is a form of monthly (and/or pay-per-usage) subscription, to use the service.

\section{Benefits and cons}
\textit{\textbf{Producer} point of view}

\labelitemize{
   \color{darkgreen}
   \textit{\textbf{Pros}}
}{
   \begin{itemize}
      \color{darkgreen}
      \item Regular cash flow
      \item Easier and less costful update management
      \item Continuous deployment: a new software version can be deployed as soon as it is tested
      \item Payment flexibility, making room for different subscription plans possibly attracting a wider range of users
      \item Try-before-you-buy options are easily available without fearing piracy, besides they'd make the product look more appealing to new customers
      \item Telemetry and data collection are way easier and hardly avoidable by customers   
   \end{itemize}
   }\\
\textit{\textbf{Consumer} point of view}
\labelitemize{
   \color{darkgreen}
   \textit{\textbf{Pros}}
}{
\begin{itemize}
   \color{darkgreen}
   \item Mobile, laptop and desktop access
   \item No \textbf{upfront costs} for software or servers
   \item Immediate and transparent software updates
   \item Reduced software \textbf{management costs}
\end{itemize}
}

\labelitemize{
   \color{darkred}
   \textit{\textbf{Cons}}
}{
\begin{itemize}
   \color{darkred}
   \item \textbf{Privacy} regulation conformance
   \item \textbf{Security} concerns
   \item Network constraints may limit the usability of the software
   \item Data exchange from/to services might be difficult if the service doesn't provide a suitable API
   \item No control over updates
   \item Service \textbf{lock-in}
\end{itemize}
}

\section{Design issues}
When designing Saas there are some critical points the developers has to make decisions on:
\begin{itemize}
   \item \textbf{Authentication} method: federated auth, personal auth, \textit{Google/Linkedin/...} credentials... 
   \item Whether some features should be available \textbf{locally}
   \item \textbf{Info leakage}
   \item \textit{Multi-tenant} vs \textit{multi-instance} \textbf{DB management}:
   i.e. single repository vs separate copies of system and database
\end{itemize}

\subsection{DB management}
\subsubsection{Multi-tenant}
\textbf{Multi-tenant} systems forsee a single DB schema shared by all system's users,
where DB items are tagged with a \textit{tenant identifier} to provide some form of "logical isolation".

Mid-size and large businesses rarely want to use a generic multi-tenant software, 
they often prefer a customized version adapted to their own requirements.
Generally it is of interest to have a custom UI mutating according to the user,
along with custom optional fields accessible only by some classes of users.
In case there's the need to expand the DB schema to achieve such results,
\textbf{storage waste} becomes a major concern.


\textbf{Security} is the makor concern of corporate customers with multi-tenant systems,
since a centralized DB may represent a single point-of-failure for data leak or damage.\\
A common solution is to implement \textbf{multilevel access control}, checking data access both at the organizational level and at individual level.
A technlogy which can clearly help in this matter is \textbf{encryption},
however it might cripple performance.

\subsubsection{Multi-instance}
Multi-instance may mainly be \textbf{VM}-based or \textbf{Container}-based.\\
With \textit{containers}, each user has an \textbf{isolated} version of software and database running in a set of container,
defining a solution perfect for products whose users work onto independently from others.
Besides, since there is no direct sharing of a single data structure, 
many security aspects are easy to manage.\\
With the other solution instead, for each \textbf{customer} there is a VM running an instance of the DB \textbf{shared} and accessible to all customer's users.\nl

Let's consider \textit{pros and cons} of such a solution.
Flexibility, security, scalability and resilience are clearly some key points of \textit{multi-instance DBs}.
However update management difficulty and cloud VMs renting costs are not negligible.

Wrapping up,
there are three possible ways of
providing a customer \textbf{database} in
a \textit{cloud-based} system:
\begin{enumerate}
   \item As a \textbf{multi-tenant} system, \textit{shared by all customers} for your product.
   This may be hosted in the cloud using large, powerful servers. 
   \item As a \textbf{multi-instance} system, with \textit{each customer database} running on its own \textit{virtual machine}. 
   \item As a \textbf{multi-instance} system, with \textit{each database} running in its own \textit{container}.
   The customer database may be distributed over several containers.
\end{enumerate}

\section{Architectural decisions}
\begin{center}
   \begin{tikzpicture}[main/.style = {draw, ellipse},node distance={3cm},>=stealth',shorten >=1pt,auto] 
      \node[main] (P) [align=center]{\textbf{What cloud platform fits best} \\ \textbf{development and delivery?}};
      \node[main] (a) [align=center,above left=1.5cm of P] {\textit{Scalability and} \\ \textit{Resilience}};
      \node[main] (b) [align=center, above=1.5cm of P] {\textit{Monolithic/Service} \\ \textit{oriented}};
      \node[main] (c) [align=center, above right=1.5cm of P] {\textit{Multi-tenant or} \\ \textit{Multi-instance DB}};
   
      \draw[->] (a) edge [out=-90, in=150] node {} (P);
      \draw[->] (b) edge node {} (P);
      \draw[->] (c) edge [out=-90, in=30] node {} (P);
      
   \end{tikzpicture}
\end{center}

\subsection{Scalability}
To allow scalability on cloud-based systems,
the product implementation must be organized so that the individual software components can be replicated and run in parallel,
besides, also a load-balancing mechanism must be implemented.

\subsection{Resilience}
Achieving resilience can be done through \textit{hot/cool standby}.
The difference lies in the fact that while cool standby relies restarting the system using backup copies of the data,
hot standby forsees a clone-instance running at the same time of the main one with a mirrored DB:
in case of system failure,
there is and entire backup system which can take its place while the main recovers.
This is clearly more costful, but more effective.

\section{Choosing cloud platform}
\begin{center}
   \begin{tikzpicture}[main/.style = {draw, ellipse},node distance={3cm},>=stealth',shorten >=1pt,auto] 
      \node[main] (P) [align=center]{\textbf{Cloud-platform} \\ \textbf{choice}};
      \node[main] (a) [align=center,above left=1.5cm of P] {\textit{Privacy and Data} \\ \textit{Protection}};
      \node[main] (c) [align=center, above right=1.5cm of P] {\textit{Expected load and} \\ \textit{laod predictability}};
      \node[main] (b) [align=center, below left=1.5cm of P] {\textit{Supported cloud} \\ \textit{services}};
      \node[main] (d) [align=center, below right=1.5cm of P] {\textit{Resilience}};
   
      \draw[->] (a) edge node {} (P);
      \draw[->] (b) edge node {} (P);
      \draw[->] (c) edge node {} (P);
      \draw[->] (d) edge node {} (P);
      
   \end{tikzpicture}
\end{center}

\begin{center}
   \begin{tikzpicture}[main/.style = {draw, ellipse},node distance={3cm}] 
      \node[main] (P) [align=center]{\textbf{Business issues when} \\ \textbf{choosing platform}};
      \node[main] (a) [align=center,above left=0.7cm and 2.5cm of P] {\textit{Cost}};
      \node[main] (b) [align=center, above right=0.7cm and 2.5cm of P] {\textit{Development} \\ \textit{experience}};
      \node[main] (c) [align=center, below right= 0.7cm and 1.5cm of P] {\textit{Service-level} \\ \textit{agreements}};
      \node[main] (d) [align=center, below left=0.7cm and 1.5cm of P] {\textit{Portability and} \\ \textit{cloud migration}};
      \node[main] (e) [align=center, above=1.5cm of P] {\textit{Target customers}};
   
      \draw[->] (P) -- (a);
      \draw[->] (P) -- (b);
      \draw[->] (P) -- (c);
      \draw[->] (P) -- (d);
      \draw[->] (P) -- (e);
      
   \end{tikzpicture}
\end{center}
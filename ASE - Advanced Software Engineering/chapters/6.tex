\chapter{Enterprise Applications}
In Enterprise applications there heterogeneous services, data sources and participants, all connected via network.\\
In short, \textbf{Enterprise Applications} are \textit{complex distributed multi-service applications} whose services must work together, them being suitably \textbf{integrated}.

\section{Integration}
The architectural question is how to integrate multiple different services to realize enterprise applications that are
\begin{itemize}
   \item \textit{coherent}
   \item \textit{extensible}
   \item \textit{maintainable}
   \item (reasonably) simple to \textit{understand}
\end{itemize}

This is really what enterprise application integration was conceived for
\begin{itemize}
   \item complexity management
   \item change management
   \item \textbf{pattern-based}
\end{itemize}

\textbf{\textit{Pattern}} refers to high-level abstraction of accepted, reusable solutions to recurring problems.\\
When facing a problem, considering existing patterns that are applicable to
solve such problem saves us from re-inventing the wheel and making
the same mistakes as others

Typically, patterns are given in terms of
\begin{itemize}   
   \item \textbf{problem statement}
   including involved software components
   \item \textbf{context}
   including involved actors
   \item \textbf{forces}
   clarifying the problem rationale and importance
   \item \textbf{solution}
   given abstractly, and independent of its actual implementations
\end{itemize}


An \textbf{EIP} (\textit{Enterprise Integration Pattern}) is a reusable abstraction of proven solutions to well-known problems raising while integrating the software components/services forming enterprise applicatioms.

\section{Messages and Communication Means}
A \textbf{message} is a discrete piece of data sent from a service to another, typically structured into header and body,
and generally sent through \textbf{one-way channels}.
Thus by itself the communication is \textit{\underline{a}synchronous}, but it can be made \textit{synchronous} by duplicating a channel and thus creating a bidirectional communication.

Channels supports extends to two main categories:
\begin{itemize}
   \item \textbf{Point-to-Point} ($1:1$) channels ensure that only one receiver will receive a given message
   \item \textbf{Publish-Subscribe} ($1:N$) channels deliver a copy of the message to each receiver
\end{itemize}

Channels generally require the data to serialized in some way: 
to this extent \textbf{adapters} "translate" (\textit{adapt}) application-specific data into a format suitable to be sent;
note that adapters are place between the \textit{application} and the \textit{channel}, not the \textit{receiver}.\\
Besides, a \textbf{message endpoint} for each node/application is required,
to handle channel connections and queue/handling the messages received before letting the application process them.\nl

\textbf{Message endpoints} and \textbf{channels} provide the most trivial integration possible,
however they do not solve the problem when the \textit{receiver application} requires the data to be in a specific format e.g. \texttt{JSON}.
\textbf{Message translators} are intermediary entities which can translate and eventually filter data.

\section{Deeper message integration}
Some problems are still not solved e.g. message \textit{routing}, \textit{splitting}, \textit{aggregating}, etc.\nl

That's where \textbf{pipes} and \textbf{filters} architecture style comes in.\\
Messages travel through many \textit{filters} (and components) processing them.
Components flush messages into \textbf{pipes} they are connected to.\\
Clearly this is (again) a flexible abstraction which can adapt to circumstances and environment.

This architecture, along with endpoints and channels,
includes \textbf{Content Enrichers}, which add contextual information which the source may not have, \textbf{Routers} which may be \textit{content-based} (header/body) or \textit{context-based} (testing/production env).\\
Speaking of \textbf{routers}, such components are connected to multiple channels and contain the logic to discern which is the correct channel onto which a message should be sent.

\subsection{Composite Patterns}
Patterns may be composed to build other patterns.
\textbf{Normalizers}, for instance, enable data received from different sources to be normalized and then sent in a proper format to receivers;
behind the curtain, a \textit{normalizer} is nothing more than a router and a set of translators.

\subsection{Parallelism}
Sometimes some integration steps may be computed in \textbf{parallel} and then results from such processes may be \textbf{aggregated} to decide which actions to perform.

\textbf{Splitters} break out composite messages into a series of individual messages, which can be \textit{processed independently}.\\
\textbf{Aggregators} collect and store individual messages until a complete set of related messages has been received,
ultimately publishing a single message which be processed as a whole. 


\section{Handling Problems}
\begin{itemize}
   \item \textbf{Validate} messages
   \item \textbf{Architectural smells} and \textbf{refactoring}
   \item \textbf{Security} issues
   \item \textbf{Isolate} features in integrated applications
   \item \textbf{Deploy serverlessly} integrated applications
\end{itemize}
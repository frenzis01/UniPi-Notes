\chapter{Deadlocks}

\begin{definition}
   [Deadlock]
   A deadlock occurs when a set of processes is waiting for resources held by other processes in the set, resulting in a circular wait where no process can proceed.
\end{definition}

{There are Four \textit{Necessary} Conditions (\textit{Coffman} Conditions) in order to have a deadlock:\ns
\begin{itemize}
	\item \textbf{Mutual Exclusion}: Resources cannot be shared.
	\item \textbf{Hold and Wait}: Processes holding resources can request new ones.
	\item \textbf{No Preemption}: Resources cannot be forcibly taken away.
	\item \textbf{Circular Wait}: A closed chain of processes exists, where each holds a resource needed by the next.
\end{itemize}
If any of the previous conditions is not met, a deadlock cannot occur.}

{There are three main approaches to handle deadlocks, which act at different points in the deadlock cycle, but in reality the last one is the only true and available solution; Prevention and Avoidance typically come at a way too high cost:\ns
\begin{itemize}
   \item \textbf{Prevention}: Ensures that at least one of the necessary conditions does not hold.
   \begin{itemize}
      \item Acquire all resources before starting.
      \item Preempt a process holding a needed resource.
   \end{itemize}
   This approach typically leads to \ul{resource underutilization and low throughput}.
   \item \textbf{Avoidance}: Checks dynamically if granting a resource would lead to a deadlock.
   \begin{itemize}
   	\item Banker's Algorithm (used in centralized systems).
	   \item Safe state evaluation in distributed environments.
   \end{itemize}
   Avoidance requires \textit{global state awareness}.
   \item \textbf{Detection and Recovery}: Identifies a deadlock state and then recovers from it.
   \begin{itemize}
      \item Once detected, deadlocks can be resolved by aborting one or more processes or preempting resources
   \end{itemize}
\end{itemize}}

\section{RAG - Resource Allocation Graph}
The Resource Allocation Graph is a directed graph that models the allocation of resources to processes. It is composed of two types of nodes, processes and resources, and two types of edges, request and assignment. 
A cycle in the graph indicates a deadlock.

\subsection{WFG - Wait-For Graph}
A Wait-For Graph is a directed graph, a simplified RAG, that models the wait-for relationship between processes. It is composed of processes as nodes and edges representing the wait-for relationship. A cycle in the graph indicates a deadlock.

\section{Detecting Deadlocks}
\subsection{Centralized Deadlock Detection}
A central node gathers information from all other nodes and constructs a global wait-for graph.
It is very simple, but not scalable, and the node can become a single point of failure.
\subsection{Hierarchical Deadlock Detection}
The system is divided into regions, with each region responsible for local deadlock detection.  Regional coordinators communicate with higher-level nodes for global detection.

This approach provides better scalability than centralized detection, but, on the other hand, coordination between regions can be complex.
\subsection{Distributed Deadlock Detection}

Here all nodes participate in deadlock detection by sharing partial information about resources and processes, and each node detects deadlocks locally and communicates with others to detect global deadlocks.

There is no single point of failure, and the approach scales well with larger systems, but poses more complex message passing and increased overhead.

\section{Resolving Deadlocks}
\begin{itemize}
	\item \textbf{Process Termination} - Abort one or more processes involved in the deadlock.
	\item \textbf{Resource Preemption} - Forcefully take a resource from one process to give it to another.\\
	It is complex to implement without causing inconsistencies. It requires the ability to roll back (restore) the state of the preempted process.
	\item \textbf{Rollback} - Roll back the state of one or more processes to a safe point before the deadlock occurred.
\end{itemize}

\subsection{System Model for Deadlock Detection}
The distributed system is composed of asynchronous processes that communicate via message passing.
A WFG models the state of the system.

\begin{itemize}
	\item Nodes:
	\begin{itemize}
      \item 
      Processes.
   \end{itemize}
	\item Edges:
	\begin{itemize}
      \item 
      Directed edges from P1 to P2 indicate
      that P1 is waiting for P2 to release a
      resource.
   \end{itemize}
	\item Deadlock occurs when there’s a cycle in the WFG
\end{itemize}
\labelitemize{
   Assumptions
}{
   \begin{itemize}
      \item Only reusable resources.
      \item Exclusive resource access.
      \item One copy of each resource.
      \item Two process states: Running (active) or Blocked (waiting for resources).
   \end{itemize}
}

Thew challenges in this system are \textbf{WFG maintenance}, which involves tracking resource dependencies across distributed nodes, and \textbf{cycle detection}, which involves searching cycles in the WFG. 
In particular, such search should find \textit{all} cycles in \textit{finite} time. Besides no false ---aka \textit{phantom}--- deadlocks should be reported.

{Having the Deadlocks detected, the system can then proceed to resolve them by \textbf{rollback}, \textbf{preemption}, or \textbf{termination}.\ns
\begin{itemize}
	\item Break the wait-for dependencies between deadlocked processes.
	\item Roll back or abort one or more processes to free up resources.
	\item Clean up the wait-for graph after resolution to avoid phantom deadlocks.
	\item Untimely and inappropriate cleaning of broken wait-for dependencies is the main reason why many deadlock detection algorithms reported in the literature are incorrect
\end{itemize}}

\section{Deadlock Models}
\subsection{Single Resource Model}
In this model, each process requires a single resource to proceed. The deadlock detection algorithm is simple and efficient, but it is not very realistic.
Since the maximum out-degree of a node in a WFG for the single resource model can be 1, the presence of a cycle in the WFG shall indicate that there is a deadlock

\subsection{AND Model}
A process can request multiple resources simultaneously, and the request is granted only if all resources are available.
A cycle in the WFG ---gues what--- implies a deadlock.\\
In the AND model, if a cycle is detected in the WFG, it implies a deadlock but \textit{\textbf{not} vice versa}. That is, a process may not be a part of a cycle, it can still be deadlocked.

\subsection{OR Model}
In the OR model, a process can make a request for numerous resources simultaneously and the request is satisfied if any one of the requested resources is granted.\\
Presence of a cycle in the WFG of an OR model does \textbf{not} imply a deadlock in the OR model.

In the OR model, the presence of a \textbf{knot} indicates a deadlock. In a WFG,a vertex $v$ is in a knot if for all $u::u$ is reachable from $v:v$ is reachable from$u$. No paths originating from a knot shall have dead ends.

Note that, \ul{there can be a deadlocked process that is \textbf{not} a part of a knot}.\\
So, in an OR model, a blocked process $P$ is deadlocked if it is either in a knot or it can only reach processes on a knot.

\section{Deadlock Detection Algorithms}
\subsection{Path-Pushing Algorithm}
Distributed deadlocks are detected by
maintaining an explicit global WFG
\begin{itemize}
	\item to build a global WFG for each site of the
distributed system
	\item Nodes propagate local WFG to other nodes.
	\item After the local data structure of each site is
updated, WFG are updated
	\item Deadlocks are detected by identifying circular
dependencies.
\end{itemize}

\subsection{Edge-Chasing}
Here, a process sends probes to check if it is part of a deadlock.
Whenever a process that is executing receives a probe message, it simply discards this message and continues. Only blocked processes propagate probe messages along their outgoing edges. If the probe returns to the initiating process, a deadlock is detected.

\begin{itemize}
	\item P1 sends a \textit{probe} to P2, which forwards it to P3.
	\item When the \textit{probe} returns to P1, a deadlock is detected
\end{itemize}

A probe is a message used in edge-chasing algorithms to \ul{trace the path of resource dependencies}.

\subsection{Diffusion Computation Algorithm}
\begin{itemize}
	\item An algorithm for deadlock detection based on diffusion computation.
	\item When a process is blocked, it propagates queries to other processes, otherwise it drops queries
	\item Queries are discarded by a running process and are echoed back by blocked processes in the following way:
   \begin{itemize}
   	\item When a blocked process first receives a query message for a particular deadlock detection initiation, it does not send a reply message until it has received a reply message for every query it sent
   	\item to its successors in the WFG).
   	\item For all subsequent queries for this deadlock detection initiation, it immediately sends back a reply message.
   	\item The initiator of a deadlock detection detects a deadlock when it has received a reply for every query it has sent out.
   \end{itemize}
\end{itemize}

\subsection{Global state detection-based algorithms}

exploit the following facts:
\begin{itemize}
	\item a consistent snapshot of a distributed system can be
   obtained without freezing the underlying computation,
   and
	\item a consistent snapshot may not represent the system
   state at any moment in time, but if a stable property
   holds in the system before the snapshot collection is
   initiated, this property will still hold in the snapshot.
\end{itemize}

Therefore, distributed deadlocks can be detected by taking
a snapshot of the system and examining it for the condition
of a deadlock



\subsection{Deadlock Detection in Dynamic and Real-time Systems}

In distributed systems with dynamic resources (e.g., cloud systems), resource requests and releases are constantly changing. Deadlock detection algorithms must adapt to these changes and avoid outdated information.

This clearly poses consistent challenges such as false detections, the need for constant
updates, and handling frequent resource changes.


In real-time distributed systems, deadlocks must be detected and resolved quickly to meet timing constraints.\\
Detection algorithms must have low latency and minimal overhead

\section{Assignment}

Study and eventually prepare a few slides on
\begin{itemize}
   \item Misra-Chandy-Haas Algorithm for AND model
   \item Misra-Chandy-Haas Algorithm for OR model
\end{itemize}
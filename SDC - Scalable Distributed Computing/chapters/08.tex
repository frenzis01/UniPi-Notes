\chapter{Raft}


\section{Logs use case}
Bear in mind that a \ul{\textbf{log} is a \textit{sequence} of records}, it is meaningless if not ordered.
Operations have an effect when executed in an order $s$, and may have a different one in another order $s'$.

In distributed systems, it is not trivial to keep logs consistent.
On local machines, timestamp ordering is enough, but in distributed systems, clocks are not synchronized, hence we need a more sophisticated approach, typically involving the sequencing of operations, not necessarily based on timestamps.

\section{Consesus - (Again)}
Consesus is the backbone of consistency, fault tolerance, and coordination in distributed systems;
it enables systems to operate reliably and predictably, even in the presence of failures and network partitions.

\subsection{Why is it relevant}
%  // TODO

\framedt{Converging}{
   Amongst other things, consesus is used to ensure \textit{\textbf{data integrity}, i.e. prevents \textbf{data divergence} in replicated systems}.
   That is that different data (replicas) will eventually converge to the same value.

   \textit{But to which of the replicas should the others converge?}\\
   Consesus!   

}

% \\ TODO

\framedt{Paxos and paper}{
   Even though it was a standard until a few years ago, people realized that Paxos was too complex to implement, and could fit best raw paper than actual software.
   Too many aspects were left to the implementer, and it was hard to get it right.
}

\section{Raft}
Raft achieves consesus through an \textit{elected leader} that coordinates the other nodes.
An entity participating to a Raft cluster can either take the role of the leader or the role of follower;
in the latter case, it may become leader through a ``candidacy'' process.\\
The leader regularly informs the followers of its existence by sending a heartbeat message.
\begin{itemize}
   \item There exist a timeout for the heartbeats from the leader
   \item In case no heartbeat is received the follower changes its status to candidate and starts a leader election.
\end{itemize}

Note that the \ul{\textbf{underlying assumption} is that \textit{everybody knows everybody}}.

\subsection{Key points}
\begin{itemize}
   \item \textbf{Leader election} - Raft uses a randomized timeout to elect a leader.
   \item \textbf{Log replication} - The leader replicates its log to the followers.
   \item \textbf{Safety and Liveness} - Raft ensures safety and liveness under failure conditions.
\end{itemize}

\subsection{Log Replication}
Log replication is a fundamental mechanism in Raft to ensure that all nodes in the cluster have the same data.

The leader is responsible for replicating its log to the followers. Such log contains a series of commands that must be applied in the same order by all nodes in the cluster.

Only the leader can append entries to the log, and each entry is committed once it is replicated to a majority of the nodes.
\nl

In case the leader crashes it is important to ensure that a new one is elected, and that it has the same log as the previous one.

% // TODO

\subsection{Leader election}
There are three states in Raft:
\begin{itemize}
   \item \textbf{Leader} - The leader is responsible for managing the replication of the log.
   \item \textbf{Follower} - The follower replicates the leader's log.
   \item \textbf{Candidate} - The candidate is a node that is trying to become the leader.
\end{itemize}

\framedt{Becoming the leader}{
   \begin{enumerate}
      \item When the current leader fails (e.g. crashes or becomes unreachable), a new leader is elected.
      \item Follower nodes can become candidates and initiate electionss if they don't receive a heartbeat from the leader within a specified time (election timeout).
      \item Leader election is the foundation of Raft's fault tolerance.
   \end{enumerate}
}

Once a leader is elected, the leader appends new log entries and replicates them to all followers.
A log entry is considered committed when it is replicated to a majority of nodes. Followers always accept entries from the current leader to maintain consistency.

\tikzset{every loop/.style={min distance=10mm,in=50,out=490,looseness=10}}
\begin{figure}[htbp]
   \centering
   \begin{tikzpicture}[
      state/.style={rectangle, rounded corners, draw=black, very thick, minimum height=3em, inner sep=3pt, text centered},
      arrow/.style={-{Latex[length=3mm, width=2mm]}, thick},
      every node/.style={font=\sffamily}
  ]
  
  % States
  \node[state] (follower) {Follower};
  \node[state, right=4cm of follower] (candidate) {Candidate};
  \node[state, right=4cm of candidate] (leader) {Leader};
  
  % Arrows
  \draw[arrow] (follower) -- node[above] {Suspects leader failure} (candidate);
  \draw[arrow] (candidate) -- node[above] {win election} (leader);
  \draw[arrow, bend left] (leader) to node[above] {heartbeat timeout} (follower);
  \draw[arrow, bend left] (leader) to node[below] {step down} (follower);
  \draw[arrow, loop above] (candidate) to node[above] {election timeout} (candidate);
  
  \end{tikzpicture}
   \caption{State diagram}
   \label{fig:state_diagram}
\end{figure}
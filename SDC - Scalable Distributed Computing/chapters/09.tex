\chapter{CAP Theorem}
\begin{definition}
   [CAP Theorem]
   It is impossible for a distributed data store to simultaneously provide more than two out of the following three guarantees:
   \begin{itemize}
   	\item \textbf{Consistency}\\
            Every read receives the most recent write or an error
   	\item \textbf{Availability}\\
            Every request receives a (non-error) response, without the guarantee that contains the most recent write
   	\item \textbf{Partition tolerance}\\
            The system continues to operate despite an arbitrary number of messages being dropped (or delayed) by the network between nodes
   \end{itemize}
\end{definition}

\note{Also known as Brewer's theorem, the CAP theorem was formulated by Eric Brewer in 2000.}

{When a network partition failure happens should we decide to either\ns
\begin{itemize}
	\item Cancel the operation and thus decrease the availability but ensure consistency
	\item Proceed with the operation and thus provide availability but risk inconsistency
\end{itemize}}
\coolquote{
   The CAP theorem implies that in the presence of a network partition, one has to choose between consistency and availability.
}{Prof. Dazzi}

\section{Trade off}
CAP theorem describes the trade-offs involved in distributed systems: a proper understanding of CAP theorem is essential to make decisions about the future of distributed system design.
Misunderstanding can lead to erroneous or inappropriate design choices.

{The CAP theorem suggests there are three kinds of distributed
systems:\ns
\begin{itemize}
	\item \textbf{CP} $\leftarrow$ Consistent when Partitioned
	\item \textbf{AP} $\leftarrow$ Available when Partitioned
	\item \st{\textbf{CA} $\leftarrow$ Consistent and Available Not Partitioned}\\
	This cannot exist in a distributed system; network partitions are inevitable. 
\end{itemize}}

Both AP and CP systems can offer a degree of consistency, and availability, as well as partition tolerance.

\section{Consistency}
AP systems provide best effort consistency, meaning that they will eventually become consistent, but not necessarily at the time of the request.\\
Examples are web caching and DNS.

\begin{itemize}
	\item \textbf{Strong} Consistency - all users see the same data at the same time
	\item \textbf{Weak} Consistency - Reads may return stale data
	\item \textbf{Eventual} Consistency - If no new updates are made to object, eventually all accesses will return the last updated value
   \begin{itemize}
   	\item \textit{Causal} consistency - Processes that have causal
   	      relationship will see
   	      consistent data
   	\item \textit{Read-your-write}
   	      consistency - A process always accesses
   	      the data item after it’s update
   	      operation and never sees an
   	      older value
   	\item \textit{Session} consistency - As long as session exists,
   	      system guarantees read-your-
   	      write consistency
   	      Guarantees do not overlap
   	      sessions
   
   	\item \textit{Monotonic read} consistency - If a process has seen a particular value of data item, any subsequent processes will never return any previous values
   
   	\item \textit{Monotonic write} consistency - The system guarantees to serialize the writes by the same process
   	\item[] These two properties may be combined
   \end{itemize}
\end{itemize}

% // TODO slides 30-40

\section{Example Scenarios}
\subsection{Dynamic Tradeoff - Dynamic Reservation}
In an airline reservation system when most of seats are available it is ok to rely on somewhat out-of-date data, availability
is more critical.

When the plane is close to be filled instead, it needs
more accurate data to ensure the plane is not
overbooked, consistency becomes more critical.

Neither strong consistency nor guaranteed
availability, but it may significantly increase the
tolerance of network disruption.

\section{Partitioning}
\begin{itemize}
	\item Data Partitioning - Different data may require different
   consistency and availability.\\
   Example:
   A \textit{Shopping cart} requires high availability, responsive, can
sometimes suffer anomalies, while \textit{Product information} need to be available, slight variations in inventory are sufferable.
Checkout, billing, shipping records must be consistent.
	\item Operational Partitioning - Different operations may require different consistency and availability.\\
   Example:
   A user login requires high availability, responsive, can sometimes suffer anomalies, while a transaction requires consistency.\\
   In general,
   \begin{itemize}
   	\item \textit{Reads} -  high availability; e.g. ``query''
	   \item \textit{Writes} -  high consistency, lock when writing; e.g. ``purchase''
   \end{itemize}
	\item Functional Partitioning
	% // TODO
	\item User Partitioning
	\item Hierarchical Partitioning
\end{itemize}


\section{PACELC Theorem}

\begin{definition}
   [PACELC Theorem]
   The PACELC theorem is an extension of the CAP theorem that considers latency and consistency in distributed systems.
   \begin{itemize}
   	\item \textbf{P} - Partition tolerance
   	\item \textbf{A} - Availability
   	\item \textbf{C} - Consistency
   	\item \textbf{E} - Else
   	\item \textbf{L} - Latency
   	\item \textbf{C} - Consistency
   \end{itemize}
\end{definition}

\framedt{PACELC in other terms}{
   \begin{itemize}
   	\item If there is a partition (P), how does the system trade off availability and consistency (A and C);
      \item else (E), when the system is running normally in the absence of partitions, how does the system trade off latency (L) and consistency (C)?
   \end{itemize}
}

There are some systems which adopt various PACELC flavors:
\begin{itemize}
	\item \textit{Give up}/\textbf{PA/EL} - \textsc{Dynamo}, \textsc{Cassandra}: Give up consistency for availability and lower latency.
	\item \textit{Refuse}/\textbf{PC/EC} - \textsc{BigTable}, \textsc{HBase}, \textsc{VoltDB}, \textsc{H-Store}: Refuse to give up (compromise) consistency for availability and lower latency.
	\item \textit{Give up(?)}/\textbf{PA/EC} - \textsc{MongoDB}: Give up consistency when a partition happens, and keep consistency in normal operations, when there is no partition.
	\item \textit{Keep}/\textbf{PC/EL} - \textsc{Yahoo! PNUTS}: Keep consistency when a partition happens, and give up consistency for lower latency in normal operations, when there is no partition.
\end{itemize}
\section{Takeaways}

\begin{itemize}
	\item \ul{Consistency Models Define Expectations}: Each model sets specific rules on how data should behave across distributed nodes, defining when data changes become visible across the system.
	\item \ul{Trade-offs Are Inevitable}: No single model can satisfy all consistency, availability, and partition tolerance (CAP theorem), so systems often need to choose between strong consistency and high availability, especially during network partitions.
	\item \ul{Choosing the Right Model Depends on Use Case}: Strong consistency is essential for scenarios demanding strict accuracy (e.g., banking), while eventual or weak consistency is more practical for applications prioritizing availability and low latency (e.g., social media).
	\item \ul{Causal and Sequential Consistency Balance Practicality and Order}: These models allow for some flexibility while still preserving the order of events, making them useful in collaborative applications or messaging systems.
	\item \ul{Eventual Consistency is the Backbone of High Scalability}: Systems prioritizing high availability and partition tolerance, like DNS or e-commerce platforms, often rely on eventual consistency, accepting slight delays in consistency for better performance.
	\item \ul{CAP and PACELC Guide Model Selection}: CAP helps in understanding consistency trade-offs during partitions, while PACELC addresses trade-offs between latency and consistency even in the absence of partitions, providing a more nuanced framework for design decisions.
	\item \ul{Practical Knowledge Matters}: Understanding how each model impacts system performance, reliability, and usability is crucial for system architects and developers to make informed decisions, especially in large-scale distributed environments.
\end{itemize}
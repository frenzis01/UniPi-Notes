\chapter{Consesus}

\section{Introduction}
\textbf{Safety} and \textbf{liveness} are two fundamental properties of distributed systems.
Safety is about avoiding incorrect decisions, while liveness is about -eventually- making progress.

Consesus is about finding an agreement over a value.

\subsection{Importance}
With Distributed databases \& replication it is necessary to ensure consistency across replicas, or at least to ensure that the replicas eventually converge to a consistent state, implementing a synchronization mechanism.

\note{\textit{Transparency} of a consesus algorithm to the DBMS depends on the level of abstraction, or the viewpoint considered in a given case. An SQL programmer may not care of a consesu algorithm on top, but a DBA (Admin) should.}

In general it may be also important for synchronizing behaviour in a distributed system, for example to agree on a leader, or allocate resources.
\nl

Consesus alorithm find wide application in Blockchains and Cryptocurrencies, where the agreement is about the validity of a transaction. An example are both Proof of Work and Proof of Stake, exploited by Bitcoin and Ethereum respectively.

% // TODO microservices

\section{Mechanisms for Consesus}
There are various ways for implementing consesus:
\begin{itemize}
   \item Majority agreement:
   Paxos and Raft work by ensuring a majority of nodes agree on a value.
   \item Leader-based coordination:
   Sometimes protocols use a leader to drive agreement.
   \item Voting and Quorums:
   Nodes vote on proposed values, and a \textit{Quorum} (majority) must agree.
   The difference with majority agreement is that a quorum is a \textit{subset} of nodes, not necessarily all the nodes in the system. 
   \item Handling failures:
   Consesus algorithms are designed to tolerate node and network failures: timeouts, retries\dots
   % // TODO
\end{itemize}

\section{Safety and Liveness}

\textbf{Safety} ensures that the systems never 
% //TODO

\textbf{Safety} can be thought of as the ``nothing bad happens''.
It prioritizes correctness: Even under failure conditions, the system should not violate its invariants.

\textbf{Liveness} ensures that the system eventually makes progress and reaches a decision.
The system must continue to work, eventually deciding on a value despite failure or delays.
In other words, we must not have a deadlock \smiley.


\framedt{
   Ensuring safety delays progress.
}{
   There is a trade-off between safety and liveness.
   Liveness might occasionally compromise safety by making premature decisions, especially asynchronous systems.
   For example, a system might decide on a value before all nodes have agreed on it.
}


\subsection{FLP Impossiblity Theorem}
\begin{definition}
   [FLP Impossibility Theorem]
   In an asynchronous system, with the possibility of node failures, it is impossible to have a consesus algorithm that is both safe and live.
\end{definition}
This is a bit pessimistic, but it is a fundamental result in distributed systems.

The theorem is named after Fischer, Lynch, and Paterson, who proved it in 1985.

% // TODO

This is an important result, since it highlights the need of a trade-off between safety and liveness in distributed systems.

\subsubsection{Proof}
\begin{enumerate}
   \item Starting Configuration - Nodes must agree on a decision (value)  
   \item Indecision Phase - As messages are exchanged there exists a state where no node has yet enough information to decide on a value yet. 
   This state is prolonged by message delays and process failures.
   \item Failure example -  If one node crashes or its messages are indefinitely delayed, other nodes are forced to wait, leading to a situation where the system the systems cannot reach consesus
   \item Conclusion -  Since there is no way to distinguish between slow and crashed nodes, and since nodes rely on receiving messages to make decisions, it's impossibile to guarantee both safety and liveness in asynchronous systems with faults.
\end{enumerate}

At step 3 we understand that in order to ensure Liveness, we must sacrifice safety, because we don't know what the unavailable node (either crashed or slow) would have decided.
On the other hand, ensuring safety means that we must wait for the node to recover, which compromises liveness.

\subsubsection{Implications}
There is an impact on consesus algortihms:
FLP shows that we must relax some assumptions on properties in real-world systems.
Raft and Paxos manage this trade-off by making practical compromises. 


\section{Requirements}
...

\section{Paxos}
Paxos is a family of algorithms for solving consesus in a network of unreliable processors.
It was first described by Leslie Lamport in 1998, but initially proposed in 1989.

\subsection{Components}

Paxos is completely asynchronous, so it does \textbf{not} assume that all nodes are in the same status when the algorithm starts.

\labelitemize{Roles}{
   \begin{itemize}
      \item Proposers - Propose a value to be agreed upon.
      \item Acceptors - Accept a value proposed by a Proposer.
      \item Learners - Learn the agreed value.
   \end{itemize}
}

\begin{enumerate}
   \item Phase 1
   \begin{itemize}
      \item \textbf{Prepare} - 
      Proposer sends a Prepare request with a proposal number. Then the proposer chooses a Quorum of Acceptors and sends the Prepare message containing n to them.
      \item \textbf{Promise} - 
      Acceptors respond with a Promise if the proposal number n is higher than any they've previously seen. Otherwise it may not respond or send a negative message back.
   \end{itemize}
   \item Phase 2
   \begin{itemize}
      \item \textbf{Accept} -
      Upon receiving a Promise from a majority of Acceptors, the Proposer sends an Accept request.
      \item \textbf{Accepted} -
      If an Acceptor receives an Accept message from a Proposer, it must accept it if and only if it has not already promised (in Phase 1b) to only consider proposals having an identifier greater than other received
   \end{itemize}
\end{enumerate}

% //TODO or n>2f?
If faulty nodes are $f$ and total nodes are $n \geq 2f + 1 $, then Paxos can tolerate $f$ faulty nodes, since it exploits strict majority.

\subsection{Challenges and Applications}
\proscons{Challenges}{Applications}{\begin{itemize}
   \item Complexity in understanding and implementing the algorithm
   \item Overhead due to message exchanges
   \item Network partitions and node failures can affect Liveness
\end{itemize}}
   
{
   \begin{itemize}
      \item 
      Google chubby
      //TODO
   \end{itemize}
}

\section{Key Takeaways}
Consesus is essential for coordination in distributed systems, and it is a fundamental problem in distributed computing.


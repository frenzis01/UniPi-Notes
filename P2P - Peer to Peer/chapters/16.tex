\chapter{Blockchain Applications}
Aside from cryptocurrencies, blockchain technology has many other applications. In this chapter, we will explore some of the most popular blockchain applications and how they are being used today.

\section{Supply Chain Management}
Blockchain can provide increased supply chain transparency, as well as
reduced cost and risk across the supply chain. This is because blockchain can provide a single source of truth for all parties involved in a supply chain, which can help to reduce disputes and errors. In addition, blockchain can help to improve traceability and accountability in the supply chain, which can help to reduce fraud and counterfeiting.

A centralized database maintained by a manufacturer designed to track components, equipments, locations, service histories, etc. may result in data silos, data duplication, data inconsistency, and may potentially be a single point of failure: data tampering may result in serious business damage for the manufacturer.
A blockchain-based system instead is both \textbf{distributed} and \textbf{tamper-proof} can provide a single source of truth for all parties involved in a supply chain, which can help to reduce disputes and errors.

\section{Voting Systems}

Missing section
% //TODO

\section{NFTs}
NFTs (Non-Fungible Tokens) are unique digital assets that are stored on a blockchain. They can represent anything from digital art to virtual real estate to collectibles. NFTs are created using smart contracts, which are self-executing contracts with the terms of the agreement between buyer and seller being directly written into lines of code. This makes NFTs secure, transparent, and tamper-proof.

Common use cases are:
\begin{itemize}
   \item \textbf{Gaming}\\
   Tradeable in-game assets, such as weapons, armor, and skins, are popular NFTs in the gaming industry. Players can buy, sell, and trade these assets on blockchain-based marketplaces.
   \item \textbf{Collectibles}\\
   An example are baseball cards, becoming popular in the last years.
   \item \textbf{Art}\\
   \textit{Rarible} and \textit{OpenSea} are popular marketplaces for NFT art. NFTs allow to monetize digital art in a way that was not possible before.
   \item \textbf{Virtual Assets}\\
   Domains such as \texttt{.eth} and \texttt{.crypto} have been turned into NFTs.\\
   \textit{Decentraland} and \textit{CryptoVoxels} are examples of virtual worlds where users can buy, sell, and trade virtual real estate.
   \item \textbf{Real-world Assets}\\
   \textit{OpenLaw} is a platform that allows users to create and manage legal agreements using blockchain technology. Users can create NFTs that represent real-world assets, such as real estate, cars, and intellectual property.\\
   \textit{Nike} has patented a system to tokenize shoes. 
   \item \textbf{Identity}\\
   NFTs would allow to store identity information on the blockchain, such as passports, driver's licenses, birth certificates, and medical history.
\end{itemize}

\section{Buying and Selling NFTs}
To buy and sell NFTs, you need to use a blockchain wallet that supports NFTs, such as MetaMask. You can connect your wallet to a blockchain-based marketplace, such as OpenSea or Rarible, and browse the available NFTs. When you find an NFT that you want to buy, you can place a bid or purchase it outright using cryptocurrency. Once you own an NFT, you can sell it on the marketplace or transfer it to another wallet.\\
There is no need to know any contracts programming knownledge to mint\footnote{i.e. Create}, buy and sell NFTs. 

\section{Identity Management}
SSI (Self-Sovereign Identity) is a concept that allows individuals to control their own digital identities. This is done by using blockchain technology to create decentralized identifiers (DIDs) and verifiable credentials. DIDs are unique identifiers that are stored on a blockchain and can be used to prove ownership of a digital identity. Verifiable credentials are digital certificates that can be used to prove claims about an individual, such as their age, address, or qualifications. SSI allows individuals to share their digital identities with others in a secure and privacy-preserving way.

A DID document is a JSON object that contains information about a DID, such as its public key, authentication methods, and service endpoints. Verifiable credentials are JSON objects that contain claims about an individual, such as their name, date of birth, or address. Verifiable credentials are signed by the issuer and can be verified by the recipient using the issuer's public key.\\
A DID (\textit{Decentralized Identifier}, i.e. key?) resolves to an abovementioned DID document, which, more precisely, contains:
\begin{enumerate}
   \item DID
   \item Set of public keys
   \item Set of authentication methods
   \item Set of service endpoints
   \item Timestamp of the document
   \item Signature of the document
\end{enumerate}
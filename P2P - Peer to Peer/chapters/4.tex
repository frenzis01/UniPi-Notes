\chapter{BitTorrent}

The goal of \textit{Content Distribution Networks} is to distribute web contents to hundreds of thousands or millions
of simultaneous users, \ul{exploiting data and/or service \textbf{replication} on different \textbf{mirror servers}}.

In \textbf{P2P CDN} the initial file request are served by a centralized server, and further requests served by peers which have already received and replicated the files (\textbf{\textit{seeders}}), without involving the initial server.

\begin{center}
\fbox{
   \begin{minipage}{0.8\columnwidth}
      \nl
      \begin{center}
         \ul{\textbf{BitTorrent} in a nutshell}
      \end{center}
      \nl

      \begin{itemize}
         \item Basically a \textit{Content Distribution Network} (\texttt{CDN})
         \item A distributed set of hosts cooperating to distribute large data set to end users.
         \item Efficient content distribution systems using \textit{file swarming}
         \item Does \textit{not} perform all the functions of a typical P2P system, like searching
         \item Rather than providing a search protocol itself, was designed to integrate seamlessly with the Web and made file descriptors available via Web, which could be searched with standard Web search
         \item \textit{File swarming}: a peer makes whatever portion of the file that is downloaded immediately available for sharing
      \end{itemize}
   \end{minipage}  
   }
\end{center}

\section{Deeper into BitTorrent}
\begin{figure}[htbp]
   \centering
   \includegraphics{images/bit_swarmschema.png}
   \caption{Swarm schema}
   \label{fig:bit_swarmschema}
\end{figure}
\subsection{Glossary}
\begin{itemize}
   \item \textbf{tracker}: active entity which coordinates
   the peers sharing the file, taking trace of who is currently providing the content
   \note{\begin{itemize}
      \item Joe connects to the tracker announcing the content
      \item the tracker now knows Joe is providing the file
   \end{itemize}}
   \item \texttt{.torrent} a descriptor of the file to be published on a server, which includes a reference to a tracker
   \item \textbf{swarm} set of peers collaborating to the distribution of the same file coordinated by the same tracker
   \item \textbf{seeder} peer which owns all the parts of the file
   \item \textbf{leecher} peer which has some part or no part of the file and downloads the file from the seeders and/or from other lechers.
\end{itemize}

\subsection{Protocol Overview}
\begin{figure}[htbp]
   \centering
   \includegraphics{images/bit_overview.png}
   \caption{BitTorrent protocol overview}
   \label{fig:bit_overview}
   BitTorrent protocol is built on top of HTTP
\end{figure}
\labelitemize{\textit{Seeder}}{
   \begin{enumerate}
      \item Upload the .torrent on a Torrent Server
      \item Opens a connection to the Tracker and informs it of its own existence: for the moment, it is the only peer which owns the file
   \end{enumerate}
}
\labelitemize{\textit{Peers}}{
   \begin{enumerate}
      \setcounter{enumi}{2}
      \item Retrieves the file descriptor (.torrent) and opens it through the BitTorrent
      client
      \item Opens a connection to the tracker and informs it of its own existence and
      receives from the tracker a list of peers of the swarm
      \item Opens a set of connections with other peers of the swarm.
   \end{enumerate}
}

Objects are serialized in \textbf{Bencode}, which is ---not popular as \texttt{JSON}--- used only in torrent; provides 4 data types: String, Integer, Lists and Dictionaries.\\
Content is split into chunks called pieces (256KB - 2MB):
when a peer receives a piece, it becomes the seeder of that piece.
\note{
   \ns
      There is a SHA-1 hash per piece stored in the .torrent file, used to check the piece once it is fully downloaded, 
      allowing to require retransmission in case the check fails.\\
      Pieces size got adapted to have a reasonably small .torrent file
}
Pieces are then split in \textbf{subpieces} (\textit{\textbf{blocks}}) of 16KB, with each one downloadable from a different peer, optimizing the bandwith and allowing \textit{pipelining}, decreasing the overall download time.

Trackers keep a database of swarms identified by torrent hash, and knows also the state of each peer in each swarm.
In the last versions, \textbf{trackerless} BitTorrent uses \textit{Kademlia DHT} to avoid the centralization point of the tracker.

\section{Pieces selection}
The order in which pieces are selected by different peers is critical for good performance, to avoid making peers end up stuck with the same pieces.
\labelitemize{\textit{Policies}}{
   \begin{itemize}
      \item \textbf{Strict Priority}\\
      Complete the ``assembling'' of a piece before asking for another piece
      \item \textbf{Rarest First}\\
      Download the rarest pieces first
      \item \textbf{Random First Piece}\\
      Choose a random piece in the bootstrap phase
      \item \textbf{Endgame}\\
      Broadcast the remaining blocks when the file download is going to terminate
   \end{itemize}
}
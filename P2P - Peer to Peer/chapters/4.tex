\chapter{BitTorrent}

The goal of \textit{Content Distribution Networks} is to distribute web contents to hundreds of thousands or millions
of simultaneous users, \ul{exploiting data and/or service \textbf{replication} on different \textbf{mirror servers}}.

In \textbf{P2P CDN} the initial file request are served by a centralized server, and further requests served by peers which have already received and replicated the files (\textbf{\textit{seeders}}), without involving the initial server.

\begin{center}
\fbox{
   \begin{minipage}{0.8\columnwidth}
      \nl
      \begin{center}
         \ul{\textbf{BitTorrent} in a nutshell}
      \end{center}
      \nl

      \begin{itemize}
         \item Basically a \textit{Content Distribution Network} (\texttt{CDN})
         \item A distributed set of hosts cooperating to distribute large data set to end users.
         \item Efficient content distribution systems using \textit{file swarming}
         \item Does \textit{not} perform all the functions of a typical P2P system, like searching
         \item Rather than providing a search protocol itself, was designed to integrate seamlessly with the Web and made file descriptors available via Web, which could be searched with standard Web search
         \item \textit{File swarming}: a peer makes whatever portion of the file that is downloaded immediately available for sharing
      \end{itemize}
   \end{minipage}  
   }
\end{center}

\section{Deeper into BitTorrent}
\begin{figure}[htbp]
   \centering
   \includegraphics{images/bit_swarmschema.png}
   \caption{Swarm schema}
   \label{fig:bit_swarmschema}
\end{figure}
\subsection{Glossary}
\begin{itemize}
   \item \textbf{tracker}: active entity which coordinates
   the peers sharing the file, taking trace of who is currently providing the content
   \note{\begin{itemize}
      \item Joe connects to the tracker announcing the content
      \item the tracker now knows Joe is providing the file
   \end{itemize}}
   \item \texttt{.torrent} a descriptor of the file to be published on a server, which includes a reference to a tracker
   \item \textbf{swarm} set of peers collaborating to the distribution of the same file coordinated by the same tracker
   \item \textbf{seeder} peer which owns all the parts of the file
   \item \textbf{leecher} peer which has some part or no part of the file and downloads the file from the seeders and/or from other lechers.
\end{itemize}

\subsection{Protocol Overview}
\begin{figure}[htbp]
   \centering
   \includegraphics{images/bit_overview.png}
   \caption{BitTorrent protocol overview}
   \label{fig:bit_overview}
   BitTorrent protocol is built on top of HTTP
\end{figure}
\labelitemize{\textit{Seeder}}{
   \begin{enumerate}
      \item Upload the .torrent on a Torrent Server
      \item Opens a connection to the Tracker and informs it of its own existence: for the moment, it is the only peer which owns the file
   \end{enumerate}
}
\labelitemize{\textit{Peers}}{
   \begin{enumerate}
      \setcounter{enumi}{2}
      \item Retrieves the file descriptor (.torrent) and opens it through the BitTorrent
      client
      \item Opens a connection to the tracker and informs it of its own existence and
      receives from the tracker a list of peers of the swarm
      \note{The tracker's address Is included in the \texttt{.torrent} file}
      \item Opens a set of connections with other peers of the swarm.
   \end{enumerate}
}

Objects are serialized in \textbf{Bencode}, which is ---not popular as \texttt{JSON}--- used only in torrent; provides 4 data types: String, Integer, Lists and Dictionaries.\\
Content is split into chunks called pieces (256KB - 2MB):
when a peer receives a piece, it becomes the seeder of that piece.
\note{
   \ns
      There is a SHA-1 hash per piece stored in the .torrent file, used to check the piece once it is fully downloaded, 
      allowing to require retransmission in case the check fails.\\
      Pieces size got adapted to have a reasonably small .torrent file
}
\ul{Pieces are then split in \textbf{subpieces} (\textit{\textbf{blocks}}) of 16KB}, with each one downloadable from a different peer, optimizing the bandwith and allowing \textit{pipelining}, decreasing the overall download time.
Pieces are the smaller units of retransmission, and are checked using a SHA-1 hash once all of its blocks are downloaded.

\ul{\textbf{Trackers} keep a database of swarms identified by torrent hash, and knows also the state of each peer in each swarm}.
In the last versions, \textbf{trackerless} BitTorrent uses \textit{Kademlia DHT} to avoid the centralization point of the tracker.

\section{Pieces selection}
\begin{figure}[htbp]
   \centering
   \includegraphics{images/bittorrent_policy.png}
   \caption{BitTorrent policies over time}
   \label{fig:bittorrent_policy}
\end{figure}
The order in which pieces are selected by different peers is critical for good performance, to avoid making peers end up stuck with the same pieces.
\labelitemize{\textit{Policies}}{
   \begin{itemize}
      \item \textbf{Strict Priority}\\
      \ul{Complete the ``assembling'' of a piece before asking for another piece}.
      \item \textbf{Rarest First}\\
      \ul{Download the rarest pieces first}.
      Allowed since the peers have a local view of the availability of each piece. Acquiring rare pieces means that many peers will be interested in them, increasing the download speed for the peer owning them (``tit-for-tat'' approach).
      \item \textbf{Random First Piece}\\
      \ul{Choose a random piece ---only--- in the bootstrap phase}.
      Used because peers need to acquire a piece ASAP to start negotiating with other peers, i.e. downloading remaining pieces
      \item \textbf{Endgame}\\
      \ul{When the file download is almost terminated, the remaining pieces are required in \textit{parallel} to all peers who own them}.
      This policy is executed for a small period of time.
      It is used because typically most downloads slow down when the file is almost complete.
   \end{itemize}
}

\subsection{Free Riders}
\ul{Free riders} in BitTorrent are \ul{peers that do not put their bandwidth at disposal} of the community.\\
Several non official BitTorrent clients enable the user to limit the upload bandwidth as they like.
The performance of the whole network relies on the \ul{cooperative behavior of the peers}, making free riders a problem.

An approach to address this issue is based on \textbf{reciprocity}, allowing a client to obtain a good service if and only if it gives a good service to the community, by exploiting a dynamic strategy based on connection monitoring called ``Tit for Tat'', implemented using \textbf{choking}:\\
``choking'' means \textit{temporarily} refusing to upload to another peer, but still downloading from them;  
the principle is to ``upload to peers who have uploaded to us'', so each peer, periodically, evaluates for each neighbor  the \textit{upload rate} and the \textit{download rate}, and decides a fixed number of neighbors to \textit{choke} (and \textit{unchoke}, see)

\labelitemize{\textit{Choking}}{
\begin{center}
   \textit{The local peer can receive data from a remote peer if}
   \begin{itemize}
      \item The local peer is \textit{interested} in the remote peer
      \item The remote peer \textit{unchoked} the local peer
   \end{itemize}
\end{center}
}

Choking only peers that upload the most to the local peers would lead to ignoring peers that recently join the network
and to the lack of discovery of connections actually better than the used ones.\\
To avoid this, BitTorrent uses \textbf{optimistic unchoking}, i.e. \ul{one random peer is being unchoked}.\\
Then, every 30s an interested and choked peer is selected at random \textbf{planned optimistic unchoke} (\texttt{POU}), and if this new connection turns out to be better than one of the existing
unchoked connections, it will replace it.

In case a peer is choked by everyone, it follows an \textbf{anti-snubbing} policy, by increasing the number of simultaneous optimistic unchocke to more than one.

For \textit{seeders} this schema does clearly not apply, since they do not have to download anything; hence they use a different choking algorithm:
\ul{unchoke peers with the highest upload rate}, ensuring that pieces get uploaded and replicated faster.

\section{DHT and BitTorrent}
Even though a tracker requires very little resources, it is still a single point of failure, and a potential bottleneck for the system, besides it may be target of several\footnote{Not only cyber, also \textit{legal} ones} attacks.

To address this problem, Bittorrent Inc. introduces its own DHT, called \textit{Mainline DHT},
based on Kademlia, but with some improvements concerning 
\begin{itemize}
   \item Routing table management
   \item Look-up
\end{itemize}
The main purpose of Mainline DHT is to provide a “trackerless” peer discovery mechanism to locate peers belonging to a swarm.

The DHT acts as a distributed tracker, and stores the content's infohash (\textit{key}) and the list of peers in the swarm (\textit{value}).
Each node implements both the DHT and the BitTorrent protocol.

The protocol provides four messages:
\begin{enumerate}
   \item \texttt{PING} to check if a node is available
   \item \texttt{FIND\_PEER(target\_id)} to find the node closest to a given ID
   \item \texttt{ANNOUNCE\_PEER} to announce that a peer is part of a swarm
   \item \texttt{GET\_PEERS(infohash)} to find the peers in a swarm
\end{enumerate}

\begin{paracol}{2}
   \colfill
   The DHT is exploited just as a regular DHT, where the value to seek is the list of seeders for a given infohash.
   \texttt{Alice} asks its neighbors if they know the seeders for an infohash (tracker info for the infohash), and if they do not, they ask their neighbors, and so on;
   When \texttt{Alice} receives from node \texttt{15} the info that node \texttt{4} is a seeder for the infohash, they will connect to node \texttt{4} to download the file.
   \colfill
   \switchcolumn
   \begin{figure}[htbp]
      \centering
      \includegraphics{images/bittorrent_dht.png}
      \caption{\texttt{15} has tracking information for the pink (\texttt{14}) content, so it returns info about all the peers providing the file: in this case, only \texttt{Joe} with \texttt{ID 4}}
      \label{fig:}
   \end{figure}
\end{paracol}
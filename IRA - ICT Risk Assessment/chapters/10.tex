\chapter{Deception with Honeypots}
A \textbf{honeypot} is a system designed exclusively to be attacked an to \textbf{collect information} about the attacker and its tactics, techniques and procedures;
the other focus of an honeypot is also to possibly slow down an attack to a system by \textbf{diverting} it on itself.

The scaled version of a honeypot, is a \textbf{honeynet},
which is an entire network attached to a real system designed to be targeted instead of the main system.

\subsection{Classification}
\begin{enumerate}
   \item \textbf{Interaction}-based
   \begin{enumerate}
      \item \textit{Low} - e.g. simple port listener
      \item \textit{Medium} - emulation of a network service that analyzes the inputs and returns some replies similar to those the real service would return.
      \begin{itemize}
         \item Simulates just some \textbf{features} of the service
         \item Easy to implement, \textbf{low risk}
         \item Can collect a low amount of information
         \item Tools $\longrightarrow$ OS + Honeyd
      \end{itemize}
      \item \textit{High} - built around real services that run on real machines to fool the attacker
      \note{realistic but dangerous due to the large amount of vulnerable software}
      \begin{itemize}
         \item Simulates \textbf{all features} of the service and of the underlyng OS
         \item The attacker may fully compromise and control it
         \item \textbf{High risk}
         \item A larger amount of information
         \item Tools $\longrightarrow$ Honeynet.
      \end{itemize}
   \end{enumerate}
   
   \item \textbf{Implementation}-based
   \begin{enumerate}
      \item \textit{Virtual}
      \item \textit{Physical}
   \end{enumerate}
   
   \item \textbf{Goal}-based
   \begin{enumerate}
      \item Production
      \item Research on attacker behaviour
   \end{enumerate}
\end{enumerate}

\section{Honeyd}
\textbf{Honeyd} is a daemon which creates virtual nodes in a network.
It is highly configurable and is able to reproduce even large and complex networks;
besides it can integrate with virtual and physical real-existing networks.\\
\textbf{Honeyd} provides many features, we can list some of the main ones:
\begin{itemize}
   \item It detects illegal activities in a network by monitoring the IP addresses that are not within a \textbf{range} named \textit{"dark space"}.
   Any attempt of connection to or from the \textit{dark space} is assumed to be an attack or a vulnerability scan.
   \item It monitors activities related to TCP and UDP ports and ICMP traffic.
   \item It can emulate network services using script in Perl, shell or other way of interacting with the attacker.
\end{itemize}

\subsection{Architecture}
\begin{itemize}
   \item \textit{Configuration database} -
   Queried to discover the model paired with the destination IP address
   \item \textit{Packet dispatcher} -
   analyzes input packets and checks correctness and integrity.
   Anything different from TCP, UDP and ICMP gets \textit{discarded}.
   \item \textit{Protocol manager} -

   \item \textit{Personality engine} -
   computes a reply packet and updates it to guarantee coherence with the OS that the destination is expected to use
   \item \textit{Optional routing component} -
   allows the routing of a packet to a real application
\end{itemize}

\subsection{Research results}
\textbf{Honeypot}s provided an important amount of data to perform research on.
Many statistics have been computed to produce estimations and interesting results,
a \textit{\textbf{UniPi} student presented a thesis on the topic} \smiley.
\note{Honeypot thesis available in the \href{https://elearning.di.unipi.it/mod/resource/view.php?id=18233}{web page of the course}}
\chapter{Attacks}
\section*{3 - Ottobre}
\section{Attacks and Vulnerabilities}
Following the discovery of a vulnerability $v$ there's an analysis to evaluate which attacks are enabled by $v$.
\textbf{Attacks} can be described as a set of attributes:
\begin{enumerate}
    \item Precondition
    \item Postcondition
    \item Success Probability
    \item Know how, abilities, tools required
    \item Noise = Probability of being discovered
    \item Automated/Potentially automatable/manual
    \item Local/Remote
    \item Actions to implement attack\footnote{See following Section on attack taxonomy}
\end{enumerate}
Even though some attack evaluation proposals map to each attribute a number and combine them into a value,
such evaluations do not consider that \textbf{risk} resides in \textit{intrusions}, not individual attacks, because they have a considerable impact on the system, and keep in mind that are composed by:
\begin{itemize}
    \item Exploration and information collection
    \item Persistence
    \item Attack \textit{chain} for privilege escalation
\end{itemize}

\section{Attack Classification}
\label{sec:attack_taxonomy}
The actions need to implement an attack may be used to define a \textbf{taxonomy} of attacks:
\begin{enumerate}
    \item buffer/stack/heap overflow
    \item \textit{sniffing} $\rightarrow$ Illegal access to info in travel
    \item \textit{replay attack} $\rightarrow$ Repeated exchange of legal messages 
    \item \textit{Interface attack} $\rightarrow$ Illegal order in the invocation of API functions
    \item \textit{Man-in-the-middle} $\rightarrow$ Interception and manipulation of info in travel
    \item Diversion of an information flow
    \item \textit{Race-condition} $\rightarrow$ Time-to-use time-to-check
    \item \textit{Cross site scripting} $\rightarrow$  XSS
    \item SQL injection
    \item \textit{Bell-Lapadula policy} $\rightarrow$ Covert channel 
    \item Masquerading as
    \begin{itemize}
        \item user
        \item machine (\textit{IP/DNS spoofing, Cache poisoning}
        \item connection (\textit{connection stealing/insertion})
    \end{itemize}
\end{enumerate}

\section{Examining attacks}
\subsection*{Replay attack}
Suppose a user asks the bank to transfer some money to $Y$ account with an $M$ message.
$Y$ may sniff and record $M$, and before the secure channel $S$ gets deleted, $Y$ sends $M$ several time.\\
Note that the attack may work even if encryption is used.

\subsection*{Man-in-the-middle}
If $A$ and $B$ communicate, $E$ may pose itself in the middle, acting as if it were $B$ to $A$ and $A$ to $B$.
Such attack is possible when no authentication is required.

\subsection*{XSS}
A website allows users to upload contents to be later (possibly) downloaded by users.
Thus a malicious user may upload hidden scripts to damage or steal information from the user who download their content.
To avoid this the website must check the content uploaded by users.\\
A well known attack of this type targeted BBC.

\subsection*{SQL Injection}
An input may insert a malicious query (i.e. \lstinline[language=SQL]{DROP TABLE USERS}) in a credentials field.
The best way to avoid this is to whitelist using RegEx.

\subsection*{Cryptography attacks}
These are a category on their own, there are many types, with different variations and features.

\subsection*{Side-channel attacks}
Any attack that measures some physical value to discover an encryption key.
Currently it is popular due to the capabilities of machine learning in exploiting large number of pairs to deduce a function.\\
Such measures may be:
\begin{itemize}
    \item Electromagnetic emissions
    \item Energy consumption
    \item Execution time to discover inner status
    \item Execetion time to discover cache usage and prediction mechanisms.
\end{itemize}

\subsection*{Virtual Machines \& Blue Pill}
Cyber system may be composed of many virtual machines onion-like organized.
Thus, attacking a low-level VM may grant access rights to higher ones.\\
Besides, an attacker may insert a new VM in the hierarchy:
this is called \textit{Blue Pill} attack, it's hard to discover and has a high impact.
A new VM may return to higher VMs fake information on the status of the underlying machines and/or send malicious commands to the underlying machines.\\
Stuxnet was a malware which used to send commands to uranium enrichment centrifuges to destroy them, and meanwhile told the operator that everything was going well.  

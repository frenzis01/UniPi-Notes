\chapter{Google System Design Strategies}

\section{Least Privilege}
Users should have the minimum amount of access to accomplish a task,
regardless of whether the access is performed by humans or systems.\\
Besides, since accounts may be compromised, systems should limit user access to the data and services required to do the \textbf{task at hand}.

These restrictions are most effective when you add them at the beginning of the development lifecycle,
during the design phase of new features
\note{
   security by design, as required by GDPR
}
Unnecessary privileges lead to a growing surface area for possible mistakes, bugs, or compromises, i.e. creating security and reliability risks that are expensive to contain or minimize in a running system.

In other terms:
\begin{itemize}
   \item it is better to reduce the attack surface in the \textbf{design} because this
   reduces the cost more than an update after deploying a system.
   \item the cost of security mechanisms depends upon the time when they
   are adopted, i.e. \textit{the later the more expensive}
\end{itemize}


\labelitemize{
   \textit{You should}
}{
   \begin{itemize}
      \item limit the cost of controls by prioritizing what to protect.
      Not all data or actions are created equal, and the makeup of your access may differ dramatically depending on the nature of your system.
      \item not protect all access to the same degree
      \item apply the most appropriate controls and avoid an \textit{"all-or-nothing"} mentality
   \end{itemize}
}

\labelitemize{
   \textit{You need}
}{
   \begin{itemize}
      \item to classify access based on impact, security risk, and/or criticality.
      \item to handle access to different types of data (public versus company
      versus user versus cryptographic secrets) differently
      \item to treat administrative APIs that can delete data differently than
      service-specific read APIs.
   \end{itemize}
}
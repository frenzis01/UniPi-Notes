\chapter{Challenge III}

\section{Outline}
\begin{enumerate}
   \item A user can launch an instance of a compiler C
   \item When launching the instance, the user specifies the name of the
   source file
   \item When the compilation is over, the user specifies the name of the
   file to store the output code
   \item The compiler updates a file log with the accounting information
   so that users pay for using the compiler
\end{enumerate}

\begin{table}[htbp]
   \centering
   \begin{tabular}{c|c|c|c|c|}
      & Compiler & \texttt{input} file & \texttt{output} file &\texttt{log} file\\
      \hline
      User & \texttt{execute} & \texttt{read} & \texttt{write} &\\
      \hline
      Compiler & & & & \texttt{write}\\
      \hline
      \textit{Instance} & & \texttt{read} & \texttt{write} & \texttt{write}\\
   \end{tabular}
   \caption{Outline schema}
   \label{fig:challenge3}
\end{table}
\begin{center}
   \textit{What happens if the user transmits as the name of the output file the one of the logfile?}
\end{center}

The key issue according to Baiardi is that the article asserts that ACLs and capabilities are not equivalent, because ACLs do not allow to prevent the attack while capabilities can.

Solve the problem using ACLs.
\underline{There is a \textit{very very very} simple solution.}

A simple idea is to not grant the \texttt{write} right to the compiler but the \texttt{append} right,
leading the compiler to simply append to the log file both the output of the compilation and the actual log.
This is not a proper solution, but still is a nice countermeasure.
\chapter{Secure OS}

\section{Rogue Firmware}
An erroneous update or because of a malicious software developer, a device can store some malicious software that attacks the
system or offers an hidden backdoor.

Some run time attacks can be discovered by memory introspection or
attestation, other ones can be discovered by an \textit{host-based} Intrusion Detection System called \textbf{Doppelganger}.

A Doppelganger first analyzes the firmware of the \textit{embedded device} to detect
live code regions\footnote{executable parts of the firmware} therein, where it randomly
inserts its \textbf{symbiotes} (\textit{watchpoints}), and computes a CRC32 checksum of such randomly selected regions.

During the firmware execution, every time the \textit{symbiote manager}\footnote{Software that the Doppelganger adds \textit{before} the firmware} detects
a symbiote in memory,
\begin{enumerate}
   \item it stops the execution process (symbiote=breakpoint),
   \item compares the current memory area checksum of the symbiote one,
   \item Dopplganger considers a mismatch an evidence of a modification
   attack and it prevents the processor from running the code.
\end{enumerate}

\note{
   Doppelganger does not defend against attacks that load code in the dynamic memory
}

\section{Secure Management of a Device}
The manifacturer must flash bootstrap credentials onto the device,
which allow it to reach its "home" bootstrap sever.
Then it can ask such bootstrap server for a DM server address and the respective credentials,
to allow the DM server to manage the device.

\section{Encryption in IoT}
\textbf{Simon} and \textbf{Speck} are two families of block ciphers proposed by NSA,
with \texttt{SIMON} being tuned for optimal hardware performance, while \texttt{SPECK} for software.


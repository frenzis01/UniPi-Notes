\chapter{Data Handling}

\section{Classification}
Do understand how to protect data several security policies \textbf{classify data} into \textit{four }categories according the need for
\textbf{confidentiality} and the resulting risk profile:
\labelitemize{\textit{Security needs classification}}{
   \begin{enumerate}
      \item \textbf{Public Data}\\
      Data can be disclosed without restriction.
      \note{Directories,
      Maps, Syllabi and Course Materials, de-identified data sets}
      \item \textbf{Internal Data}\\
      Confidentiality of data is preferred, but information contained in
      data may be subject to open records disclosure.
      \note{email
      correspondence, budget plans, employee EmplID}
      \item \textbf{Sensitive Data}\\
      Data confidentiality required by law, policy, or contractual
      obligation
      \item \textbf{Restricted Data}\\
      Restricted data requires privacy and security protections.
      Special authorization may be required for use and collection.
      \note{data
      sets with individual Social Security Numbers (or last four of SSN), credit card
      transaction or cardholder data, patient health data, financial data}
   \end{enumerate}
}

Another classification is the one based on the availability needed for data:
\labelitemize{Availability needs classification}{
   \begin{enumerate}
      \item \textbf{Supportive Data}\\
      Data that is necessary for day-to-day operations, but
      is not critical to the mission or core functions.
      \item \textbf{High-priority Data}\\
      Availability of data is necessary for organization function. 
      Destruction or temporary loss of data may have an adverse
      affect on organization mission, but would not affect organization-wide
      function.
      \item \textbf{Critical Data}\\
      Critical data has the highest need for availability. If the
      information is not available due to system downtime, modification,
      destruction, etc., the organization functions and mission would be
      impacted. Availability of this information must be rigorously protected.
   \end{enumerate}   
}

\section{Protection}
Having classified the data, we can discuss how to protect it:
\begin{enumerate}
   \item Data \textbf{discovery and inventory}\\
   The first step in data protection is discovering data sets in the organization, paying attention to which are business-critical and which contain
   sensitive data that might be subject to compliance regulations.
   \item Data \textbf{loss prevention} (\textbf{DLP})\\
   DLP is a set of strategies and tools to prevent data from
   being stolen (exfiltration), lost, or accidentally deleted. 
   DLP besides protecting against data loss,
   also helps to recover it.
   \item Storage with \textbf{built-in data protection}\\
   Modern storage equipment provide
   \textit{built-in} disk clustering and redundancy \texttt{RAID}.
   For example, some store
   providers offer up to 14 nines of durability, low cost enabling storage of large
   volumes of data, and fast access for minimal $RecTimeO/RecPercObj$.
   \item \textbf{Backup}\\
   A backup is copy of data and stored separately from the source;
   allows to restore the data to a previous state in case of loss or modification.
   \item \textbf{Snapshots}\\
   A snapshot is a backup of a complete image of a system, including data and
   system files. 
   A snapshot can be used to restore an entire system to a specific point in time.
\end{enumerate}

\subsection{RAID}
\begin{center}
   \textit{\underline{R}edundacy through \underline{A}rray of \underline{I}nexpensive \underline{D}isks}
\end{center}

\begin{itemize}[left=0pt, align=left]

   \item \textbf{RAID 0: Striping}\\
      Data is striped across multiple disks to improve performance, but there is no redundancy. If one disk fails, all data is lost.

   \item \textbf{RAID 1-4: Mirroring}\\
      Data is mirrored on two or more disks for redundancy. If one disk fails, the data is still available on the mirrored disk(s).

   \item \textbf{RAID 5: Striping with Parity}\\
      Data is striped across multiple disks with distributed parity for fault tolerance. If one disk fails, data can be reconstructed from parity information on the remaining disks.

   \item \textbf{RAID 6: Striping with Double Parity}\\
      Similar to RAID 5, but with an additional set of parity data, providing fault tolerance against the failure of two disks simultaneously.

   \item \textbf{RAID 10: Mirrored Striping}\\
      It combines mirroring and striping. Data is both mirrored and striped for performance and redundancy. It requires a minimum of four disks.

   \item \textbf{RAID 50: Striped RAID 5 Arrays}\\
      It combines the straight block-level striping of RAID 0 with the distributed parity of RAID 5. It requires a minimum of six disks.

   \item \textbf{RAID 60: Striped RAID 6 Arrays}\\
      Similar to RAID 50, but with double distributed parity for enhanced fault tolerance. It requires a minimum of eight disks.

\end{itemize}

\labelitemize{\textit{\textbf{Protection }Solutions}}{
   \begin{enumerate}
      \item \textbf{Replication}\\
      Copying data \textit{periodically} on an ongoing basis locally or to another location, 
      providing a living up-to-date data copy,
      allowing both recovery and failover to the copy if the primary system goes
      down.
      \item \textbf{Encryption}\\
      Altering data content according to an algorithm that can only be
      reversed with the right encryption key. Encryption makes data unreadable, protecting it from
      unauthorized access even if stolen.
      \item \textbf{Data erasure}\\
      Eraruse limits liability by deleting data that is no longer needed.
      This can be done after data is processed and analyzed or periodically when data
      is no longer relevant.
      Erasing unnecessary data is a \textit{requirement} of many compliance regulations, \footnote{such as GDPR}.
      \note{\textbf{Effectively} erasing data is not as trivial as it may seem}
      \item \textbf{Disaster recovery}\\
      \textit{Disaster recovery} is A set of practices and technologies that determine how to
      deal with a disaster, such as a cyber attacks, natural disasters, or large-scale
      equipment failures.
      \textit{Disaster recovery} typically involves setting up a remote
      disaster recovery site with copies of protected systems, and switching
      operations to those systems in case of disaster.
   \end{enumerate}
}

\section{API}
The \textbf{API} of a distributed system includes all the ways someone can \textit{query} or
\textit{modify} its \textbf{internal state}.

\textbf{Administrative} \texttt{API}s are clearly more critical than "user" ones for what concerns reliability and security:
typos and trivial mistakes by an admin may result in catastrophic outages or leaks of
huge amounts of data,
making such APIs very \textit{appealing for attackers}.\\
Administrative APIs include APIs for \textit{installing/removing} software and deploying containers or VMs,
and some \textbf{maintenance} and emergency APIs to delete corrupted user data or state, or
to restart misbehaving processes.

When designing with the \textit{Least privilege principle} in mind,
the most intuitive solution is to \textbf{decompose} a large API in smaller ones.
The \textbf{POSIX} API, popular for its flexibility, is huge and kind of suffers from this point of view;
for instance, traditional host setup and maintenance is often performed via an interactive
\texttt{OpenSSH} session or with tools that script against the \texttt{POSIX API},
in both cases exposing the whole API to the caller,
making it difficult to constraint user activities only to the ones he actually needs.
\nl

However, in some situations, minimal APIs may prevent an admin(/user) from performing some tasks.
\textit{\textbf{Breaking-glass} Mechanisms} (BGM) have been created to this extent,
essentially allowing an admin to bypass {--}violate{--} access control
mechanisms \footnote{i.e. the Access Control Matrix} and execute critical
operations such as shutting down a machine or killing a process.\\
BGM may consideratively speed up problem solving and error recovering, but their usage should not be \textit{abused};
actually it should be \textbf{ruled} by the security policy,
and recorded {--}\textit{logged}{--} and checked {--}\textit{audit}{--}.

The ability to use a BGM should be \textit{highly restricted}. 
In general, it should be available only to the team responsible for the operational \textbf{SLA}\footnote{\textit{Service Level Agreement}} of a system.\\
In \textit{ZeroTrust} networks, BGMs should be available only from specific locations, e.g. panic rooms, 
resulting in a strategy which \textbf{distrusts} \textit{network location} but \textbf{trusts} a few locations
with additional \textbf{physical access controls}.\\
BGMs should not only be logged and check (\textit{audited}),
but also regularly tested to ensure that they work when actually needed.
Besides, after solving an issue via a BGM,
the security team should \textbf{inspect} the underlying problem and provide a solution to avoid needing to use a BGM in case such problem arises again.

\paragraph{Service Level Agreement}
Each service contract determines performance indexes to evaluate quality of service, and the minimum performance the service \textbf{must} provide.
Some indicators for a \textbf{SLA} focused on performance are:
\begin{itemize}
   \item \textit{Network performance} (bandwidth, \textbf{availability})
   \item System \textit{component availability} (mainframe, network, bandwidth, ...)
   \item End-to-end \textit{service availability}
   \item Web \textit{response time}
   \item \textbf{Uptime}
\end{itemize}

\section{Yale - BGM Guidelines}
\begin{enumerate}
   \item \textit{Break-glass} solutions are based on \textbf{pre-staged} emergency user accounts (created in advance, to carefully define respective access controls),
   managed and distributed in a way that can make them quickly available
   without unreasonable administrative delay. 
   In other words they should be \textit{simple},
   \textit{effective}, and \textit{reliable}.\\
   Account \textit{usernames} should be simple and \textbf{meaningful}, while \textit{passwords} should be \textbf{effective} but \textbf{easy to remember} by the admins.

   \item \textit{Account Permissions} should be set to \textit{\textbf{minimum} necessary privilege}.
   \item \textbf{Auditing} should be enabled if available, to log details of the account usage and
   details of the work carried out while using the account.
   \item The individuals who create the accounts are \textbf{not} those ones reviewing the audit trails, since this can be a source of abuse.
   \item The \textit{break-glass} accounts and distribution procedures should be \textbf{documented}
   and tested as part of implementation.
   \item BGMs require the emergency-account details to be made available in an appropriate and reasonable manner: such details may be provided on a media
   e.g. a printed page, a magnetic-stripe card, a smart card or a token. 
   Some distribution possibilities for break-glass accounts include the following:
   \begin{enumerate}
      \item Kept behind \textbf{actual glass} in a cabinet, where access to the accounts literally requires 
      to break the glass \footnote{similar to a fire extinguisher or alarm},
      providing an obvious indication that the accounts have been accessed and a deterrent to casual use.
      \item Maintained within \textbf{sealed envelopes}, where a broken seal would be an obvious
      indication that the accounts have been accessed.
      \item \textbf{Locked} in a desk drawer that only \textit{specific people} can access.
      \item Sealed and taped to the side of a monitor in a criical area,
      \textbf{visible} to many so it
      will be obvious when it is missing or damaged.
      \item For cases where \textit{more than one person} is needed to declare an emergency,
      \textbf{locked} in a safe or cabinet where one person knows the combination or has the
      \textbf{cabinet key} and a different person has the \textbf{room key}.
   \end{enumerate}
\end{enumerate}

\section{Testing}
A system designed for least privilege is more complex than one that grants any privilege to anyone,
so \textbf{testing} is needed.
Sadly, also testing is more complex too, because a problem (i.e. an error) arises any time an \textit{access right} is missing.
Testing and debugging to discover \textbf{legal rights} that are \textit{not} granted both
\textbf{differ} from those to check the correct implementation of functions.

A \textbf{testing infrastructure} monitoring a deployed system may be adopted but two problems follow:
\begin{enumerate}
   \item \textit{Testing \textbf{of}} least privilege to ensure that access is properly granted only to
   \textit{proper resources}.
   \item \textit{Testing \textbf{with}} least privilege to ensure that the \textit{infrastructure} for testing has only
   the access it needs.
\end{enumerate}

\section{Authentication \& Authorization}
\begin{center}
   \textit{\textbf{Authenticate}} $\longrightarrow$ verify the \textit{identity} of a user or process as we have seen\\
   \textit{\textbf{Authorize}} $\longrightarrow$ evaluate (to permit) a request from an \textit{authenticated} party
\end{center}
\textbf{Authorization} may consider additional parameters aside from \textit{identity} and \textit{operation} requested, such as:
\begin{enumerate}
   \item Requesting \textbf{device}
   \item Request \textbf{parameters}
   \item Metadata about \textbf{geolocation} of the device, \textbf{security status} of the device
\end{enumerate}

For both \textit{MultiParty authorization} (\texttt{MPAuto}) and \textit{3-Factor Auth/Auto} (\texttt{3F}), 
request authorization is confirmed by \textbf{another node},
beloging to a subset of more robust and secure node with the role of confirming choices of less robust.\\
An interesting case is when the "more robust node" is a \textit{smartphone},
which makes \texttt{3FA} very useful when a malware on a \textbf{workstation} tries to implement an attack,
but useless against \textbf{insiders} because they can freely use their smartphones.

\textbf{Temporary Access}
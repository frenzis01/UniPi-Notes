\chapter{AI - Discovering Vulnerabilities}
\begin{enumerate}
   \footnotesize
   \item \textit{If your data is poisoned or tampered, how would you know?}
   \item \textit{Are you training from user-supplied inputs?}
   \item \textit{If you train against online data stores, what steps do you take to ensure the security of the connection between your model and the data?}
   \item \textit{How sensitive is the data you train from?}
   \item \textit{Can your model output sensitive data?}
   \item \textit{Does the model only output results necessary to achieve its goal?}
   \item \textit{Does your model return raw confidence scores or any other direct output which could be
   recorded and duplicated?}
   \item \textit{What is the impact of your training data being recovered by attacking/inverting your
   model?}
   \item \textit{If confidence levels of your model output suddenly drop, can you find out how/why, as well as the data that caused it?}
   \item \textit{Have you defined a well-formed input for your model? What are you doing to ensure inputs meet this format and what do you do if they don’t?}
   \item \textit{If your outputs are wrong but not causing errors to be reported, how would you know?}
   \item \textit{Do you know if your training algorithms are resilient to adversarial inputs on a mathematical level? What about its goal?}
   \item \textit{How do you recover from adversarial contamination of your training data}
   \item \textit{Are you using Reinforcement Learning on uncurated public content?}
   \item \textit{Start thinking about the lineage of your data: were you to find a problem, could you track it to its introduction into the dataset? If not, is that a problem?}
   \item \textit{Know where your training data comes from and identify statistical norms in order to begin understanding what anomalies look like}
   \begin{itemize}
      \item What elements of your training data are vulnerable to outside
      influence?
      \item Who can contribute to the data sets you’re training from?
      \item How would you attack your sources of training data to harm a
      competitor?
   \end{itemize}
\end{enumerate}

\begin{center}
   The real question is \emph{"Can you discover whether your data has been \emph{"attacked"} in some way?}
\end{center}
\newpage   
\section{Mitigation}
Mitigation addresses how can we enrich the \textbf{robustness} against AI data attacks, and mitigate the eventual impact;
currently the key point in mitigation is improving the \textbf{quality} of data, not patching or adding security mechanisms (which should be used anyway).

\begin{table}[]
   \centering
   \caption{Mitigation techniques summary}
   \label{tab:mitigation_summary}
   \begin{tabular}{@{}llll@{}}
   \toprule
   \multicolumn{2}{|l|}{Attack}                                                                                                      & \multicolumn{1}{l|}{Model enhancement approach} & \multicolumn{1}{l|}{Model-agnostic approach}        \\ \midrule
   \multicolumn{1}{c}{}                           & & {Enhance data quality}     & {Output Restoration}           \\
   \multicolumn{1}{c}{}                           & & {Data sanitazation}        & \\
   \multicolumn{1}{c}{}                           & \multirow{-3}{*}{{Poisoning attack}} & {Block Poisoning}          & \\ \cmidrule(l){2-4} 
   \multicolumn{1}{c}{}                           & & {Enhance data quality}     & {Trigger Detection}            \\
   \multicolumn{1}{c}{}                           & & {Data sanitazation}        & {Trigger Deactivation}         \\
   \multicolumn{1}{c}{}                           & & {Trigger Detection}        & {Backdoor Detection}           \\
   \multicolumn{1}{c}{\multirow{-7}{*}{{Training}}} & \multirow{-4}{*}{{Backdoor}}         & {Model Restoration}        & \\ \midrule
   & & {Data preprocessing}       & {AE detection}                 \\
   & & {Model Hardening}          & {Input Restoration}            \\
   & \multirow{-3}{*}{{Evasion attack}}   & {Robustness evaluation}    & {Output Restoration}           \\ \cmidrule(l){2-4} 
   & & {IP management}            & {Queries number bound}         \\
   & & & {Stealing detection}           \\
   & & & {Output obfuscation}           \\
   & \multirow{-4}{*}{{Model Stealing}}   & & {Fingerprint}                  \\ \cmidrule(l){2-4} 
   & & {Embed data privacy}       & {Queries number bound}         \\
   \multirow{-9}{*}{{Inference}}                    & \multirow{-2}{*}{{Data Extraction}}  & {Training with privacy}    & {Obfuscated confidence scores} \\ \bottomrule
   \end{tabular}
   \end{table}

\subsection{Poisoning - Data Sanitization}
\note{
   We only focus on \textit{data sanitazation} for what concerns poisoning attacks.
}
A method to mitigate data \textbf{poisoning} is \textbf{RONI} (\textit{Reject On Negative Impact}), which relies on identifying outliers by training the model with and without each (data) point and comparing the \textit{performance}.\\
Due to frequently retraining its run-time \textbf{overhead} is significant,
and as a result its performance is worse than later proposals.

An alternative is \textbf{TRIM},
which iteratively estimates the model parameters and trains on the
subset of best-fitting input points at the same time, until convergence is reached.\\
It is much better than RONI \smiley.
However, TRIM is devised for \textbf{linear regression} only and thus \textit{not} applicable to (deep) neural networks.

There are also \textbf{provenance-based} techniques, which use metadata about data provenance\footnote{the model assumes such provenance metadata is correct}
to cluster data accordingly.\\
Due to the additional information, it achieves \textbf{better results} than RONI and
is \textbf{more efficient} by the average cluster size, since the model is \underline{not}
retrained \underline{for each} individual point but only the \textit{cluster centroids}.\\
The intuition is that the probability of poison for data points of common
provenance is strongly correlated.

\textbf{Keyed Non-parametric Hypothesis} Tests (\textit{KNHT}) assume a set of clean training data describing intended \textit{data distribution} and
inspect newly-collected ones to compare the two sets and if their similarity
is insufficient, the newly collected ones are further inspected.\\
KNHT does comparison after mapping the distributions into another space
via functions with secret keys.

\subsection{Backdoors and Triggering}
Backdoors attacks involve two steps:
\begin{enumerate}
   \item \textit{Backdoor embedding}
   \item \textit{Triggering}
\end{enumerate}

In \textbf{development} stage, backdoors and triggers can be detected and
even removed by modifying the addressed model;
in \textbf{deployment} stage instead, backdoor detection aims at revealing potential
weaknesses of the addressed model and trigger detection goal is to
prevent backdoors from being triggered.

Enhancing \textbf{data quality} and \textbf{sanitazation} are effective also against backdoor attacks,
since the backdoor embedding is a form of data poisoning.

\textbf{Trigger detection} is usually based either on \textit{\underline{spectral signature}} of the covariance of a feature representation learned by the neural network,
or on \textit{\underline{activation clustering}}, i.e. observing that triggers can be detected from the neuron
activation in the \textit{final hidden layer} of a network.

\textbf{Model restoration} methods {--}even if it sounds awkward{--} remove backdoors \textit{without} detecting them:
\begin{enumerate}
   \item \textbf{Retraining} the model with a small subset of clean training data
   \item \textbf{Fine-pruning} a network by eliminating some less important neurons leads to less \textit{accuracy}, but counters backdoor embedding
   \item \textbf{Neural-Cleanse}: 
   small input perturbations are reverse engineered from the
   model, so those that trigger backdoor behavior in the model can be identified and the model retrained with purified data.
   \item \textbf{TABOR} backdoor detection is transformed into a non-convex optimization
   problem with a new objective function to narrow down search space.
   Triggers found can be efficiently reverse engineered.
\end{enumerate}

\subsection{Inference attacks}

\subsubsection{Evasion}
In \textbf{Evasion} attacks attackers manipulate input samples to evade a deployed model at \textbf{inference} time.
\note{e.g. an attacker introduces crafted noise in a cat image such that the
manipulated cat image is classified as a dog image by the deployed model}

\labelitemize{\textit{Model Enhancement}}{
   \begin{enumerate}
      \item Data \textbf{preprocessing}
      \item Data \textbf{trasformation} to diminish possible perturbations, i.e. JPEG compression
      \item \textbf{NULL labelling} to label training examples resembling possible adversarial examples
   \end{enumerate}
}

\labelitemize{\textit{Model Hardening}}{
   \begin{enumerate}
      \item \textbf{Adversarial training}: model is trained with a data set augmented with correctly labelled adversarial examples
      \item \textbf{Regularization} prevents small input perturbations from changing model output
      \item \textbf{Certifiable training} implies multiple training phases, with first involving a data set augmented with \textit{Gaussian noises}, where the output is a probability distribution over all classes;
      the final model is trained via the augmented dataset with labels indicating the class with the highest probability.
      \item \textbf{Gradient masking}: to defend against gradient-based attacks, shattered gradients introduce non-differentiable operations such that attackers cannot get correct gradients for crafting adversarial
      examples.
      \item \textbf{Distillation Network}: A distilled network using the output probabilities emitted by a teacher model trained early on.
   \end{enumerate}
}

\subsubsection{Model stealing}
In \textbf{model stealing} attacks attackers without knowledge of the addressed model
try to steal model capabilities; 
two approaches possible:
\begin{enumerate}
   \item stealing model parameters by various means, such as side channel attacks or simply reading out model parameters from the underlying device
   \item generating a substitute model by exploiting the leaked information from a set of model input and output.
\end{enumerate}

To protect the copyright of DNN models, the \textbf{Watermarking} technique involves embedding watermarks in the training dataset that enable to verify the ownership of deployed DNN services.

The method creates the watermarks in the training dataset, then assigns pre-defined labels to the different watermarks, and finally trains the watermarks with
pre-defined labels to DNNs.

\subsubsection{Data Extraction}
In \textbf{data extraction} attacks attackers with ability of querying the
addressed model attempt to infer the training dataset, violating data confidentiality.
They include \textbf{Model inversion} and \textbf{Membership inference} attacks.

Existing mitigations against data extraction attacks focus on embedding data
privacy, training with privacy and limiting leaked information.

\section{Robustness Evaluation}
\textbf{Adversarial Testing} consists in testing the robustness of model with respect to adversarial examples crafted using various methods.\\
To obtain \textit{\underline{certifiable} robustness} there are some \textbf{Model verification} techniques based on the \textit{distance} between crafted adversarial examples $\bar{x}$ and corresponding intact inputs $x$.

\subsection{Google, OWASP, MITRE}

For a summary on Google, OWASP and MITRE documentation onto AI vulnerabilities refer to the last $~20$ slides of the course. 
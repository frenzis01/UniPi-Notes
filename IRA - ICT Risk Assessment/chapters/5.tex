\chapter{Discovering Vulnerabilities}
\section{Classification}
To address the problem of finding out vulnerabilities many classifications have been proposed,
and each one has its own purpose: module affected, how to discover it, enabled attacks, ...\\
It is mandatory to understand a classification goal before using it.

It is possible to consider where the vulnerability resides to provide some sort of classification:
\begin{enumerate}
    \item \textbf{Procedural}: the actions executed are not correct
    \item \textbf{Organization}: actions well defined but wrongly executed
    \item \textbf{Tool}: actions well defined and correctly executed but by bad tools, e.g. OS, compiler, run time support...
    \note{\textit{Password transmitted in clear, missing checks on boundaries...}}
\end{enumerate}

About \textbf{tool vulnerabilities}, it is important to pay attention to \textbf{code reuse}.
It must be considerend that reusing code may mean re-enable a vulnerability in such code.
Code reuse is ok, but only along with \textbf{code Hardening}, which means, removing instructions and libraries which are not needed.

About the implementation, it is relevant to avoid missing controls on stuff like user input, function parameters, confused program-flow...
Generally a strong type system may aid to address these kind of problems.

Besides, to avoid structural vulnerabilities, it is crucial to check whether certain modules depends on the security checks performed by others.

\section*{Searching for Vulnerabilities}
Aside from the distinction between \textit{Local} and \textit{Emerging} vulnerabilities, it is also important to distinguish between \textbf{standard} modules (OS, web servers...) and \textbf{specialized} ones (dynamic pages produced by the server).

\section{Vulnerability Life-Cycle}
\begin{enumerate}
    \item Born when someone does something wrong
    \item Known when someone discovers the error
    \item Public when its presence is revealed and it is inserted in some public database
    \item Some look for a remedy/fix while others search for an exploit\footnote{A program to implement an attack that exploits it}
    \item Vulnerability might become exploited
    \item If existing, the fix should be applied ASAP
\end{enumerate}
\note{Note that this life-cycle doesn't take into account a \textbf{zero-day} vulnerability, which is a vulnerability not made public, whose discovery is shared only among few teams or people}

Historically the most dangerous vulnerabilities are public and exploited ones;
even the oldest ones are exploited because attackers are lazy and defenders even more.

\section{Attacker vs Owner POV}
Considering the point of view of an \textbf{owner} who wants to search for vulnerabilities to improve their system's security.\nl

In order to search for vulnerabilities, an inventory of \textbf{all} the system modules is required.
This is not a trivial task, but it is necessary. 
\note{\textit{You cannot protect what you don't know.}}


The opposite view is the \textbf{attacker}'s one.
Usually vulnerabilities of standard modules are \textbf{known}, thus an attacker may only need to know which \textit{modules} compose the system.
An attacker may acquire knowledge on the vulnerabilities from public or private (by paying) databeses,
or by buying such information in the deep web.
It's rare for an attacker to look by himself for vulnerabilities in a module.\nl


\section{Scanning}

\subsection{Fingerprinting}

\textbf{Active fingerprinting} is a (set of) tool which exploits the fact that modules communicate through ports, and appears quite appealing from both the mentioned views:
given a range of IP addresses, it sends packets on each port and analyzes the replies to \textit{fingerprint}\footnote{i.e. "discover the identity of"} the module listening on the port.\\
The \textit{owner} might want to run the tool on the entire network, while an attacker may target single (or a few) hosts at a time.


\textit{Active fingerprinting} is noisy and might considerably slow down the network performance, which in some systems, e.g. \textit{ICS} (\textit{Industrial Control System}), must be avoided.\\
(TODO - CAPIRE MEGLIO)\nl

A \textbf{Passive fingerprinting} does not imply direct interaction with modules, but acts as a \textbf{sniffer}, analyzing packets the modules exchange in a transparent way.
It exploits info in TCP and IP headers to fingerprint modules.
The counterpart is that in networks with low noise/packets exchange, passive fingerprinting may take long to discover all the features of interest.
Usually cannot be used by attackers.\nl

It is important to note that a scanner may not know whether a \textbf{patch} has been applied or not to a module,
hence it may report vulnerabilties which in fact have been patched, generating a \textbf{false positive}.
A scanner may also generate \textbf{false negatives}, since some vulnerabilities for a given module may not appear in the DB the scanner uses to map modules to related vulnerabilities.
There are also some \textbf{breach and simulation} tools which besides scanning also execute an exploit to check whether given vulnerabilities have been patched or not.
Even though interesting, it may be dangerous to run such software in low-tolerance systems.


To evaluate vulnerabilities discovery methods it is common practice to use a \textbf{confusion matrix}\footnote{\href{https://en.wikipedia.org/wiki/Confusion_matrix}{wikipedia.org/wiki/Confusion\_matrix}},
which provides, amongst others measures, \textit{accuracy}, \textit{precision}, \textit{recall}\footnote{sensitivity}, \textit{specificity}.\nl

\subsection{Stealth scanning}
Clearly, the owner is interested in discovering if anyone aside from him is currently scanning the system.
An attacker may configure message frequency and the number of nodes to scan, to reduce the chance to be detected.

\subsection{More on scanning}
An owner may combine:
\begin{itemize}
    \item External vulnerabilities scan: Try to access the sytem from outside, to understand what can an attacker discover before starting an intrusion.
    \item Internal vulnerabilities scan: Aims to test and fingerprint devices and modules inside a network.
    It might be ran by either owner or attacker after the initial access.
    \item Intrusive scans: i.e. breach and simulation. These are the most stringent scans, but may be disruptive.
\end{itemize}

Anyway note that it is crucial to \textbf{periodically scan} a system, due to its eventually mutable nature but way more importantly, because about 20 new vulnerabilities get published every day, 
along with new potential attackers and attack techniques.

\section{Searching in a Module}
Vulnerabilities can be searched and assessed when designing the system.
Modules can be standard (e.g. OS), and thus be affected by public vulnerabilities, or specialized modules, whose vulnerabilities are unknown.

\section*{2 - Ottobre}

\textit{Static Application Security Testing} (SAST) indicates tools of static analysis, which has advantages like scalability and easy patching,
but is limited on other aspects, such as:
\begin{itemize}
    \item Authentication Issues
    \item Dangerous Management of access rights
    \item Unsafe use of cryptography
    \item Many False negatives
    \item Cannot evaluate runtime values
\end{itemize}

Opposed to static analysis, there are \textit{Dynamic Analysis} techniques on which big companies generally rely on.
The most common technique is \textbf{Fuzzing}.

\subsection{Fuzzing}
\textbf{Fuzzing} is associable to \textit{"Chaos Monkey"} test design paradigm:
the idea is to send \textit{malformed inputs} to a module.\\
If the system responds with a crash to malformed inputs, such crash may indicate a bug, i.e. a \textit{vulnerability}.\\
A fuzzing tool results from the composition of three modules:
\begin{center}
% \begin{tikzpicture}[shorten >=1pt,auto,node distance=2.8cm]
%     \node[shape=circle,draw=gray,align=center] (A) at (0,0) {Interpreter \\ Test \\ Modulea};
% \end{tikzpicture}
\begin{tikzpicture}
    \node[shape=ellipse,draw=gray] (A) {\textit{Malformed input generator}};
    \node[shape=ellipse,draw=gray, right=1cm of A] (B) {\textit{Input scheduler}};
    \node[shape=ellipse,draw=gray, right=1cm of B] (C) {\textit{Monitoring crashes tool}};
\end{tikzpicture}
\end{center}

Before executing a fuzz test, a \textbf{tainting analysis} is perfomed.
Its aim is to computer for each input the set of variables that could be affected by the input.
Usually this is done along with a coloring analysis, i.e. \textit{TODO}.

(...)

\textbf{Black-box} fuzzing makes no assumptions on the implementation details of modules.
It has grown a lot in popularity since it is faster and, more importantly, nowdays many modules, especially in IoT sector, do not provide open source code.
\nl

Some rules to efficiently apply fuzzing should be considered:
First of all, input-format knowledge should be kept in mind especially for black-box debugging;
in general the longer you run a test, the more bugs you may find, until a saturation point is reached;
It is advisable to use different fuzzers, since they may find different bugs.

\section{Web Vulnerability Scanner}
Despite the name, it doesn't work as other scanners: 
its focus is onto discovering vulnerabilities in a website whose behaviour is determined by the dynamic generation of pages.
Passwords and credentials may be given as input to look deeper into the website.
Such scanners work much more like to \textit{breach and simulation} then to regular scanners.

\begin{itemize}
    \item SQL Injection: inserting or deleting information from a database
    \item XSS: inserting a malware on a website to be later downloaded and executed by the end user (\textit{cross site scripting})
    \item CSRF: forces an end user to perform unwanted actions on a website he is authenticated. (\textit{cross site request forgering}) 
\end{itemize}

\section*{3 - Ottobre}

\section*{5 - Ottobre}

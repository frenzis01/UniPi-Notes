\chapter{Vulnerability, Attack, Intrusion}
\section{Vulnerability}
A \textbf{vulnerability} is a defect in a person, a component or a set of rules which enables a threat agent to execute an \textbf{attack}:
action that grants access rights that violate the security policy.

In short,
\begin{center}
    \textit{A \textbf{vulnerability} is a bug that enables an \textbf{attack}}
\end{center}

\note{\textit{Every} vulnerability is a bug, but \textit{not} every bug is a vulnerability}

\section{Attack}
An \textbf{attack} is an action and/or the execution of some code that may grant to the person or the module that executes it some illegal access rights, and it is related to a vulnerability that enables it.\\
The output of an attack is \textbf{stochastic}, it may fail according to a probability distribution.


\section{Threat Agent}
A \textbf{threat agent} is a source of \textbf{attacks},
it may be \textit{natural} (floodings, earthquakes...) or \textit{man-made }(adversary with a goal).
\textit{Man-made} may be malicious or random (employee which clicks on something dangerous accidentally).

\begin{center}
    It is possible to \textbf{assess risk} only if assets, vulnerabilities and threat agents are known for a given system. 
\end{center}

\section{Intrusion}
An \textbf{intrusion} is a sequence of \textit{actions} and \textit{attacks} of a threat agent to reach its goal,
which initially owns its legal access rights and aims to gain illegal ones,
hoping to control {---} a subset of {---} an ICT/OT system.
Some actions may be actuals attacks, while others may collect information to discover possible attacks.
Such actions (and attacks) can be implemented by a program called \textit{exploit}.

Once a threat agent gained control over an ICT/OT system:
\begin{itemize}
    \item Collect and exfiltrate infomration from the system
    \item Update any information in the system
    \item Prevent access to any resource/information in the system
\end{itemize}
\begin{center}
\includegraphics[width=0.7\textwidth]{images/intrusion_steps.png}
\end{center}
The steps of an intrusion include a recursive phase highlighted in \text{\color{red}red} in the picture;
it appears clear that an attacker cannot plan an entire intrusion in advance before starting it,
since an attack reveals information and (possibly) vulenon the system which the next attack will be based on.

\section{Initial Access}
A set of techniques that adversaries may use in an intrusion as entry vectors to gain an initial foothold within an ICT/OT environment.

Informations gathered through initial access are sold on the deep web to hackers team who aim to penetrate a system.

\section{Countermeasure}
The \textbf{attack chain} is the sequence of \textit{useful} attacks in an intrusion.
A defendant wants to increase the number of \textit{useless} attacks to slow down an intrusion.
Besides, it is not mandatory to remove \textit{all} vulnerabilities to prevent an in intrusion,
but even only one may be sufficient to interrupt te \textit{attack chain}, 
thus preventing the attacker from collecting information that would lead to further attacks.\nl

There are two main approaches when considering security:
\begin{itemize}
    \item \textbf{Unconditional security}: Assume that any vulnerability will be exploited regardless of costs and complexity
    \item \textbf{Conditional security}: Consider who is interested in attacking the system and which vulnerabilities their intrusion can exploit. 
\end{itemize}

\section{Risk assessment}
To wrap up, let's define what \textbf{Risk assessment and management} involves,
keeping in mind that \textit{cyber risk} resembles the average loss for instrusions.
\begin{enumerate}
    \item Asset analysis
    \item Threat agent analysis
    \item Vulnerability analysis
    \item Adversary emulation
    \item Impact analysis
    \item Risk evaluation and management: \textit{compute and minimize loss}
    \begin{itemize}
        \item Compute the risk
        \item Accept some risk
        \item Reduce some risk (countermeasures + scheduling)
        \item Transfer residual risk (insurance)
    \end{itemize}
\end{enumerate}
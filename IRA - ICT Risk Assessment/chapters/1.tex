\chapter{Introduction}
\section*{19 - Settembre}
\section{Info and Contact}

\note{Info on the exam...}

Question time Monday from 16:00

\section{ICT Security Fundamentals}
A system security policy should preserve:
\begin{enumerate}
    \item \textbf{Integrity} - Only users allowed to \textit{update} certain information are actually able to update it
    \item \textbf{Availability} - Users who want to \textit{use} the system must be able to use it in a finite and reasonable amount of time
    \item \textbf{Confidentiality} - Only users allowed to \textit{read} certain information are actually able to read it
\end{enumerate}

Depending on the adversary, the terminology changes
\[\text{Natural events} \Rightarrow \textit{Safety}\]
\[\text{Malicious and intelligent} \Rightarrow \textit{Security}\]

There are other properties regarding systems and their security:
\begin{enumerate}
    \item \textbf{Robustness} evaluates how well the system resists and it is not violated by an attack
    \item \textbf{Resilience} evaluates how well a system recovers and resumes its normal behaviour after it has been violated
    \note{Resilience is preferred to robustness because it is more cost effective}
    \item \textbf{Vulnerabilities} are defects in the system which reduce robustness, hence safety and/or security. \nl
    NOT every \textit{defect} is a vulnerability,
    but every vulnerability is a \textit{defect}.
\end{enumerate}

Secondarily there are properties derived from \textit{security}:
\begin{enumerate}
    \item Traceability - Discover who has invoked a given operation
    \item Accountability - Those who use a resource should pay for it
    \item Auditability - Whether the security policy is enforced and satisfied
    \item Forensics - Proving that some action has occurred and who has executed it
    \item Privacy/GDPR - Who can read and update a certain information
\end{enumerate}

\note{\textit{Forensics} is distinguished by \textit{Traceability} because it is related to what can be proved in a court of law,
not only who performed a single operation on the system.
Forensics refers to a set of information able to convince a non-expert that something }

\section*{21 - Settembre}

\section{Security policy - Glossary}
First of all an \textbf{Asset analysis} is required.
It is mandatory for a business to determine which resources are critical for their system to work,
allowing to focus security efforts on specific assets.\nl
It is also crucial to determine the \textbf{impact}:
\begin{itemize}
    \item A business process is stopeed (integrity or availability)
    \item A resource has to be rebuilt ex novo (integrity)
    \item Attacker discovers the information in the resource (confidentiality)
\end{itemize}

\textbf{Asset \textit{discovery}} is usually done through an application installed in specific assets and discovers all the assets in the company network.\nl

An \textbf{externality} is a cost or benefit incurred or received by a third party that has no control over the creation of that cost or benefit.
It's important also to consider this because often the security of an ICT system may depend on third parties factors and entities,
whose security isn't controlled by the owner of the system.\nl

\textit{Security} is a job \textit{shared} by many individuals,
hence there may be some \textbf{free riders}, i.e. individuals who tend to shirk and be negligent, and whose lack of effort affects security.\nl
There may be three prototypical case which define on which individuals depends the security of a system:
\begin{enumerate}
    \item \textbf{Weakest-link}: security depends on agents with the \textit{lowest} benefit-cost ratio. (worst-case scenario)
    \item \textbf{Best shot}: security depends on agents with the \textit{highest} benefit-cost ratio.
    \item \textbf{Total effort}: security depends on the sum of efforts of \textbf{many} agents.
\end{enumerate}
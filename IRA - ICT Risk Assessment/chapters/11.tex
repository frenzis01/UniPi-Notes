\chapter{Countermeasures}
\section*{20 - Ottobre}
\section{Robust Programming}
Ideally it indicates a programming style focused on minimizing vulnerabilities and the impact of any vulnerability still exploitable.

\textbf{Robust programming} can be summarized with a few guidelines:
\begin{enumerate}
   \item \textit{Validate} program inputs aka input is evil
   \item \textit{Prevent} buffer overflow aka check sizes
   \item A robust implementation minimizes any \textit{information leaked outside} 
   e.g. module, object, function ...
   \note{
      \begin{itemize}
         \item Logical pointers rather than physical ones
         \item Validate any information that is exchanged
      \end{itemize}
   }
   \item Check values transmitted to other functions (egress filtering)
   \item Check returned results
\end{enumerate}
Besides, it is important to focus also on \textbf{interaction controls},
robustness must be enforced on both malicious and erroneous behaviour.

\subsection{Input validation}
Usually input validation is achieved with a form of \textit{default \textbf{deny}} by defining a legal input structure and discarding every input which doesn't satisfy it.
\note{In case of string this may be done through RegEx, max length, ...}

It is important that the checks to validate the input should be specified when the program is designed rather than after an attack;
besides a check should be designed in an simple and readable way,
to easily ensure its correctness.
Some examples of input which usually must be validated are:
\begin{itemize}
   \item Environment variables
   \item File names (blanks, .., /)
   \item Email addresses
   \item URL
   \item HTML headers/body
   \item Data
\end{itemize} 

Memory allocation and strings length is a crucial aspect:
only library functions with an explicit string length specified should be used\footnote{e.g. \lstinline|strncpy()| instead of \lstinline|strcpy()|},
and in general,
it is appropriate to
allocate only the memory actually needed by a data structure according to its sizeto avoid leaving
space to store dangerous values or inputs.\nl

Speaking of functions,
attention must be paid to a rigorous \textbf{interfaces definition} and to avoid making assumptions on relationships between input and output values of function;
in other words, if a function $A$ takes as input the value $x$ returned by $B$,
it must not be \textit{asserted} that $x$ is for sure a \textbf{valid} value,
$B$ should check the correctness of the input regardless of knowing how it was generated.

\subsection{CWE - Vulnerabilities Ranking}
\href{https://cwe.mitre.org/top25/archive/2023/2023_top25_list.html#tableView}{This article} by \textit{\textbf{CWE}} (\textit{Common Weaknesses Enumeration}) lists the most dangerous and frequent software weaknesses of 2023,
based on data provided by \textit{NIST}.

The scoring formula to calculate a ranked order of weaknesses
considers the \textbf{frequency} a CWE is the root cause of a vulnerability
with the \textbf{severity} of its exploitation. Both frequency and severity are
\textit{normalized} relative to the minimum and maximum values seen.
\textbf{Frequency} is obtained by counting weaknesses occurences in the \textit{National Vulnerabilities Database} (NVD),
while \textbf{severity} is the average computed on the Vulnerabilities score in the \textit{CVSS}\footnote{\textit{Common Vulnerabilities Scoring System}} a given weakness is mapped to.\\
The \textbf{final weakness score} is computed by multiplying frequency and severity scores.

\subsubsection{Biases and limitations}
There are two biases which CWE doesn't take into account,
which somehow negatively affect how valid CWE's scores are:
\begin{enumerate}
   \item \textbf{Metric bias}
   \begin{enumerate}
      \item 
      Indirect prioritization of implementation faults over design flaws
      \item 
      Prefers frequency over severity due to distributions of real-world
   \end{enumerate}
   \item \textbf{Data bias}
   \item \begin{enumerate}
      \item Only uses NVD data based on publicly-reported CVE Records
      \item Many CVEs do not have sufficient details to assign a CWE
      mapping, omitting them from ranking
      \item There may be an over-representation of certain programming
      languages, frameworks, or weakness-detection techniques
   \end{enumerate}
\end{enumerate} 

There also a few aspects which this scoring system cannot represent and should be taken care of.
First of all,
weaknesses that are rarely discovered will not receive a high score,
regardless of the consequence of an exploitation.
Weaknesses that begin with a root cause of a mistake leading to
other mistakes, create a chain relationship.
As we have seen, chains of mistakes/vulnerabilities/attacks are a key point in security,
but CWE's scoring system treats any $\langle V_1, V_2 \rangle \wedge V_1 \rightarrow V_2$ as if $V_1$ and $V_2$ were independent i.e. $V_1 \not\rightarrow V_2$.


\section{Firewall}
A \textbf{firewall} is a module to filter all the messages exchanged by \textit{two} networks with a distinct security level;
\textit{all and only} the messages travelling on the wires
connecting the two networks cross the firewall and therefore get filtered.
A firewall works correctly under the assumption that a network has been split (\textit{segmented}) into two \textbf{subnets}, 
and that it \textit{correctly implements} a security policy, which should \textit{\textbf{not}} define the policy by itself.
Firewalls are usually \textbf{classified} on the known and manageable \textbf{protocols} and on their \textbf{implementation}.

\subsection{Segmenting}
Firewalling goes along with \textbf{segmenting} a network,
which results in multiple subnets with different security levels whose interaction is determined by firewalls inbetween them.
This architecture increases \textbf{robustness} by preventing an attacker from having \textbf{initial access} on an entire system and from freely performing \textbf{lateral movements};
besides this architecture perfectly integrates with \textbf{honeypot} deception mechanisms.

\subsection{Classification}
Firewall may operate on different levels of the TCP/IP stack and in different ways:
\begin{itemize}
   \item Packet filtering firewall
   \item Circuit-level gateway
   \item Application-level gateway (aka \textit{Proxy Firewall}):
   firewall which recognizes application level protocols and can make assumptions on it
   \item Stateful inspection firewall
   \textit{Stateful} means that the firewall inspects also the contents of a communication and the properties related to the status of a connection.
   \item Next-generation firewall (NGFW)
\end{itemize} 

At level 3 (IP Packet Inspection) the firewall can check only the header of IP packets,
while at level 4 (circuit level firewall) ...\nl

TODO\nl

\subsection{Pros \& Cons}
\begin{itemize}
   \color{darkgreen}
   \item A Single device can filter traffic for an entire network
   \item Extremely fast and efficient in scanning traffic
   \item Inexpensive
   \item Minimal effect on other resources, network performance and end-user experience
\end{itemize}

\begin{itemize}
   \color{darkred}
   \item Since traffic filtering is entirely based on IP addresses and ports,
   it lacks broader context that informs other types of firewalls
   \item Doesn't check the payload and can be easily spoofed
   \item Not an ideal option for every network
   \item Access control lists can be difficult o set up and manage
\end{itemize}
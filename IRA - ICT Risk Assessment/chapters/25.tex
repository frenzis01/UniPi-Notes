\chapter{Stuxnet}

In the summer of 2010 \textbf{Stuxnet}, a malware of unprecedented complexity made the
news, by exploiting \textbf{multiple} (3) zero-day exploits.

\textbf{Stuxnet} didn’t act like any previous malware before and can be considered pioneristic from many points of view, 
but mostly
because its objective was \textbf{not} the theft or manipulation of data,
but the \textit{\textbf{physical destruction} of gas centrifuges} in the Natanz fuel
enrichment plant, the crown jewel of Iran’s nuclear program.\\
So, Stuxnet was not implementing a strict ICT attack, because its purpose was to cause a physical effect.
\note{Does "IoT" ring a bell?}

In ICS most automated factory processes are ran by \texttt{PLC}s (Programmable Logic Controllers),
which can be programmed (typically from Windows) using basic scripting or GUI-aided logic.
However, they typically are not connected to the internet, thus creating an \textit{"air-gap"}, between the internet and the network between the \texttt{PLC}s.\\
Stuxnet targetted a specific PLC control system, \texttt{SIMATIC PCS 7 Process Control System}, programmed using WinCC/STEP 7.

Stuxnet distributed mostly through USB keys, bypassing the air gap.

\nl

The attack was never reclaimed by any group; even though there were some hidden clues in the source code, none of them were consistent, besides
\begin{center}
   \textit{"Symantec cautions readers on drawing any attribution conclusions.
   Attackers would have a natural desire to implicate another party"}
\end{center}
\chapter{Virtualization}

Virtualization consists in virtualizing hardware resources, such as CPU, memory, storage, and network interfaces. This allows to run multiple operating systems on the same physical machine, which is called \textit{host}.

It is not equal to \textit{emulation} which consists in simulating hardware, and is much slower than virtualization.

TODO more on this

\ul{Virtualization is a strong way of isolating things.}

There are two kind of virtualization systems:
\begin{enumerate}
   \item VirtualPC (Microsoft), VirtualBox (Oracle), VMware Workstation (VMware), Parallels (Apple) : these are desktop virtualization systems, which are used to run multiple operating systems on the same physical machine.\\
   These solutions aim to provide ``interactive'' computers, with a GUI, peripheral support, etc. 
   \item VMware ESXi, Microsoft Hyper-V, KVM, Xen : these are server virtualization systems, which are used to run multiple servers, typically GUI-less, on the same physical machine.\\
   Hypervisors introduce a \textbf{crucial} piece of software called \textbf{Virtual Switch}, which is responsible for managing the network of the virtual machines.
   The virtual switch's uplink is the host's physical network interface.
\end{enumerate}

Similar to the ones in storage systems, there are \textbf{checkpoints}, which are used to save the state of a virtual machine at a certain point in time. This is useful to revert to a previous state in case of problems.
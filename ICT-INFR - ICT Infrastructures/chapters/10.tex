\chapter{Supercomputers}

Recall that \textbf{TOP500} is the list of the 500 most powerful computer systems in the world. The list is updated twice a year and the first one was published in June 1993. The list is compiled by Hans Meuer of the University of Mannheim, Erich Strohmaier and Horst Simon of NERSC/Lawrence Berkeley National Laboratory, and Jack Dongarra of the University of Tennessee. The TOP500 project aims to provide a reliable basis for tracking and detecting trends in high-performance computing and to provide a basis for tracking the progress of the supercomputing industry.


\section{Supercomputers in the TOP500 list}
It is interesting that \textbf{Microsoft} has applied to have a supercomputer in the TOP500 list. It is ranked 11th in the list and is located in the United States. The supercomputer is called \textit{Azure} and is operated by Microsoft. It has 2,596,016 cores and a performance of 27,580.0 TFlop/s. The supercomputer is based on the HPE Cray EX supercomputer and is used for commercial purposes.

\textbf{Leonardo} is the most powerful supercomputer in Europe and is located in Italy. It is ranked 7th in the TOP500 list and is operated by CINECA. The supercomputer has 14,000,000 cores and a performance of 32,800.0 TFlop/s. The supercomputer is based on the HPE Cray EX supercomputer and is used for research purposes.

\framedt{``\textit{Alps} is really really important'' -prof. Cisternino}{
   Alps is the most powerful supercomputer in the world and is located in Switzerland. It is ranked 1st in the TOP500 list and is operated by the Swiss National Supercomputing Centre. The supercomputer has 2,289,024 cores and a performance of 63,460.0 TFlop/s.

   According to Cisternino it is of utmost importance because its CPU is designed by NVIDIA and it is the first supercomputer to use this technology. The CPU is called \textit{Grace Hopper} and is based on the ARM architecture. The CPU is designed for high-performance computing and is used for research purposes.\\
   It may represent a turning point in the CPU architecture choice, and ARM may become the new standard for high-performance computing, superseding x86.\\
   \textit{Grace} is capable of transferring up to 1Tbit/s of data between the CPU and the RAM, which is ``quite a nice number''.

}

\subsection{Brexit and supercomputers}
Almost one third of the computers in the list are in Europe, 
and in the top 10 there are 3 European supercomputers (?).
Since most supercomputers were made in the UK, after the Brexit Europe focused on the need to have its own supercomputers. 
\note{This is why the \textit{Leonardo} supercomputer was built in Italy. }
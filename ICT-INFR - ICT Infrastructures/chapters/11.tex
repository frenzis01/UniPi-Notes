\chapter{Exam}

\begin{enumerate}
   \item 
   \textit{Do you think that cabling may affect the PUE?}\\
   Yes, because the cabling may affect the airflow and the cooling system, which in turn may affect the PUE.
   But not only this, because part of such PUE may be hidden in the ``compute'' part of a system:
   fans inside the servers, when detecting a higher temperature, may increase their speed, (or lower the CPU performance) thus consuming more energy.
\end{enumerate}

% Spine-leaf must be related to protocol SFP, because it is a protocol that is used to connect switches in a data center.

\section{Domande mie}

\subsection{Three-tier architecture: What does it mean that ``provides active-passive redundancy which leads to inefficient east west traffic''?}

Answer by Copilot: 
``The three-tier architecture provides active-passive redundancy, which leads to inefficient east-west traffic. This is because the traffic between servers in the same tier must go through the core layer, which can lead to congestion and increased latency. In contrast, a spine-leaf architecture provides active-active redundancy, allowing traffic to flow more efficiently between servers in the same tier.'' 
   

\subsection{LACP: even though the bandwidth is aggregated (i.e. $2\times 25Gbps$), the single stream is still limited to the bandwidth of a single link (i.e. $25Gbps$), because the traffic goes only from one way to the other each time.}

Answer by Copilot: 
``LACP (Link Aggregation Control Protocol) is a method of combining multiple physical links into a single logical link to increase bandwidth and provide redundancy. However, even though the bandwidth is aggregated (e.g., 2x25Gbps), a single stream is still limited to the bandwidth of a single link (e.g., 25Gbps) because the traffic is hashed based on the source and destination MAC addresses, IP addresses, or TCP/UDP ports. This means that traffic between the same source and destination will always use the same physical link, limiting the effective bandwidth of a single stream to that of a single link.''
\chapter{Containers}
\label{chap:containers}

\ul{A VM is better than a process because it provides \textbf{isolation}}:
a typical problem in cybersec is that if an attacker cracks a process, he may access the whole system;
with a VM, the attacker can only access the VM, not the host.
However, a VM introduces overhead due both to the hypervisor and to the OS (cache, kernel, storage management\dots).

\note{The idea is to tell a process that a process that its root is a subdirectory of the host's root, resulting in a strong isolation.}

\textbf{Docker} provides a \textit{differential filesystem}, which is a filesystem that is a diff of the host's filesystem, an abstraction where the new ``inner'' file system is based on a given one, where the system will only store the differences between the original file system and the new one.

Docker has become the de facto standard for containers, but there are other solutions like \textbf{LXC} (Linux Containers) and \textbf{rkt}.
A Dockerfile contains the information to build a container.

One of the key differences between a VM and a container is that containers \textbf{do \textit{not} have a virtual switch}.
A container uses the same MAC address of the host.\\
Docker containers are processes, and only see a portion of the host's filesystem.

Inside a Dockerfile you may create temporary containers by exploiting multiple images and lastly build the resulting slimmed image.

\section{Upgrading a system}

Typical procedure when a system needs to be updated: create first a snapshot before updating the containers (especially if there are updates related to the database), then update, run back the containers, check if everything works, and then delete the snapshot. If anything goes wrong, the snapshot will help to roll back.

\section{Docker compose}
Docker compose is a tool to define and run multi-container Docker applications. It uses a YAML file to configure the application's services, networks and volumes. With a single command, you can create and start all the services from your configuration.

Kubernetes is a more advanced tool for container orchestration, way more complex than Docker compose, it allows to manage thousands of containers, providing fundamental scalability features.

\section{Docker security}
An attack is to put the machine under heavy workload and observe from the container the performance to infer what other processes may be. This is called \textbf{side channel attack}.

Google has thousands of VMs and each runs a \textit{container for each query}.
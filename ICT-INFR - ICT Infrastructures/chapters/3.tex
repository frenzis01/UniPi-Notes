\chapter{Fabric}
``Fabric'' is the term used to refer to the \textit{interconnection} between nodes of a datacenter.\\
\ul{Cabling is of paramount importance.}
\note{Prof. Cisternino learnt it ``the hard way'' when he performed the cabling of the first UniPi datacenter by himself}
\begin{enumerate}
   \item Maintenance
   \item Cooling
   \begin{enumerate}
      \item Cables may heat up
      \item Cables may obstruct air flow
   \end{enumerate}
   \item Determines which machines interact with each other (\textit{fabric})
   \item Bandwidth
   \item Not neglectable cost
\end{enumerate}

We refer to North-South traffic indicating the traffic outgoing and incoming to the datacenter (internet), while we refer to East-West as the internal traffic between servers.
Most of the network (or fabric) traffic is processed horizontally (North-South traffic)\footnote{Seems odd that ``horizontal'' refers to North-South traffic, but that's how it is.}.


\section{Bandwidth and Storage implications}
\label{sec:bandwidth_storage}
A standard datacenter has servers connected with 25Gbit links in both directions, summing up to 50Gbit total bandwidth.
Current SSDs using NVMe provide much more, about $3.5 GB/s$, making \ul{4 drives are enough to saturate a 100Gbit/s link.}\\
We moved from a situation where the \textbf{bottleneck} were slow Hard Drives, to the current one where the bottleneck is the ---network--- \textbf{bandwidth}.\\
Recently the PCI 3.0, which lasted very long ---providing roughly$\sim\texttt{120Gbit/s}$ using 16 pin---, suddenly become unsufficient to handle the needed traffic.

Considering this, \ul{datacenters must be designed to allow \textit{Terabytes} of data to be moved in east-west traffic.}

\begin{center}
   \ul{\textit{The \textbf{fabric} is the glue that makes the datacenter possible.}}
\end{center}

Besides, a single server is \textit{unable} to handle 10TBs of data and handling requests from 3000 users simultaneously. It is necessary to \textbf{distribute} the requests.

HDDs are still currently used for \textbf{cold storage};
CPUs will access data exclusively from SSDs, and sometimes the server is shipped with on board \textbf{full-flash storage}.\\
The difference in price between SSDs and HDDs becomes negligible since you pay for top CPU, top GPU, top RAM;
furthermore, you can't waste ---the high amount of--- energy ---consumed by such components--- by waiting for a slow drive.

SSDs have a known write limit, but today, the usually last enough time: if you write the whole disk every day it will last for 5 years. Most-likely after five years you'd have to renew some components anyway, besides the failure is a predictable event.

\section{Cables and standards}
\subsection{Optical}
Electric current propagates at a speed $s = {\sim}0.6c$.
Hence \textbf{optical fiber} is ---at least in theory?--- faster.

\textbf{Lasers} are a coherent beam of equal fotons. It is possible to transfer energy through such fotons. Something resembling a laser is used for optical fibers.

Blu-Ray came out when scientists managed to create light using frequencies in the Blu area, which are the higher ones.
Currently, the best and most expensive optical fibers exploit blu-lasers as source of light.

Note that with optical you always need 2 fibers, one sending and the other receiving. The two possible connectors are \texttt{SC} and \texttt{LC}.
Sometimes the two ends of the cable are detachable so that the cables may be switched; this is useful because sometimes you may want to attach the TX cable on the RX plug and viceversa. 

\begin{figure}[htbp]
   \centering
   \includegraphics[width=0.25\columnwidth]{images/SC.png}
   \includegraphics[width=0.25\columnwidth]{images/LC.png}
   \includegraphics[width=0.25\columnwidth]{images/LC_coupled.JPG}
   \caption{\texttt{SC} and \texttt{LC} connectors}
   \label{fig:sc_lc_connectors}
\end{figure}

\subsection{Copper wires}
In case of electricity there are many aspects to be considered. Interferences, cable diameter/size, length, and also the fact that if a \texttt{1} has been transmitted for some time, it takes longer to transmit a \texttt{0}, due to the \textit{commutation} that must happen.
\nl

\begin{paracol}{2}
   \colfill
   \textbf{RJ45} is a standard physical interfaced for copper wires, which allows up to 1Gbit regularly.
   The \texttt{Cat 7} cables still use the RJ45 as connector and provide instead 10Gbit/s, but are very uncomfortable, they are so thick that they are difficult to bend.
   
   It is estimated that there have been installed $70 \times 10^9m$ of Ethernet cables, making them the most used.
   \colfill
   \switchcolumn

   \begin{figure}[htbp]
      \centering
      \includegraphics{images/RJ45_T568B.jpg}
      \caption{RJ45 - T568B}
      \label{fig:RJ45_T568B}
   \end{figure}
\end{paracol}

\subsection{SFP - Small Form-factor Pluggable}
There can be a cable with a LC in one side and a SC on the other side.
Instead of making switches with the optical plugs, switches were created with electrical plugs that would be able to host a \textbf{standard transceiver}.
The latter is a pluggable module that will receive current power and electrical signals for the transmission, which is responsible for transitioning between electrical signals and Optical signal (and viceversa).

\begin{paracol}{2}
   
   The aim of SFP is to decouple the optical transceivers from the server modules.
   \note{Is this correct?}
   They allow to go \textit{optic-copper}, \textit{copper-optic}, \textit{optic-optic} and \textit{copper-copper}.\\
   SFP and GBIC (oldest one, now dead) pluggable modules acting as active transceivers for optical wiring using RJ45 connector.\\
   \ul{A single cable having SFP ends costs about 100€}.
   The cost ain't neglectable \smiley.
   \note{\begin{itemize}
      \item[\texttt{SFP}] $\longrightarrow 1Gbit $
      \item[\texttt{SFP+}] $\longrightarrow 10Gbit $
      \item[\texttt{SFP28}] $\longrightarrow 25Gbit$
      \note{This is the current standard}
      \item[\texttt{QSFP28}] $\longrightarrow 4\times25Gbit$
   \end{itemize}}


   \switchcolumn

   \begin{figure}[htbp]
      \centering
      \includegraphics{images/sfp.jpeg}
      \caption{SFP transceivers form factors}
      \label{fig:sfp}
   \end{figure}
\end{paracol}

\note{Fun fact: ci sono 9 cavi USB-C e solo due portano informazioni video.}

\subsubsection{Issues about cabling and fabric}
The key point is that it would be desirable for cabling to be reconfigurable, that's why transceivers are so important.

\note{There are things called \textit{``Muffole''}, which are used for joining optical fiber cables, allowing for longer distances to be covered.
They are designed to be underground.}

Data traffic is always at least SFP+.
Current standard is SFP28. Various SFP are typically compatible, the shape of the plug should stay the same.
On switches there also some ports which are \texttt{QSFP+} or \texttt{QSFP28}, which allow up to \texttt{40} and \texttt{100Gbit/s} respecitively, and are used for north-south traffic.
\note{The \texttt{Q} letter stands for \textit{Quality}}

Switches for datacenters should be \textbf{non-blocking}, meaning that no port has to wait for other ones ---or any other thing--- before transmitting, they can also transmit simultaneously.


\ul{In every datacenter it is \textit{MANDATORY} to document the cabling.}

\subsection{InfiniBand}
Even though Ethernet is famous, it is not the only standard. InfiniBand is another one, which is used in supercomputers and known for its very high throughput and very low latency ($\sim 2\mu s$).
It may send messages up to 2GB each, with 16 priority levels.
It is a \textit{lossless protocol}, meaning that if a packet is received, its integrity is guaranteed.

\textit{IB} avoids TCP/IP stack, which is very heavy, and instead uses MPI (\textit{Message Passing Interface}), which is a way to distributed parallel programs, also exploited by OmniPath.

InfiniBand cables and connectors look similar to Ethernet ones, but they are not compatible.

Aside from HPC environments, it is uncommon to build an entire network with InfiniBand, typically there is an IB switch to whom the servers equipped with IB NICs are connected, which intereacts with the rest of the network with Ethernet, because ``it's cheaper and it works''.
\note{Today, we are about 400 Gbit/s on both IB and Ethernet.}

\subsection{RDMA - Remote Direct Memory Access}

RDMA is a technology API based (not a protocol!) that allows to access memory of a remote machine without involving the CPU or the OS of the remote machine.

RDMA supports zero-copy networking by enabling the network adapter to transfer data directly to or from application memory, eliminating the need to copy data between application memory and the data buffers in the operating system.\\
The main use case is distributed storage.

RoCE (\textit{RDMA over Converged Ethernet}) is a network protocol that allows remote direct memory access (RDMA) over an Ethernet network.

\subsection{Omni-Path}
Omni-Path is a high-performance computing network architecture, developed by Intel. It is a successor to Intel's InfiniBand, and competes with InfiniBand's EDR and HDR technologies.

Intel plans to develop technology that will serve as the on-ramp to \textit{exascale computing}\footnote{A computing system capable of the least one exaFLOPS}, which is the next frontier in high-performance computing.


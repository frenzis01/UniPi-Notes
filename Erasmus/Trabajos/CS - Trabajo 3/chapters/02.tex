\chapter{Tarea 2}

% TAREA 2.

%     Visualiza el siguiente vídeo:

% https://www.google.com/search?client=firefox-b-d&q=Kaspersky+Industrial+CyberSecurity+demonstration+for+Energy+sector+-+Bing+video+#fpstate=ive&vld=cid:fc74c561,vid:7LNtjWx17mA,st:0

%     Contesta a las siguientes preguntas:

%     Identifica en la red industrial que se describe en el vídeo:

% a. Lo componentes físicos

% b. Los componentes ciberfísicos

% c. Los componentes ciber

%     Identifica los tipos de ataque que se describen en el vídeo, y explica brevemente como piensas que se podrían implementar.

%     Explica cómo plantea la solución de Kaspersky en el vídeo la defensa frente a los posibles ataques.

\section{Componentes de la red industrial}

\begin{itemize}
   \item \textbf{Componentes físicos:}
   \begin{itemize}
      \item 100kv high-voltage incoming line
      \item Power transformer
      \item 10kV bus feeder
      \item Primary switching equipment
   \end{itemize}
   \item \textbf{Componentes ciberfísicos:}
   \begin{itemize}
      \item transformer protection
      \item 2 bay controllers
      \item Industrial Ethernet switch
      \item Router
   \end{itemize}
   \item \textbf{Componentes ciber:}
   \begin{itemize}
      \item Kaspersky Industrial CyberSecurity
      \begin{itemize}
         \item KICS for Nodes - Endpoint Protection
         \item KICS for Network - Anomaly and Breach Protection
         \item Centralized security management
         \item Kasperky Security Center - Manager installed on nodes
      \end{itemize}
      \item Firewall
      \item Remote Control Center tools
      \item SCADA server
   \end{itemize}
\end{itemize}

% Software / hardware / virtual appliance

% Passive traffic analysis

% No influence on network stability

% It can detect:

% . Unauthorized network access

% . Cyberattacks and intrusions
% . Unauthorized commands to industrial
% equipment
% · Technological parameters anomalies:
% - rules based
% . machine learning
% . Assets and its parameters

% . Abnormal dataflow on network map

\section{Ataque sobre la infraestructura industrial y defensa}

El ataque que se muestra en el video comprende tres fases, también se analizan las posibles implementaciones de cada una de ellas:
\begin{enumerate}
   \item \textbf{Breach} - Obtener acceso a un componente
   \begin{itemize}
      \item Instalar un malware a traves de un documiento \texttt{.pdf} en un USB, que cuando se abre, se conecta al atacante, que obtiene acceso la computadora infectada, que puede utilizar como fuente para futuros ataques en la red.
      \item \textit{Defensa} - KICS for Networks detecta una comunicación no autorizada entre la computadora infectada y una dirección IP externa, y detecta tamién un payload potencialmente peligroso, y envía una alerta Kaspersky Security Center.
      \note{Si puede bloquear el ataque aquí, pero en el video se supone que no lo haces para mostrar el comportamiento defensivo en las fases siguientes}
   \end{itemize}
   \item \textbf{Discovery} - Obtener información y datos sobre el sistema
   \begin{itemize}
      \item A traves del componente infectado, el atacante puede escanear la red y obtener información sobre los dispositivos ciber y ciberfísicos conectados, lo que le permite establecer las vulnerabilidades actuales y planificar futuros ataques explotándolas.
      \item \textit{Defensa} - KICS for Networks detecta un escaneo de red no autorizado y envía una alerta a Kaspersky Security Center. Imagino que KICS también puede detectar cualquier movimiento lateral del atacante.
   \end{itemize}
   \item \textbf{Technlogical Attack} - Ataque a la infraestructura
   \begin{itemize}
      \item Si no se bloquea el ataque en la fase anterior, el atacante puede enviar comandos no autorizados a los dispositivos ciberfísicos.\\
      El ejemplo que se hace en el video es de utilizar una vulnerabilidad ---conocida--- del firmware del componente de protección del transformador para enviar un comando inapropiado para updatear el firmware de modo que el dispositivo deje de cumplir su función de protección.
      \item En este punto, el atacante puede causar daños físicos como cortocircuitos y similares
      \item \textit{Defensa} - KICS for Networks detecta un comando no autorizado y envía una alerta a Kaspersky Security Center.\\
      Aunque no se detenga el ataque, podemos utilizar la información recopilada para mitigar futuros ataques.
   \end{itemize}
\end{enumerate}

\framedt{
   GOOSE spoofing
}{
   GOOSE spoofing es un ataque que consiste en enviar mensajes GOOSE (Generic Object Oriented Substation Event) falsos a los dispositivos de la subestación, lo que puede provocar fallos en el sistema.
}
\chapter{Tarea 1}


% Visualiza el siguiente vídeo:

% https://www.youtube.com/watch?v=dEvtsZNrCSQ

%     Contesta a las siguientes preguntas:

%     Identifica en el sistema de distribución de agua:

%     Lo componentes físicos
%     Los componentes ciberfísicos
%     Los componentes ciber

%     Identifica los tipos de ataque que se describen en el vídeo, y explica brevemente como piensas que se podrían implementar.
%     Explica cómo se plantea en el vídeo la defensa frente a los posibles ataques.

\section{Componentes del sistema de distribución de agua}

\begin{itemize}
   \item \textbf{Componentes físicos:}
   \begin{itemize}
      \item Reservoirs
      \item Tanks
      \item Valves
      \item Pipes
      \item Pumps
      \item Taps in houses
   \end{itemize}
   \item \textbf{Componentes ciberfísicos:}
   \begin{itemize}
      \item Sensors
      \begin{itemize}
         \item Water temperature
         \item Water pressure
      \end{itemize}
      \item Logic Controllers (PLCs) que, por ejemplo, pueden activar una valvula si una cisterna está casi vacía 
   \end{itemize}
   \item \textbf{Componentes ciber:}
   \begin{itemize}
      \item Networks
      \item Computadoras
   \end{itemize}
   \item SCADA (Supervisory Control and Data Acquisition)
\end{itemize}

\section{Tipos de ataque}
Components of cyberphysical systems often do not enforce strong security measures, making them a target for attackers, who can gain initial access to a system by exploiting their vulnerabilities.     
\begin{itemize}
   \item \textbf{Robo de datos (Stealing data):} Podrían implementarse mediante la infiltración en los sistemas SCADA para extraer información confidencial sobre la infraestructura, patrones de uso, o datos de clientes. Esto podría realizarse mediante malware especializado o aprovechando vulnerabilidades en el software de control. Es posible también a traves de \textit{packet sniffing}. 
   
   \item \textbf{Daño al equipamiento (Damaging equipment):} Manipulación de PLCs para operar bombas fuera de sus límites operativos. Similar al ataque Stuxnet que dañó centrifugadoras en Irán modificando frecuencias de operación.
   
   \item \textbf{Corte del suministro de agua (Cutting off water supply):} Cierre de válvulas o apagado de bombas mediante acceso no autorizado a HMIs (Human-Machine Interfaces).
   
   \item \textbf{Liberación de sustancias tóxicas (Releasing toxic chemicals):} Alteración de sistemas de dosificación química explotando PLCs con vulnerabilidades conocidas.
   
   \item \textbf{Ataques de interceptación (Eavesdropping attacks):} Captura de tráfico no cifrado entre sensores y controladores mediante herramientas como Wireshark. Muchos protocolos SCADA (Modbus, DNP3) carecen de encriptación, y, en cualquier caso, existen técnicas de Deep Packet Inspection (DPI) que permiten deducir información incluso de paquetes cifrados.\\
   
   \item \textbf{Denegación de servicio (DoS):} Saturación de interfaces de red de controladores RTU/PLC, o explotación de vulnerabilidades para inhabilitar dispositivos.
   
   \item \textbf{Ataques de engaño (Deception attacks):} Falsificación de lecturas de sensores mediante ataques man-in-the-middle en protocolos vulnerables como OPC UA. El ataque a Ukrainian Power Grid (2015) utilizó técnicas similares para ocultar cambios en el sistema.
\end{itemize}

Parte de estos ataques van a mostrar efectos evidentes en el sistema de distribución, pero los atacantes también pueden cubrir sus huellas manipulando los datos que se envían al sistema de control, engañando potencialmente tanto a humanos como a algoritmos.

\section{Defensa frente a ataques}
La mejor defensa para Water Distributio Networks es la simulación de ataques, sin embargo, actualmente no existe un método estándar para hacerlo.
``Attack models'' son modelos matemáticos que simulan los posibles comportamientos de un atacante.
El video muestra \texttt{epanetCPA} es una herramienta que funciona en MATLAB que permite, dado un modelo de ataque, de ejecutar el modelo en una red PA, que es un modelo industrial estandar de red de agua, y ver cómo se comporta el sistema.\\
\texttt{epanetCPA} controla tanto el estado físico del sistema como el estado cibernético emulado del sistema, así que puede comparar el comportamiento del sistema real con el comportamiento del sistema simulado.

Tras un estudio, se observó que ataques a distintos componentes conducen a resultados similares, entonces encontrar un comportamiento anomalo en el sistema puede ser insuficiente para determinar cual componente ha sido atacado, lo que hace necesaria una evaluación humana más completa.
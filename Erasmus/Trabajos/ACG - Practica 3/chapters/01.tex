\chapter{Práctica 3}
\lstset{language=Java}
\section{Errores en el código}

\subsection{CoffeeMaker}

\begin{paracol}{2}

\begin{lstlisting}
@Test
public void testAddInventory() {
    // Comprueba los valores iniciales
    Inventory inventory = coffeeMaker.checkInventory();
    assertEquals(15, inventory.getCoffee());
    
    // Agrega inventario
    boolean added = coffeeMaker.addInventory(5, 3, 2, 1);
    
    assertTrue(added); // Falla
        
\end{lstlisting}

\switchcolumn

\begin{lstlisting}
public boolean addInventory(int amtCoffee, int amtMilk, int amtSugar, int amtChocolate) {
    boolean canAddInventory = true;
    // ERROR!!! amtSugar > 0
    if(amtCoffee < 0 || amtMilk < 0 || amtSugar > 0 || amtChocolate < 0) { 
        canAddInventory = false;
    }
    else {
        ...
\end{lstlisting}
    
\end{paracol}

\begin{paracol}{2}
\begin{lstlisting}
@Test
public void testMakeCoffee() {
    coffeeMaker.addRecipe(recipe1);
    
    Inventory initialInventory = coffeeMaker.checkInventory();
    int initialCoffee = initialInventory.getCoffee();
    
    int change = coffeeMaker.makeCoffee(recipe1, 100);
    
    assertEquals(50, change);
    
    Inventory finalInventory = coffeeMaker.checkInventory();
    int finalCoffee = finalInventory.getCoffee();
    
    // Falla porque se suma en lugar de restar
    assertEquals(initialCoffee - 6, finalCoffee); 
}
\end{lstlisting}

\switchcolumn

\begin{lstlisting}
public int makeCoffee(Recipe r, int amtPaid) {
    boolean canMakeCoffee = true;
    if(amtPaid < r.getPrice()) {
        canMakeCoffee = false;
    }
    if(!inventory.enoughIngredients(r)) {
        canMakeCoffee = false;
    }
    if(canMakeCoffee) {
        // ERROR!!! Se anade en lugar de restar
        inventory.setCoffee(inventory.getCoffee() + r.getAmtCoffee());
        ...
\end{lstlisting}
\end{paracol}

\subsection{Inventory}
\begin{paracol}{2}

    \begin{lstlisting}
@Test
public void testSetChocolate() {
    inventory.setChocolate(5); // 15 + 5 = 20
    assertEquals(20, inventory.getChocolate());
    
    inventory.setChocolate(-5);
    assertEquals(0, inventory.getChocolate());
}
    \end{lstlisting}
    
    \switchcolumn

    \begin{lstlisting}[caption={Hay un \lstinline|+=| en lugar de \lstinline|=|}]
public void setChocolate(int chocolate) {
    if(chocolate >= 0) {
        Inventory.chocolate += chocolate; // ERROR!!!
    }
    else {
        Inventory.chocolate = 0;
    }
    
}
    \end{lstlisting}
\end{paracol}

\begin{paracol}{2}
    
    \begin{lstlisting}
@Test
public void testMultiplesInstancias() {
    // Crea una nueva instancia y verifica los valores
    Inventory inventory2 = new Inventory();
    
    assertEquals(15, inventory.getCoffee());
    assertEquals(15, inventory.getMilk());
    assertEquals(15, inventory.getSugar());
    assertEquals(15, inventory.getChocolate());
}
    \end{lstlisting}

    \switchcolumn

    \begin{lstlisting}[caption={No tiene sentido que estos valores sean estaticos, hay un inventario por cada instancia}]
public class Inventory {
    private static int coffee;
    private static int milk;
    private static int sugar;
    private static int chocolate;
    \end{lstlisting}
\end{paracol}

\subsection{Recipe}

\begin{paracol}{2}
    \begin{lstlisting}
@Test
public void testSetAmtSugar() {
   Recipe r = new Recipe();
   r.setAmtSugar(5);
   assertEquals(5, r.getAmtSugar());
   r.setAmtSugar(-5);
   assertEquals(0, r.getAmtSugar());
}
    \end{lstlisting}
    
    \switchcolumn

    \begin{lstlisting}
public void setAmtSugar(int amtSugar) {
    if(amtSugar >= 0) {
        // ERROR!!! 
        this.amtMilk = amtSugar;
    }
    else {
        this.amtSugar = 0;
    }
}
    \end{lstlisting}
\end{paracol}

\section{Leer el fichero JSON}

\begin{lstlisting}
public class Inventory {
    ...
    private void loadFromJson(String filename) throws IOException {
        // Crea el ObjectMapper de Jackson
        ObjectMapper mapper = new ObjectMapper();
        
        // Lee el archivo JSON
        File jsonFile = new File(filename);
        JsonNode rootNode = mapper.readTree(jsonFile);
        
        // Obtiene el array de ingredientes
        JsonNode ingredientes = rootNode.get("ingredientes");
        
        // Procesa cada ingrediente
        for (JsonNode ingrediente : ingredientes) {
            String nombre = ingrediente.get("nombre").asText();
            int cantidad = ingrediente.get("cantidad").asInt();
            
            // Establece el valor segun el nombre del ingrediente
            switch (nombre) {
                case "coffee":
                    setCoffee(cantidad);
                    break;
                case "milk":
                    setMilk(cantidad);
                    break;
                case "sugar":
                    setSugar(cantidad);
                    break;
                case "chocolate":
                    setChocolate(cantidad);
                    break;
            }
        }
    }
}
\end{lstlisting}
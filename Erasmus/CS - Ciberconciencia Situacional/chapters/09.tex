\chapter{Critical Infrastructure Protection}
\label{chap:critical-infrastructure-protection}

\section{Introduction}

\begin{definition}
 La Unión Europea establece que las infraestructuras críticas son
aquellas instalaciones, redes, servicios y equipos físicos y de tecnología de
la información cuya interrupción o destrucción pueden tener una
repercusión importante en la salud, la seguridad o el bienestar económico
de los ciudadanos o en el eficaz funcionamiento de los gobiernos de los
Estados miembros\\
\textit{CNPIC}
\end{definition}

\begin{definition}
Systems and assets, whether physical or virtual, so vital to the United States that the
incapacity or destruction of such systems and assets would have a debilitating impact
on security, national economic security, national public health or safety, or any
combination of those matter.\\
\textit{NIST}
\end{definition}

\begin{definition}
   [Critical Services]
``The subset of mission essential services required to conduct critical
operations''\\
``Function or capability that is required to maintain health, safety, the
environment and availability for the equipment under control''
\end{definition}

\begin{itemize}
	\item Energy Production and Distribution Services
	\item Dams and their operation
	\item Financial Services
	\item Nuclear Reactors, Materials, and Waste Sector
	\item Food and Agriculture Sector
	\item Water and Wastewater Systems Sector
	\item Healthcare and Public Health Services
	\item Emergency Services
	\item Transportation Systems
	\item Chemical Sector
	\item Telecommunications lnfrastructures and Services
	\item Information Technology Sector
	\item Defense Industrial Base Sector
	\item Critical Manufacturing Sector as manufacturers of metals, machinery,
automotive and transportation equipment and electrical equipment
producer
	\item Government Facilities
	\item Commercial Facilities
\end{itemize}

% // TODO
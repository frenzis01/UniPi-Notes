\chapter*{Tarea 1 - Conceptos complementarios de Ciberconciencia Situacional}

\begin{enumerate}
   \item 
   \href{https://www.youtube.com/watch?v=cVaX07btaiU}{youtube.com/watch?v=cVaX07btaiU}\\
   Este vídeo aborda el tema de la conciencia cibersituacional en la producción de OT.
   Entre los conceptos más relevantes mencionados se encuentran:
   \begin{enumerate}
      \item El Monitoring si divide en \textbf{\ul{Event} Monitoring} y \textbf{\ul{Network} Monitoring}, el primero basado en una tecnología de \ul{event collection} (SW) que se instala en los dispositivos que los generan, y el segundo basado en la \ul{heurística} sobre el tráfico de red. El vídeo señala cómo la heurística puede conducir a veces a falsos positivos y entonces sea necesaria interpretación humana.
      \item La importancia de definir ambos los scenarios de ataque y los de defensa: más precisamente, agregando raw event data se pueden identificar scenarios (secuencia de eventos)en una lista de \textit{\textbf{\ul{use-cases}}}, y a partir da uno \textit{use-case}, un técnico humano puede buscar en un \textit{\textbf{\ul{runbook}}} lo que tiene que hacer para mitigar lo \textit{use-case} de ataque.  
   \end{enumerate}
   \item \href{https://www.youtube.com/watch?v=Sn6c5s3WFWw}{youtube.com/watch?v=Sn6c5s3WFWw}
   \begin{enumerate}
      \item Este vídeo subraya la importacia de la ciberconciencia situaciónal  especialmente para hacer frente a \ul{``\textbf{unknown threats}''}, que no coinciden con ninguna regla o pattern específico ya conocido (algo como Zero-Day Vulnerabilities).
   \end{enumerate}
   \item \href{https://www.youtube.com/watch?v=4geDznrTdbQ}{youtube.com/watch?v=4geDznrTdbQ}
   \begin{enumerate}
      \item Este vídeo introduce el tema de la \textbf{\ul{priorización}}: en las organizaciones medianas y grandes, es habitual tener enormes cantidades de posibles amenazas, y es necesario priorizarlas para poder actuar de manera eficiente. La conciencia situaciónal puede ser de grande ayuda en este sentido.
      \item \textbf{Common Operating Picture}, parece referirse a evitar mantener la información divisa en ``silos'', y a entender cómo y qué datos \ul{\textbf{agregar}}, para obtener una visión más completa de la situación. Esta agregación de datos puede variar según la ``Mission'' de la organización. 
   \end{enumerate}
   \item \href{https://www.youtube.com/watch?v=T9bmqccjfkg}{youtube.com/watch?v=T9bmqccjfkg}
   \begin{enumerate}
      \item \textbf{\ul{Attack scenario graphs}} son una herramienta para visualizar los relaciones entre las vulnerabilidades de un sistema, y entonces cómo multi-step ataques pueden ser realizados.\\
      Estes grafos pueden ser relacionados con \textit{software dependency graphs}, para visualizar como uno step de ataque a un componente puede afectar otros componentes que dependen de él.
      \item El video destaca el aspecto de ``attack cascade'' también al hablar de la \textbf{\ul{superficie de ataque}}, cuya definición típica carece del concepto de daño de una brecha en la superficie al igual que los posibles pasos de ataque posteriores, limitándose a una visión más simple que sólo considera los entry points.
      \item Otro aspecto mencionado es la importancia y la dificultad de \ul{\textbf{agregar datos}} de diferentes fuentes, que ponen un desafío a la ciberconciencia situacional, así como la limitación de los modelos de scoring de las vulnerabilidades, que además de estar limitados por ellos mismos, necesitan ser relacionados con el contexto de la organización.
   \end{enumerate}
\end{enumerate}

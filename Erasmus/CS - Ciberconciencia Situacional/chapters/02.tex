\chapter{Tarea 1 - Conceptos complementarios de Ciberconciencia Situacional}

\begin{enumerate}
   \item 
   \href{https://www.youtube.com/watch?v=cVaX07btaiU}{youtube.com/watch?v=cVaX07btaiU}\\
   Este vídeo aborda el tema de la conciencia cibersituacional en la producción de OT.
   Entre los conceptos más relevantes mencionados se encuentran:
   \begin{enumerate}
      \item El Monitoring si divide en \textit{\ul{Event} Monitoring} y \textit{\ul{Network} Monitoring}, el primero basado en una tecnología de \ul{event collection} (SW) que se instala en los dispositivos que los generan, y el segundo basado en la \ul{heurística} sobre el tráfico de red. El vídeo señala cómo la heurística puede conducir a veces a falsos positivos y entonces sea necesaria interpretación humana.
      \item La importancia de definir ambos los scenarios de ataque y los de defensa: más precisamente, agregando raw event data se pueden identificar scenarios (secuencia de eventos)en una lista de \textit{\ul{use-cases}}, y a partir da uno \textit{use-case}, un técnico humano puede buscar en un \textit{\ul{runbook}} lo que tiene que hacer para mitigar lo \textit{use-case} de ataque.  
   \end{enumerate}
   \item \href{https://www.youtube.com/watch?v=Sn6c5s3WFWw}{youtube.com/watch?v=Sn6c5s3WFWw}
   \begin{enumerate}
      \item Este vídeo subraya la importacia de la ciberconciencia situaciónal  especialmente para hacer frente a \ul{``unknown threats''}, que no coinciden con ninguna regla o pattern específico ya conocido (algo como Zero-Day Vulnerabilities).
   \end{enumerate}
   \item \href{https://www.youtube.com/watch?v=4geDznrTdbQ}{youtube.com/watch?v=4geDznrTdbQ}
   \begin{enumerate}
      \item Este vídeo introduce el tema de la \textbf{\ul{priorización}}: en las organizaciones medianas y grandes, es habitual tener enormes cantidades de posibles amenazas, y es necesario priorizarlas para poder actuar de manera eficiente. La conciencia situaciónal puede ser de grande ayuda en este sentido.
      \item \textbf{Common Operating Picture}, parece referirse a evitar mantener la información divisa en ``silos'', y a entender cómo y qué datos \ul{agregar}, para obtener una visión más completa de la situación. Esta agregación de datos puede variar según la ``Mission'' de la organización. 
   \end{enumerate}
\end{enumerate}

\chapter{Cyber Intelligence Visualization}

Objetivo principal: producir para analistas y responsables de la toma de decisiones mecanismos útiles para
comprender, de un vistazo, la información relevante y las tendencias dentro de las enormes cantidades
de datos en bruto que les proporcionamos actualmente en las herramientas cibernéticas.

Las herramientas  de ciber inteligencia generan una gran cantidad de datos, en gran parte testuale, y es necesario que los analistas sean capaces de procesarlos y entenderlos de manera rápida y eficiente.

The needs for the cyber intelligence domain are pretty specific:
\begin{itemize}
   \item Breakdown the overwhelming amount of data into manageable pieces to find the data we are actually interested in.
   \item Topological representations to show the relationships among the elements.
   \item Adapt the representation to the timing and pace of the cyberspace.
   \item Coupling cyber space domain data with physical domain data.
\end{itemize}

Es frequentemente necesario representar multi-dimensional data en un espacio 2D o 3D, y utilizar visualizaciones interactivas para permitir a los analistas analizar empezando por la información clave más relevante y siguiendo con datos más finos.

Los puntos clave de la visualización de la inteligencia cibernética son:
\begin{itemize}
   \item Dimensionality reduction and complexity reduction.
   \item Assuming inhernet non-linearities and couplings
   \item Tools and visualization techniques are need to help in the iterative process:
\end{itemize}

\section{Visualization Charts}
\begin{table}[htbp]
   \centering
   \begin{tabular}{|c|c|c|c|c|}
\hline Area     & Bar       & BoxPlot      & Bubble       & Column \\
\hline Doughnut & ErrorBar  & FastLine     & Funnel       & Kagi \\
\hline Line     & Pie       & Point        & Polar        & Radar \\
\hline Range    & Spline    & StackedArea  & StackedBar   & StepLine\\
\hline
   \end{tabular}
   \caption{Basic tecniques for Cyber Intelligence Visualization}
   \label{tab:03/tecnicasVisualizacion}
\end{table}

\begin{paracol}{2}
   
   Los investigadores y profesionales descubrieron que las técnicas de visualización existentes no satisfacen las necesidades de representación del ciberespacio, mientras que la \emph{graph-based} visualización gráficos proporciona medios para mostrar datos interrelacionados multidimensionales en un gráfico de pocas dimensiones.\\
   Una tecnica eficiente para reducir las dimensiones de los datos es utilizar el color.

   \switchcolumn
   \begin{figure}[htbp]
      \centering
      \includegraphics{images/03/color.png}
      \caption{Relational color-based dimension reduction}
      \label{fig:03/color}
   \end{figure}
\end{paracol}

\begin{figure}[htbp]
   \centering
   \includegraphics{images/03/sunburst.png}
   \includegraphics{images/03/dendogram.png}
   \includegraphics{images/03/circlepacking.png}
   \includegraphics{images/03/interactivenodelinks.png}
   \caption{Graph-based visualización techniques}
   \label{fig:03/graphbased}
\end{figure}

\begin{figure}[htbp]
   \centering
   \includegraphics{images/03/ddos.png}
   \caption{Según el profesor, este gráfico es muy importante porque muestra que para identificar lo que es \emph{anormal}, es necesario saber lo que es \emph{normal}.}
   \label{fig:03/ddos}
\end{figure}

\subsection{Georeferenced visualizations and IP-Port mapping}
Hoy en día, mucha información del ciberespacio se acopla a magnitudes físicas del mundo real.
Por ejemplo, si conocemos la localización de una dirección IP, podemos determinar de donde proviene el ataque.
También es posible colorear un mapa según la distribución geográfica de los ataques.

Gráficos que muestran las conexiones abiertas a los puertos de un nodo de red, o entre nodos de red, pueden mostrar fácilmente la actividad de exploración.
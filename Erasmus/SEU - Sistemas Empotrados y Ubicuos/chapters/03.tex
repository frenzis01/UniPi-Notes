\chapter{FreeRTOS - Tasks}
\lstset{language=C}
\section{Crear una Tarea}


\begin{lstlisting}[caption={Definir la funcion de tarea},label={lst:task}]
   void vATaskFunction( void *pvParameters )
   {
       for( ;; )
       {
           -- Task application code here. --
       }

       /* Tasks must not attempt to return from their implementing
       function or otherwise exit.  In newer FreeRTOS port
       attempting to do so will result in an configASSERT() being
       called if it is defined.  If it is necessary for a task to
       exit then have the task call vTaskDelete( NULL ) to ensure
       its exit is clean. */
       vTaskDelete( NULL );
   }
\end{lstlisting}

\href{http://www.freertos.org/a00125.html}{\texttt{xTaskCreate()}}
es el componente fundamental de un sistema multitarea.
Probablemente la más compleja de todas las funciones del API.

\begin{lstlisting}
   portBASE_TYPE  xTaskCreate(
      pdTASK_CODE       		pvTaskCode,
      const signed char 		*const pcName,
      unsigned short    		usStackDepth,
      void 	          		* pvParameters,
      unsigned portBASE_TYPE uxPriority,
      xTaskHandle       		*pxCreatedTask
      );
      
\end{lstlisting}

\begin{itemize}
	\item \lstinline|pvTaskCode|\\
Puntero a la función que implementa la tarea (simplemente el nombre de la función)
	\item \lstinline|pcName|\\
Un nombre descriptivo para la tarea. FreeRTOS no lo usa, pero ayuda a la depuración.
	\item \lstinline|uStackDepth| es el tamaño de la pila en numero de palabras, no en bytes.
Por ejemplo, la pila de un Cortex-M3 tiene un ancho de palabra de 32-bits.
\lstinline|pvParameters|
El valor asignado a pvParameters será el valor pasado a la función de tarea a través del parámetro con el mismo nombre
\lstinline|uxPriority|
Define la prioridad con la que se ejecutará la tarea.
Las prioridades pueden asignarse desde 0, (la menor prioridad), a (\lstinline|configMAX_PRIORITIES-1|), que es la mayor.
Varias tareas pueden tener la misma prioridad
\lstinline|pxCreatedTask|
Devuelve un identificador para la tarea creada, para poder hacer referencia a ella en futuras llamadas del API
P.ej., cambiar la prioridad o eliminar la tarea
Se puede pasar NULL si no se va a usar
Valor de retorno, dos posibilidades:
pdTRUE : la tarea se ha creado con éxito.
\lstinline|errCOULD_NOT_ALLOCATE_REQUIRED_MEMORY|
tarea no creada por insuficiente memoria dinámica (heap) para asignar las estructuras de datos y pila necesarios.

\end{itemize}
\section{}
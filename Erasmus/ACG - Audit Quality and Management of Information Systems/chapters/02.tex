\chapter{Ejercicio - 17 febrero}
\label{chap:ejercicio-17-febrero}


Figura 1. Pregunta 1. ¿en qué nivel de madurez se encuentra el AC 9 Métricas?
El AC 9 Métricas ha cumplido con los 3 PV del del nivel Controlado y con los 4 PV del nivel Eficiente, entonces se encuentra en el nivel Eficiente.

Figura 1. Pregunta 2. ¿en qué nivel de madurez se encuentra el AC 5 Comunicación?
En vez, el AC 5 aunque ha alcanzado los 3 PV del nivel Eficiente, ha cumplido sólo con el primero de 4 PV del nivel Controlado, por lo que aún se encuentra en el nivel Inicial.  

Figura 2. Pregunta 1. ¿en qué aspectos se debería centrar la organización para mejorar su proceso de pruebas?
La MM de ejemplo de la Figura 2 muestra la priorización de los PVs de las Area Claves para una empresa que utiliza Scrum. De conformidad con el modelo TPI, es necesario alcanzar prima le letra A, luego la B, y así sucesivamente. Por lo tanto, la organización debería centrarse primero en mejorar las 5 primeras zonas clave, y después las 4 últimas,  juntos, de nuevo, con AC 2, 4 y 5;
respectivamente son:
1 - Compromiso de los Interesados
2 - Grado de Participación
3 - Estrategia de Pruebas
4 - Organización de las Pruebas
5 - Comunicación
----
1 - Profesionalismo del analista de pruebas
2 - Diseño de casos de prueba
3 - Herramientas de prueba
4 - Ambiente de pruebas

Despues de estas 9 áreas, la organización se encontrará en el nivel Controlado, y podrá empezar a mejorar todo su proceso de pruebas.
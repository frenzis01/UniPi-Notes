\section*{Información General para la evaluación}
Di lunedì seminarios fino a fine marzo...
I seminari sono parte del$ 25\%$ della observación
Da marzo iniziano le prácticas, che sono un altro $25\%$
$30\%$ è una prova scritta
$20\%$ è una prova orale 

\chapter{Calidad}

\section{¿Qué entendemos por ``sistema de información''}

\begin{definition}[Sistema de información]
   Conjunto único de hardware, software, bases de datos,
   telecomunicaciones, personas y procedimientos configurado
   para recolectar, manipular, almacenar y procesar datos para
   convertirlos en información
\end{definition}

Entonces un sistema de información necesita de una \textbf{entrada} de datos, un \textbf{procesamiento} de los mismos y una \textbf{salida} de información.

\section{Calidad}
\begin{definition}[Calidad - 1] 
   La calidad para Pressman (1998) es el \textit{cumplimiento} con:
   \begin{itemize}
      \item los requerimientos funcionales y de rendimiento explícitamente establecidos,
      \item los estándares de desarrollo explícitamente documentados
      \item con las características implícitas que se esperan de todo
      software desarrollado profesionalmente.
   \end{itemize}
\end{definition}

\begin{definition}[Calidad - 2]
   Según las standards ISO e IEEE la calidad es
   El grado con el que un sistema, componente o proceso
   cumple con los requisitos especificados y las necesidades o
   expectativas del cliente o usuario.
\end{definition}

\begin{definition}[Calidad - 3]
   Según la ISO 91260, la calidad es el conjunto de propiedades
La totalidad de características de un producto de software
que tienen como habilidad, satisfacer necesidades explícitas o
implícitas.
\end{definition}

\subsection{Modelos de Calidad}
Los modelos de calidad son herramientas que permiten evaluar la calidad de un producto o servicio. Ellos apuntar a identificar características estándar relacionadas con la
calidad del software a través de atributos de calidad.
{Atributos de calidad incluyen:\ns
\begin{itemize}
   \item Adeguación Funcional
   \item Seguridad
   \item Fiabilidad (reliability)
   \item Usabilidad
   \item Eficiencia
   \item Mantenibilidad
   \begin{itemize}
      \item Reparabilidad
      \item Adaptabilidad
      \item Portabilidad
   \end{itemize}
   En relación con Mantenibilidad, el OPEN/CLOSED principle dice que \ul{un software debe estar abierto para extensión pero cerrado para modificación}. 
   \item Compatibilidad
\end{itemize}
\note{Estos atributos pueden cambiar según los modelos}
}

En general los atributos pueden ser esternos o internos. Los primeros derivados de la relación entre el entorno y el sistema (para ello, el proceso o el sistema debe ejecutarse), e.g. reliability, robustness, usability. Los segundos derivados directamente de la descripción del producto o del proceso.

\subsection{Métricas}
{Es necesario desarollar métricas de calidad, que deben ser:\ns
\begin{itemize}
   \item Simples y faciles da usar
   \item Empírica e intuitivas
   \item Consistente y objectivas
\end{itemize}}

Por ejemplo, centrémonos en la mantenibilidad. Podemos medirla con las siguientes métricas:
\begin{itemize}
   \item Aclopamiento
   \item Cohesión
   \item Complejidad Ciclomática de McCabe
   \item Código Chum
   \item Code Coverage
   \item Código Muerto
   \item Duplicación de Código
   \item 
\end{itemize}


\section{Requisitos}
Los requisitos son fundamentales en el software, y pueden ser utilizados para medir la calidad de este. 
Pero es importante notar que es necesario poder verificar si los requisitos están satisfechos con la implementación.
Además, necesitamos también algunos controles sobre los requisitos, come complejidad, consistencia, completitud, corrección, claridad, verificabilidad, rastreabilidad, prioridad, viabilidad, flexibilidad, no ambigüedad, no redundancia, no contradicción, no vaguedad, no sobre-especificación, no sub-especificación.
% TODO verifica requisitos controles

Requisitos no funcionales son más fáciles de verificar que los funcionales, porque son más objetivos.
\begin{definition}
   [Requisito no funcional verificable]
   Una frase que incluye alguna medida que puede ser objetivamente probada.
\end{definition}

Matrices de trazabilidad son herramientas que permiten verificar la trazabilidad de los requisitos.
Hay muchas matrices en las slides, algunas relacionan requisitos con casos de uso, otros con pruebas, otros con componentes del sistema.

\framedt{Ejercicio 1 / Analisis Requisitos}{
   Javier dice cosas correctas. Aparte de las cosas que mencionó, se puede ver cómo no hay muchos numeros en los requisitos; el documento dices que es necesario hacer backups, pero, ¿cuantós backups? ¿con qué frecuencia? ¿con qué software?.
}

\section{Gestión de la Calidad}
La calidad del proceso contribuye a la calidad del producto, y la calidad del producto contribuye a la calidad en uso.

La ISO 9001 es una norma internacional que especifica los requisitos para un sistema de gestión de la calidad (SGC). Las organizaciones utilizan la norma para demostrar la capacidad para proporcionar productos y servicios que cumplen con los requisitos del cliente y los reglamentarios aplicables.

Propone también un metodo de \textbf{mejora continua}, el PDCA (Plan, Do, Check, Act).


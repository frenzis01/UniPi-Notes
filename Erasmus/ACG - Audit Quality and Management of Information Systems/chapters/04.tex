\chapter{Practica 1}
\section{Practica}

Soy Francesco, un estudiante italiano en Erasmus. Yo he leido el documento de esta práctica. Entiendo bastante bien el español, pero no lo hablo fluidamente.\\
No soy capaz de hacer una presentación oral clara, eficaz y persuasiva.\\
Lo siento especialmente porque he participado en algunas conferencias como orador en el pasado, sé cómo hablar a un audiencia, sé que si hay profesionales del sector o estudiantes universitarios son cosas diferentes, que tienes que mostrar datos en manera eficaz, y todas las demás cosas escritas en el documento.\\
No conozco suficientemente bien el español para comunicarme de manera eficaz.\\
\ul{Sin duda puedo \textbf{escuchar} atentamente a mis compañeros}, y si lo quieres puedo probar a decir algo, quizás sobre los primeros puntos, más introductorios, menos técnicos. Pero no sé\dots
\nl
\nl


Ante todo,\\
gracias por vuestro tiempo y por estar aquí a escuchar la \textit{"oveja negra de la familia"},

yo he estudiado en Pisa, que es más pequeña que valencia, pero tiene también una buena universidad. 
Mi padre quería que yo estudiara ingeniería mecánica, no informática, pero allí se hace \textbf{demasiada} matematica y fisica,\\
que Me gustaban en el instituto, pero no hasta el punto de ser mi pasión, mi vida de estudiarlas por 3, o 4 o 5 años;\\
En cambio, había estudiado allí un poquito de informática y quería profundizar en el tema, tenía curiosidad.\\
Pensaba que lo tenía fácil al principio debido a mis conocimientos previos, pero después de sólo la primera semana, los temas que conocía ya se habían acabado!\\
Así que inmediatamente me sentí estimulado a aprender cosas nuevas. Claro que no todo lo que brilla es oro, en su mayor parte eran cosas muy interesantes, per había también algunas divagaciones aburridas sobre viejas tecnologías que ya no se utilizan o teorías demasiado teóricas difíciles de aplicar en escenarios reales, o repeticiones redundantes de conceptos ya vistos.\\
Pero creo que sea bastante asì en todas universidades.\\
Y por cierto, había muchos contenidos sobre tecnologías y técnicas modernas que se utilizan en la realidad, como Agile, Docker, Python, Microservices, Kafka, así como diversos fundamentos más teóricos que ayudan a aprender y estudiar nuevos conceptos;\\
que creo que sea un requisito fundamental para trabajar, porque las tecnologías, las herramientas cambian con el tiempo, por lo que es necesario saber adaptarse en un contexto de trabajo.



El mensaje tiene que ser claro, no debe incluir experiencias personales.

``Yo preguntaría a algunos de los chicos del público qué esperan encontrar en una carrera en TI, como contenido o como resultados.''
Tienes que ser una pregunta que atrae más atención.

Riguardo la immagine, bisogna renderla memorabile. Ad esempio, invece di mostrare un istogramma con i salari, si potrebbe mostrare cosa ti puoi comprare con il salario più alto, e cosa con quello più basso, ad esempio una tesla o un pandino.
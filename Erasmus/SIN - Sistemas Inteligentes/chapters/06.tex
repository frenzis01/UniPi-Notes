\chapter{Ontologías y Comunicación}
$\circ\circ$


Una ontología define los términos y conceptos comunes
empleados para describir y representar un área de
conocimiento   

Descripción mediante
\begin{itemize}
	\item Clases
	\item Instancias
	\item Relaciones
	\item Propiedades
	\item Funciones / procesos
	\item Restricciones
\end{itemize}
Representación
\begin{itemize}
	\item Frases que combinan la
terminología para expresar
relaciones entre los términos
	\item Estas frases aportan
significado
\end{itemize}

\section{Ontologías y OWL}

Las ontologías permiten representar conocimiento de forma estructurada y compartible entre sistemas. Para construir modelos ontológicos se requieren cuatro niveles:

\begin{itemize}
    \item \textbf{Sintaxis}: Define el orden, formato y estructura (elementos sintácticos como XML)
    \item \textbf{Estructura}: Organización de elementos en jerarquías, herencia y relaciones parte-de
    \item \textbf{Semántica}: Mapeo entre datos estructurados y modelos de objetos que aportan significado
    \item \textbf{Uso}: Pragmática que indica cómo utilizar la semántica
\end{itemize}

\subsection{Lenguajes para Representación de Ontologías}

\subsubsection{RDF (Resource Definition Framework)}
RDF proporciona un marco estandarizado para representar conocimiento en la web mediante tripletas:
\begin{itemize}
    \item \textbf{Recurso} (Sujeto): Entidad de la que se habla
    \item \textbf{Propiedad} (Predicado): Define relaciones o características
    \item \textbf{Objeto}: Entidad a la que se refiere el predicado
\end{itemize}

\subsubsection{RDFS (RDF Schema)}
Extiende RDF permitiendo definir:
\begin{itemize}
    \item Jerarquía de clases e instancias
    \item Restricciones sobre propiedades
    \item Jerarquía de propiedades
\end{itemize}

\subsubsection{OWL (Web Ontology Language)}
OWL aporta mayor expresividad, permitiendo definir:
\begin{itemize}
    \item Clases como combinaciones booleanas (union, intersection, complement)
    \item Clases disjuntas
    \item Equivalencia entre clases o individuos
    \item Cardinalidad en propiedades
    \item Propiedades transitivas, inversas, funcionales
\end{itemize}

\subsection{Protégé}
Herramienta gratuita y open source para:
\begin{itemize}
    \item Editar ontologías
    \item Gestionar bases de conocimiento
    \item Definir estructuras ontológicas
    \item Administrar instancias
\end{itemize}

\section{Comunicación entre Agentes}

La comunicación es fundamental para los sistemas multiagente, permitiendo la cooperación, coordinación y negociación entre agentes.

\subsection{Teoría de Actos de Habla}

Los actos de habla consideran el lenguaje como acción, donde cada declaración realiza un acto:
\begin{itemize}
    \item \textbf{Actos asertivos}: Dan información sobre el mundo
    \item \textbf{Actos directivos}: Solicitan algo al destinatario
    \item \textbf{Actos de promesa}: Comprometen al locutor a acciones futuras
    \item \textbf{Actos expresivos}: Indican estados mentales
    \item \textbf{Actos declarativos}: La declaración realiza el acto
\end{itemize}

Componentes de los actos de habla:
\begin{itemize}
    \item \textbf{Locución}: Modo de producción de frases
    \item \textbf{Ilocución}: Acto realizado (fuerza ilocutoria + contenido proposicional)
    \item \textbf{Perlocución}: Efectos en el destinatario
\end{itemize}

\subsection{Lenguajes de Comunicación entre Agentes}

\subsubsection{KQML (Knowledge Query Management Language)}
Lenguaje basado en actos de habla con:
\begin{itemize}
    \item Performativas (tell, ask, achieve, reply...)
    \item Estructura de mensaje con pares atributo-valor
    \item Niveles de contenido, comunicación y mensaje
\end{itemize}

\subsubsection{FIPA ACL}
Estándar desarrollado por Foundation for Intelligent Physical Agents:
\begin{itemize}
    \item Define estructura de mensajes y performativas
    \item Estandariza arquitectura de plataformas de agentes
    \item Incluye servicios de directorio, gestión y comunicación
\end{itemize}

\subsection{Comunicación en JASON}
JASON implementa comunicación mediante:
\begin{itemize}
    \item Estructura de mensajes: $\langle$emisor, fuerza\_ilocutoria, contenido$\rangle$
    \item Acción interna: \texttt{.send(receptor, performativa, contenido)}
    \item Performativas: tell, untell, achieve, askOne, askAll...
\end{itemize}

\section{Sistemas Multiagente}

Los sistemas multiagente consisten en conjuntos de agentes que interactúan entre sí:
\begin{itemize}
    \item Capacidad para coordinar, cooperar y negociar
    \item Requieren servicios de localización (páginas blancas y amarillas)
    \item Utilizan protocolos de interacción estandarizados
\end{itemize}

\subsection{Plataformas de Agentes FIPA}
El modelo conceptual FIPA incluye:
\begin{itemize}
    \item Agent Management System (AMS): gestión de agentes
    \item Directory Facilitator (DF): servicio de páginas amarillas
    \item Agent Communication Channel (ACC): canal de comunicación
    \item Message Transport System: infraestructura de comunicación
\end{itemize}
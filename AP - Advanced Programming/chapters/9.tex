\chapter{Generics}
\textbf{Generics} are instance of \textit{Universal Polymorphism} with explicit parameters (see Fig \ref{fig:polymorphism_classification}).

\section{Methods}
\begin{lstlisting}
   public static <T> T getFirst(List<T> list)
\end{lstlisting}
Invocations of generic methods must instantiate all
type parameters, either explicitly or implicitly.
Some sort of \textit{type inference} is applied in case of implicit instantiation.

\begin{lstlisting}
class NumList<E extends Number> {
      void m(E arg) {
            arg.intValue();   // OK, since...
            // Number and its subtypes support intValue()
         }
   }
\end{lstlisting}

Type parameters can also be \textbf{bounded} as in the above example,
allowing methods (and fields) defined in the \textbf{bound} to be invoked on objects of the type parameter \lstinline|T|.\\
There may be various kinds of type bounds:
\begin{lstlisting}
<TypeVar extends SuperType>
   // UPPER bound; SuperType and any of its subtype are ok.
<TypeVar extends ClassA & InterfaceB & InterfaceC & ...>
   // MULTIPLE UPPER bounds
<TypeVar super SubType>
   // LOWER bound; SubType and any of its supertype are ok
\end{lstlisting}

Unlike C++ where \textit{overloading} is resolved and can \textbf{fail} after
instantiating a template, in Java \textbf{type checking} ensures that
overloading will succeed.

\section{Inheritance and Arrays}
There are two major issues which came up along with generics.
The first one regards \textbf{inheritance}; 
consider the following example:
\begin{center}
   Since \lstinline|Integer| is a \textit{subtype} of \lstinline|Number|,\\
   is \lstinline|List<Integer>| \textit{subtype} of \lstinline|List<Number>|?\nl

   {\color{red}\textbf{\textit{NO!}}}
\end{center}
In a formal way, \textit{subtyping is invariant} for Generic classes.
Informally, given \lstinline|A,B| concrete types, \lstinline|MyClass<A>| has no relationship to \lstinline|MyClass<B>|,
even if \lstinline|A,B| have one.\\
On the other hand if \lstinline|A extends B| and are \textit{generic} classes, 
then \lstinline|A<C> extends B<C>| for any type \lstinline|C|.
For example, \lstinline|ArrayList<Integer> extends List<Integer>|.
\note{
   Note that the common parent of \lstinline|MyClass<B>| and \lstinline|MyClass<A>| is \lstinline|MyClass<?>|.
}
\nl

Let's now discuss \textbf{covariance} and \textbf{contravariance}, with the aid of a few examples.

\begin{lstlisting}
List<Integer> lisInt = new ...;
List<Number> lisNum = new ...;
lisNum = lisInt; // ??? - Reassign pointer
lisNum.add(new Number(...)); // NOT ALLOWED
listInt = lisNum; // ??? - Reassign pointer
Integer n = lisInt.get(0); // NOT ALLOWED
\end{lstlisting}

\lstinline|List<Integer>| is neither a subtype or a supertype of \lstinline|List<Number>|,
thus the above operations aren't allowed.
However there are \textit{read-only} and \textit{write-only} situations where they may be allowed.

\begin{lstlisting}
RO_List<Integer> lisInt = new ...;
RO_List<Number> lisNum = new ...;
lisNum = lisInt; // ???
Number n = lisNum.get(0); // OK
\end{lstlisting}
It is ok to \textit{read} a \textbf{supertype} starting from a \textbf{subtype}.
\begin{center}
   \textit{\textbf{covariance} is safe if the type is \textbf{read-only}}
\end{center}

\begin{lstlisting}
WO_List<Integer> lisInt = new ...;
WO_List<Number> lisNum = new ...;
lisInt = lisNum; // ???
lisInt.add(new Integer(...)); // OK
\end{lstlisting}
It is ok to \textit{write} a \textbf{subtype} in the place of from a \textbf{supertype}.
\begin{center}
   \textit{\textbf{contravariance} is safe if the type is \textbf{write-only}}
\end{center}
\section*{17 - Ottobre}
\subsubsection*{Other languages}
In the case of \textbf{C\#}, generic classes can be marked with the keyword \lstinline|out| (\textit{covariant}) or \lstinline|in| (\textit{contravariant}),
otherwise the class is invariant.
In \textbf{Scala} the same happens,
but with the \lstinline|+| or \lstinline|-| operators.\nl

Let's now discuss \textbf{arrays}.\\
Let \lstinline|A extends B|, then \lstinline|A[] extends B[]| even if instead \lstinline|Array<A>| is not related to \lstinline|Array<B>|.
\begin{center}
   Thus, \textit{arrays in Java are \textbf{covariant}.}
\end{center}

However there is a counterpart, since this allows rule-breaking assignments
which are allowed by the compiler but which lead to a runtime \lstinline|ArrayStoreException|.
This happens because the dynamic type of an array is checked at runtime.
Knowing this, for each array update, a runtime check is performed by the JVM which throws the exception if needed.
\begin{lstlisting}
Apple[] apples = new Apple[1];
Fruit[] fruits = apples;      // Ok, covariance
fruits[0] = new Strawberry(); // Compiles!
// Throws ArrayStoreException at runtime
\end{lstlisting}

After compilation Generic are all \textbf{type-erasured} to \lstinline|Object| or to their first \textit{bound}, if present.
This choice has been made mainly for compatibility with legacy code, leading all instances of the same generic type to have the same type at runtime; 
i.e.
\begin{lstlisting}
   List<String> lst1 = new ArrayList<String>();
   List<Integer> lst2 = new ArrayList<Integer>();
   assert(lst1.getClass() == lst2.getClass())   
\end{lstlisting}

\subsection{Generic Arrays}
What about \textit{arrays of generics}?
Such arrays in Java are \textbf{not allowed},
because every array update needs a runtime check which is impossible to perform on generics,
since at runtime generics are all of the same type due to \textit{type-erasure}.

\section{Wildcards}
\textbf{Wildcards} are strongly related to the topic of \textit{covariance} and \textit{contravariance}.\\
As briefly mentioned before, wildcards are the only relationship between generic classes.

To use \textit{wildcards}, the \textbf{PECS} principle is applied:
\textit{\textbf{P}roducer \textbf{E}xtends, \textbf{C}onsumer \textbf{S}uper}.
\begin{itemize}
   \item \lstinline|? extends T| to \textbf{get} values from a \textit{Producer}: \textbf{covariance} allowed
   \item \lstinline|? super T| to \textbf{insert} values into a \textit{Consumer}: \textbf{contravariance} allowed
   \item Never use \lstinline|?| when both insertion and retrieving is needed, \lstinline|T| is sufficient and way more appropriate.
\end{itemize}

Wildcards improve type-safety, allowing a program to fail at \textit{compile-time} instead of \textit{runtime}.
\begin{lstlisting}
List<Apple> apples = new ArrayList<Apple>();
List<? extends Fruit> fruits = apples;
fruits.add(new Strawberry()); // COMPILING FAILS
\end{lstlisting}

\section{Generics Limitations}
\begin{itemize}
   \item Cannot instantiate Generics with primitive types:
   \begin{lstlisting}
      ArrayList<int> a = ...                       // compile error
   \end{lstlisting}
   \item Cannot create instances of type parameters
   \item Cannot declare static fields whose types are type parameters
   \begin{lstlisting}
      public class C<T>{ public static T local; ...}
   \end{lstlisting}
   \note{Because static fields are represented in the \textbf{unique} representation of the class in the dedicated static memory area of the JVM for classes}
   \item Cannot use casts or instanceof with parameterized types
   \begin{lstlisting}
      mylist instanceof ArrayList<Integer>         // fails
      mylist instanceof ArrayList<?>               // OK
   \end{lstlisting}
   \item Cannot Create arrays of parameterized types
   \item Cannot create, catch, or throw objects of parameterized types
   \item Cannot overload a method where the formal parameter types of each overload erase to the same raw type.
   \begin{lstlisting}
      public class Example {                       // does not compile
         public void print(Set<String> strSet) { }
         public void print(Set<Integer> intSet) { } }
   \end{lstlisting}
\end{itemize}


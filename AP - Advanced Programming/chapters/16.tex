\chapter{Streams}

Streams were introduced to support functional-style operations on streams of elements, such as map-reduce transformations on collections.\\
Let's consider the properties of \textbf{Streams}, and we'll clearly see how they are different from Collections.
\begin{itemize}
   \item \textbf{No storage}\\
   A stream is \textit{not} a data structure that
   stores elements; instead, it conveys elements
   from a \textit{source} \footnote{e.g. a data structure, an array, a
   generator function, an I/O channel,...} through a pipeline of computational operations.
   \item \textbf{Functional} in nature\\
   An operation on a stream produces a \textit{result}, but does \textit{not} modify its source.
   \note{In fact ---if I recall correctly--- modifying a stream while iterating over it with \lstinline|forEach| would result in messed up behavior; more precisely, a \lstinline|ConcurrentModificationException|}
   \item \textbf{Laziness-seeking}
   Many stream operations can be
   implemented lazily, exposing opportunities for optimization.
   Stream operations are divided into \textbf{intermediate} 
   (\textit{stream-producing}) operations {---}which are \textit{always lazy}{---} and
   \textbf{terminal}
   (\textit{value- }or \textit{side-effect-producing}) operations.\\
   Intermediate operations such as \lstinline|map()| or \lstinline|filter()| are not provessed immediately, they instead create a \textit{pipeline} of operations holding the transformation logic that is only executed when a terminal operation (\lstinline|collect()|, \lstinline|forEach()|) is invoked.
   \item Possibly \textbf{unbounded}\\
   While collections have a \textit{finite size}, streams need \textit{not}.
   Short-circuiting operations such as \lstinline|limit(n)| or
   \lstinline|findFirst()| can allow computations on \textit{infinite streams} to
   complete in \textit{finite time}.
   \item \textbf{Consumable}\\
   The elements of a stream are only \textit{visited once}
   during the life of a stream. 
   Like an Iterator, a new stream must be generated to \textit{revisit} the same elements of the source.\\
   The Stream is considered \textit{consumed} when a \textit{terminal}
   operation is invoked. 
   No other operation can be performed on
   the Stream elements afterwards.
\end{itemize} 
\section{Pipelines}

{A typical pipeline contains\ns
\begin{enumerate}
   \item A \textbf{source}, producing (by need) the elements of the stream
   \item Zero or more \textbf{intermediate} operations, producing streams
   \item A \textbf{terminal} operation, producing side-effects or non-stream values
\end{enumerate}}
\note{
   Example of typical pattern: \texttt{filter / map / reduce}
} 

\begin{lstlisting}
   double average = listing // collection of Person
      .stream() // stream wrapper over a collection
      .filter(p -> p.getGender() == Person.Sex.MALE) // filter
      .mapToInt(Person::getAge) // extracts stream of ages
      .average() // computes average (reduce/fold)
      .getAsDouble(); // extracts result from OptionalDouble
\end{lstlisting}

\subsection{Sources}
Common sources are \lstinline|Collection|s via \lstinline|stream()| and \lstinline|parallelStream()| methods,
but there are several and various other sources,
like \lstinline|IntStream.range(int, int)|, \lstinline|Stream.iterate(Object, UnaryOperator)|, \lstinline|BufferedReader.lines()|, \lstinline|Random.ints()|, and many others.

\subsection{Intermediate operations}
An intermediate operation keeps a stream \textit{open} for further operations.
Intermediate operations are lazy, and 
several of them have arguments of \textit{functional interfaces},
thus \textbf{lambdas} can be used.\\
Exampls are \lstinline|map(), peek(), distinct(), sorted()|

\subsection{Terminal operations}
A terminal operation must be the \textit{final operation} on a stream.
Once a terminal operation is invoked, the stream is consumed and is no longer usable.
As said before, the typical approach is to collect values in a data structure, reduce to a value, and lastly print or
cause other side effects.\\
Examples are \lstinline|reduce(),forEach(),allMatch()| and others.
\note{\lstinline|reduce()| is basically our well-known \lstinline|fold|}

\section{Mutable Reduction}

Suppose we want to concatenate a stream of strings:
\begin{lstlisting}
   String concatenated = listOfStrings
      .stream()
      .reduce("", String::concat)
\end{lstlisting}
The above works, but is highly inefficient: it builds one new string for each element, since \lstinline|Strings| are immutable in Java.\\
It would be better to "accumulate" the elements in a mutable object (e.g. a \lstinline|StringBuilder|, a \lstinline|collection|, ...).
In our aid comes the \textbf{mutable reduction operation} which is called \lstinline|collect()|, which requires
{three functions:\ns
\begin{enumerate}
   \item a \textbf{supplier} function to \textit{construct} new instances of the result container,
   \item an \textbf{accumulator} function to \textit{incorporate} an input element into a result container,
   \item a \textbf{combining} function to \textit{merge} the contents of one result container into another.
\end{enumerate}}
\begin{lstlisting}
   <R> R collect( Supplier<R> supplier,
                  BiConsumer<R, ? super T> accumulator,
                  BiConsumer<R, R> combiner);
\end{lstlisting}

\begin{lstlisting}
   // NO streams
   ArrayList<String> strings = new ArrayList<>();
   for (T element : stream) {
      strings.add(element.toString());
   }

   // with streams and lambdas
   ArrayList<String> strings =
   stream.collect(
      () -> new ArrayList<>(),         // Supplier
      (c, e) -> c.add(e.toString()),   // Accumulator
      (c1, c2) -> c1.addAll(c2));      // Combining
   
   // with streams and method references
   ArrayList<String> strings = stream.map(Object::toString)
   .collect(   ArrayList::new,      // Supplier
               ArrayList::add,      // Accumulator
               ArrayList::addAll);  // Combining
\end{lstlisting}

However, \lstinline|collect()| can also be invoked with a \lstinline|Collector| argument,
which encapsulates the functions used as
arguments to collect (\textit{Supplier}, \textit{BiConsumer},
\textit{BiConsumer}),
allowing for reuse of collection strategies and composition of collect operations.

\begin{lstlisting}
   Map<String, List<Person>> peopleByCity =
      personStream.collect(Collectors.groupingBy(Person::getCity));
\end{lstlisting}

\section{Parallelism}
Streams facilitate parallel execution:
stream operations can execute either in serial
(default) or in parallel, with the runtime support transparently taking care of using multithreading
for parallel execution.
If operations \textit{don’t} have side-effects, \textit{thread-safety} is
\textbf{guaranteed} even if \textit{non-thread-safe collections} are
used (e.g. \lstinline|ArrayList|).

Also \textit{concurrent mutable reduction} (collect) is supported for parallel streams, however
Order of processing stream elements depends on
serial/parallel execution and intermediate operations,
and may not be predictable.

\begin{lstlisting}
   double average = persons   // average age of all males
      .parallelStream()       // members in PARALLEL
      .filter(p -> p.getGender() == Person.Sex.MALE)
      .mapToInt(Person::getAge)
      .average()
      .getAsDouble();

      sorted_listOfIntegers.parallelStream()
      .forEach(e -> System.out.print(e + " "));
      // may print: 3 4 1 6 2 5 7 8
\end{lstlisting}

\subsection{Summing up}
One should use Parallelism
\begin{itemize}
   \item When operations are independent, and
   \item Either or both:
   \begin{itemize}
      \item Operations are computationally expensive
      \item Operations are applied to many elements of efficiently splittable data structures
   \end{itemize}
\end{itemize}
\emph{"Always measure before and after \emph{parallelizing}!"}


\subsection{Critical Issues}
\begin{itemize}
   \item \textbf{Non-interference}
   \begin{itemize}
      \item Behavioural parameters (like \textit{lambdas}) of stream
      operations should \textit{not affect} the source (i.e. \textit{non-interfering behaviour})
      \item Risk of \lstinline|ConcurrentModificationExceptions|, even in
      single-threaded execution
   \end{itemize}
   \item \textbf{Stateless behaviours}
   \begin{itemize}
      \item \textit{Stateless} behaviour for intermediate operations is
      \textit{encouraged}, as it facilitates parallelism, and functional
      style, thus maintenance
   \end{itemize}
   \item \textbf{Parallelism and thread safety}
   \begin{itemize}
      \item For parallel streams with \textit{side-effects},
      ensuring thread safety is the programmers’ responsibility
   \end{itemize}
\end{itemize}

\begin{lstlisting}
   String concatenatedString = listOfStrings
   .stream()
   .peek(s -> listOfStrings.add("three")) // DON'T DO THIS!
      // Interference occurs here.
   .reduce((a, b) -> a + " " + b)
   .get();
\end{lstlisting}

\section{Monads in Java}
\begin{lstlisting}
   public static <T> Optional<T> of(T value)
   // Returns an Optional with the specified present non-null value.
   <U> Optional<U> flatMap(Function<? super T,Optional<U>> mapper)
\end{lstlisting}
If a value is present, \lstinline|flatMap| applies the provided \textit{Optional-bearing} mapping
function to it, return that result, otherwise return an empty
\lstinline|Optional|.

\begin{lstlisting}
   static <T> Stream<T> of(T t)
   // Returns a sequential Stream containing a single element.
   <R> Stream<R> flatMap(
   Function<? super T,? extends Stream<? extends R>> mapper)
\end{lstlisting}
Here \lstinline|flatMap| returns a \lstinline|Stream| consisting of the results of replacing each element
of this stream with the contents of a mapped stream produced by applying
the provided mapping function to each element.
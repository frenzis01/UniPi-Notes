\chapter{Python}
Python is very concise and good for scripting.
It is the most used general purpose language, and there are many nice Machine Learning libraries.

Its two key aspects are that it is \textbf{interpreted} ($\neq compiled$) and \textbf{dynamically typed}.

Aside from these two main aspects, we can discuss a few others:\\
even if not as much as Java, it's \textbf{object} oriented;
Python supports both \textbf{imperative} and \textbf{functional} paradigms.\\
Python \textbf{Iterators} can be associated with \texttt{List}s in Haskell and \texttt{Stream}s in Java.
Other Python features are:
\begin{itemize}
   \item Powerful subscripting (slicing)
   \item Higher-order functions
   \item Flexible signatures, which compensate the lack of complex polymorphism
   \item Java-like exceptions
   \item (bad) Multithreading support
\end{itemize}

Let's discuss how Python addresses \textbf{typing}:
\begin{enumerate}
   \item Variables come into existence when first assigned to
   \item Variables are \textit{not} \textit{typed}, but \underline{Values are}!
   \item A variable can refer to an object of any type
   \item It is \textbf{Strongly typed} in the sense that the \textit{value type} does not change in unexpected ways
   \item It is \textbf{Type-safe} since no conversion or operation can be applied to values of the \textbf{wrong} type
\end{enumerate}


Variables come into existence when first assigned and a variable can refer to an object of any type,
besides all types are (almost) treated the same way.
Even if this allows for concise syntax and intuitive code writing, it also implies a main drawback: \underline{\textbf{type errors} are only caught at \textit{runtime}}.

\section{Some Python aspects}
\begin{itemize}
   \item \textbf{Indentation} matters, and is used opposed to brackets \texttt{\{\}}
   \item A variable is created when some value is assigned to it
   \item Assignments in Python \textbf{do not} create a copy
   \note{Even for \texttt{Integer}s!}
   \item Objects are deleted by the garbage collector once they become unreachable
   \begin{itemize}
      \item \texttt{CPython} uses \textit{reference counting} along with \texttt{Mark \& Sweep} for garbage collection
   \end{itemize}
   \item 
   \end{itemize}

\section{Data Types and operators}
\textit{Integers} are \textbf{unbounded},
while \texttt{Float}s are represented with 64 bits.\\
There are no logical symbols, the operators are \texttt{and, or, not}.\\
Strings can be enclosed in \texttt{''}, \texttt{""}, \texttt{""" """},
with the third that allows multiline strings (\textit{cool!}).

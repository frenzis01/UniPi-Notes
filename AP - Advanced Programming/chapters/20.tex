\chapter{Python}
Python is very concise and good for scripting.
It is the most used general purpose language, and there are many nice Machine Learning libraries.

Its two key aspects are that it is \textbf{interpreted} ($\neq compiled$) and \textbf{dynamically typed}.

Aside from these two main aspects, we can discuss a few others:\\
even if not as much as Java, it's \textbf{object} oriented;
Python supports both \textbf{imperative} and \textbf{functional} paradigms.\\
Python \textbf{Iterators} can be associated with \texttt{List}s in Haskell and \texttt{Stream}s in Java.
Other Python features are:
\begin{itemize}
   \item Powerful subscripting (slicing)
   \item Higher-order functions
   \item Flexible signatures, which compensate the lack of complex polymorphism
   \item Java-like exceptions
   \item (bad) Multithreading support
\end{itemize}

Let's discuss how Python addresses \textbf{typing}:
\begin{enumerate}
   \item Variables come into existence when first assigned to
   \item Variables are \textit{not} \textit{typed}, but \underline{Values are}!
   \item A variable can refer to an object of any type
   \item It is \textbf{Strongly typed} in the sense that the \textit{value type} does not change in unexpected ways
   \item It is \textbf{Type-safe} since no conversion or operation can be applied to values of the \textbf{wrong} type
\end{enumerate}


Variables come into existence when first assigned and a variable can refer to an object of any type,
besides all types are (almost) treated the same way.
Even if this allows for concise syntax and intuitive code writing, it also implies a main drawback: \underline{\textbf{type errors} are only caught at \textit{runtime}}.

\section{Some Python aspects}
\begin{itemize}
   \item \textbf{Indentation} matters, and is used opposed to brackets \texttt{\{\}}
   \item A variable is created when some value is assigned to it
   \item Assignments in Python \textbf{do not} create a copy
   \note{Even for \texttt{Integer}s!}
   \item Objects are deleted by the garbage collector once they become unreachable
   \begin{itemize}
      \item \texttt{CPython} uses \textit{reference counting} along with \texttt{Mark \& Sweep} for garbage collection
   \end{itemize}
   \item 
   \end{itemize}

\section{Data Types and operators}
\textit{Integers} are \textbf{unbounded},
while \texttt{Float}s are represented with 64 bits.\\
There are no logical symbols, the operators are \texttt{and, or, not}.\\
Strings can be enclosed in \texttt{''}, \texttt{""}, \texttt{""" """},
with the third that allows multiline strings (\textit{cool!}).

\subsection{Sequence types}
\texttt{Tuples}, \texttt{Strings} and \texttt{Lists} are the sequence types in Python, they are \textit{immutable} except for Lists.

They all support \textbf{slicing}, concatenation (\lstinline|+|) and membership (\lstinline|in|).

\textbf{Slicing} returns a subsequence of the original sequence, a \textbf{copy}.
Start copying
at the first index, and stop copying \textit{before} the second index.

\lstset{language=python}
\begin{lstlisting}
   >>> t = (23, 'abc', 4.56, (2,3), 'def')
   >>> t[1:4] # ('abc', 4.56, (2,3))
   >>> t[1:-1] # negative indices ('abc', 4.56, (2,3))
   >>> t[1:-1:2] # optional argument: step ('abc', (2,3))
   >>> t[:2] # no first index: from beginning (23, 'abc')
   >>> t[2:] # no second index: to end (4.56, (2,3), 'def')
   >>> t[:] # no indexes: creates a copy (23, 'abc', 4.56, (2,3), 'def')
\end{lstlisting}

Lists, since they are \textbf{mutable}, support some specific operators \lstinline|append(),insert(),extend(),index(),count(),remove(),sort()...| 

Lists also support \textbf{List Comprehension} pretty similarly to Haskell, allowing also filtered and nested comprehensions.
\begin{lstlisting}
   >>> li = [3, 2, 4, 1]
   >>> [elem*2 for elem in
      [item+1 for item in li] if elem > 1]
   [8, 6, 10]
\end{lstlisting}

\subsection{Sets and Dictionaries}
\textbf{Sets} support membership, and common set operations, but not indexing:
\begin{lstlisting}
   >>> a = set('abracadabra')
   >>> b = set('alacazam')
   >>> a # unique letters in a
   {'a', 'r', 'b', 'c', 'd'}
   >>> a - b # letters in a but not in b
   {'r', 'd', 'b'}
   >>> a | b # letters in a or b or both
   {'a', 'c', 'r', 'd', 'b', 'm', 'z', 'l'}
   >>> a & b # letters in both a and b
   {'a', 'c'}
   >>> a ^ b # letters in a or b but not both
   {'r', 'd', 'b', 'm', 'z', 'l'}
\end{lstlisting}

\textbf{Dictionaries} are very similar to \lstinline|maps| in Java, and they map a key of an \underline{\textit{immutable hashable}} type,
and support the following operations on the $\langle key,value \rangle$pairs:
\begin{enumerate}
   \item define
   \item modify
   \item view
   \item lookup
   \item delete
\end{enumerate}

\section{Higher-order functions}
Functions can be returned as result and passed as arguments to other functions, for example the predefined \lstinline|map| and \lstinline|filter| combinators.
When handling higher-order functions there's a heavy use of iterators, which support laziness.

Typically List comprehensions can replace the \lstinline|map| and \lstinline|filter| combinators.

\begin{lstlisting}
   map(func, *iterables) --> map object
   Make an iterator that computes the function using
   arguments from each of the iterables. Stops when the
   shortest iterable is exhausted

   >>> map(lambda x:x+1, range(4))  # lazyness: 
   <map object at 0x10195b278>      # returns an iterator
\end{lstlisting}

\subsection{Decorators}
A \textbf{decorator} is any callable Python object that is used
to modify a function, method or class definition.
A decorator is passed the original object being defined
and returns a modified object, which is then bound to the name in the definition.
The basic idea is to \textbf{wrap a function},
and the use cases may vary a lot, 
and may include \textit{measuring execution time}, caching \textit{intermediate results}, and so on.

\begin{lstlisting}
   def do_twice(func):
   def wrapper_do_twice():
      func() # the wrapper calls the
      func() # argument twice
      return wrapper_do_twice
\end{lstlisting}

\section{Namespaces and Scopes}
A \textbf{namespace} is a mapping from \textit{names} to \textit{objects}, typically
implemented as a dictionary.
\begin{enumerate}
   \item builtin names (predefined functions, exceptions,...)
   \begin{enumerate}
      \item Created at intepreter's start-up
   \end{enumerate}
   \item global names of imported modules
   \begin{enumerate}
      \item Created when the module definition is read
      \item Note: names created in interpreter are in module \lstinline|__main__|
   \end{enumerate}
   \item local names of a function invocation
   class names, object names,...
\end{enumerate}
\nl

A \textbf{scope} is a textual region of a Python program where a
namespace is directly accessible, i.e. reference to a name
attempts to find the name in the namespace.

Scopes are \textit{determined statically}, but are \textit{used \underline{dynamically}},
at runtime at least three namespaces are directly accessible, searched in the following order:
\begin{enumerate}
   \item the scope containing the \textbf{local names}
   \item the scopes of any \textit{enclosing functions}, containing both \textbf{non-local} and \textbf{non-global} names
   \item the \textit{next-to-last} scope containing the current \textbf{module’s global names}
   \item the outermost scope is the namespace containing \textbf{built-in names}
\end{enumerate}

Python supports \textbf{closures}: 
even if the scope of the
outer function is reclaimed on return, the non-local variables referred to by the nested function are saved
in its attribute \lstinline|__closure__|

\begin{lstlisting}
   def counter_factory():
      counter = 0
      def counter_increaser():
         nonlocal counter
         counter = counter + 1
         return counter
      return counter_increaser
   >>> f = counter_factory()
   >>> f()
   1
   >>> f()
   2
   >>> f.__closure__
   (<cell at 0x1033ace88: int object at 0x10096dce0>,)
\end{lstlisting}

\subsection{Scope issues}
\begin{lstlisting}
   def test(x):
      print(x)
      for x in range(3):
         print(x)
      print(x)
   >>> test("Hello!")
   Hello!
   0
   1
   2
   2  # 'x' changed!
\end{lstlisting}
\chapter{Component-Based software}
% \section*{2 - Ottobre}
Component software indicates \textbf{composite systems} made of \textbf{software components}.
In short, component software allows reuse, improving reliability\footnote{Industries may even require to use \textit{certified} components} and reducing costs.\\
Bertrand Meyer suggests some guidelines regarding Object-Oriented software construction (1997):
\begin{enumerate}[noitemsep]
    \item Modular
    \item Reliable
    \item Efficient
    \item Portable
    \item Timely
\end{enumerate}

\section{Definitions}
\begin{definition}
    [Software Component]
    A \textbf{software component} is a unit of composition with
    contractually specified \textbf{interfaces} and explicit context
    dependencies only. A software component can be
    deployed independently and is subject to
    composition by third parties.
\end{definition}

Let's break down this definition, and explain the key concepts it mentions.


A \textbf{contract} is a specification attached to an \textbf{interface} (component specification) that
mutually binds the clients and providers of the components.\\
However, contracts in general specify more than dependencies and interfaces, they also specify how the component can be deployed, instatiated, and how its instances behave, ultimately summing up to more than a simple set of per-interface methods.

\textbf{Context dependencies} are specification of the deployment
environment and run-time environment.
This goes beyond the simple interfaces required and provided which are specified in the contract.
Context dependencies include required tools, platform and resources.
\note{e.g. Database, libraries, frameworks, config files, environment variables, etc.}

\textbf{Deployed independently} means that a component can be plugged or unplugged from an architecture, even at runtime in some cases.
It is common-practice to deploy ``small components'' called connectors,
to resolve situations where two components supposed to interact do not provide identical interfaces,
creating the need for a intermediary.
\note{Even though the component may be packaged and deployed independently (e.g. as JAR files), it still needs the context dependencies to be satisfied.}

\textbf{Composition by third party} means that a component may interact with third parties components without knowing the internals of such components.

\section{Concepts of Component Model}
\begin{itemize}
    \item \textbf{Component interface} - describes the operations implemented and exposed by the component
    \item \textbf{Composition mechanism} - How components can be composed to work together to accomplish a task
    \item \textbf{Component platform} - A platfom for the development of the components
\end{itemize}
Concepts should be language/paradigm agnostic, laying the ground for language interoperability.
\nl

The ancestors of Components are \textbf{Modules}, whose support has been introduced in Java 9, but aren't very common.
Some concepts related to modules can be found in more modern notions such as classes, components and packages.\\
For example, objects inside a module are visible to each other, but not visible from outside unless exported.
Modules worked ---pretty much like classes--- as abstraction mechanism $\rightarrow$ \textit{collection of data with operations defined on them}.
In OOP the concept of \textbf{inheritance}, unknown in modules, is introduced.
\nl


\section{Components and Programming Concepts}
Components can be anything and can contain anything, they can be \textit{collections of} classes, objects, functions/algorithms, data structures.\\
{They support:\ns
\begin{itemize}
	\item \textbf{Unification} of data and function
	\item \textbf{Encapsulation}: no visible state
	\item \textbf{Identity}: each software entity has a unique identity
	\item Use of interfaces to represent specification dependencies
\end{itemize}}

\subsection{OOP vs COP Comparison}
\begin{center}
\textit{¡Note that \textbf{OOP} $\neq$ \textbf{COP}!}
\end{center}

\ul{OOP \textit{isn't} focused on \textbf{reuse}, instead its focus is onto appropriate domain and problem representation}. Objects, classes, inheritance and polymorphism are concepts/tools typically exploited in OOP.\\
COP (more precisely CBSE\footnote{Component-based software engineering}), on the other hand, is focused on \textbf{reuse} and \textbf{composition} of software components. It is a way to build software systems by assembling prefabricated components.
In general it provides methods and tools for:
\begin{itemize}
	\item Building systems from components
	\item Building components as reusable units
	\item Performing maintenance by replacement of
	      components and introducing new components
	      into the system
	\item System architecture detailed in terms of
	      components
\end{itemize}

\subsection{Component forms}
\begin{itemize}
    \item \textbf{Component Specification} describes the behavior (as a set of \textit{Interfaces}) of a set of Component Objects and defines a unit of implementation.
    \item \textbf{Component Implementation} is a realization of \textit{Component Specification} which can be independently deployed\footnote{It does \textbf{not} mean that it cannot have \textit{dependencies} nor that it must be a \textit{single file}}.
    \item \textbf{Installed Component} is an installed (i.e. \textit{deployed}) copy of a \textit{Component Implementation}.
    \note{A Component Implementation is deployed by registering it with the runtime environment}
    \item \textbf{Component Object} is an instance of a \textit{Installed Component}.
    It is a runtime concept, an object with its own data and unique identity.
    Ideally, it is the "thing that performs the implemented behaviour".
    An \textit{Installed Component} may have multiple \textit{Component Objects}.
\end{itemize}

Some examples of successful components are Plugin architectures, Microsoft's Visual Basic, Operating Systems, Java Beans, and others.
It is clear that components can be purchased by independent providers and deployed by the clients, and that multiple components can coexist in the same installation.
Besides, components exist on a level on abstraction where they directly mean something to the deploying client.
\nl

Recalling the comparison with modules,
while modules are usually seen as part of a program, \textit{components are parts of a system}.


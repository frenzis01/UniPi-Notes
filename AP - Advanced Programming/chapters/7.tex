\chapter{Annotations}
\section*{9 - Ottobre}
In java, \lstinline{static,private,...} modifiers are \textit{meta-data} describing properties of program elements.
\textbf{Annotations} can be understood as (user-) definable modifiers.
They are composed by one or two parts:
\begin{enumerate}
    \item \textbf{name}
    \item finite number of \textbf{attributes} i.e. \lstinline{name=value}.
    There may be no attributes.
\end{enumerate}
The syntax is the following:
\begin{lstlisting}
    @annName            // e.g. Override
    @annName{constExp}  // shorthand for @annName{value=constExp}
    @annName{name_1 = constExp_1, ..., name_k = constExp_k}
\end{lstlisting}
\lstinline{constExp} are expression which can be evaluated at \textit{compile time}.
Besides, attributes have a \textit{type}, thus the supplied values have to
convertible to that type.\\
Annotations can be applied to almost any syntactic element, from packages to parameters and any type use.

\section{Defining annotations}
\begin{lstlisting}
@interface InfoCode {
    String author ();
    String date ();
    int ver () default 1;
    int rev () default 0;
    String [] changes () default {};
}
\end{lstlisting}

This defines the custom annotation InfoCode, imposing some fields possibly with default values.
It can be used as follows:
\begin{lstlisting}
@InfoCode(author="Beppe", date="10/12/07")
    public class C {
    public static void m1() { /* ... */ }
    @InfoCode(author="Gianni",
        date="4/8/08", ver=1, rev=2)
    public static void m2() { /* ... */ }
}
\end{lstlisting}
\chapter{Java Memory Model}
A \textbf{memory model} for \textit{multithreaded} systems specifies how mem actions in a program will appear to execute to the programmer, i.e. {---}more specifically{---} which value each read of a memory location may return.\\
Every hardware and software interface of a system that admits multithreaded access to shared memory \textbf{requires} a memory model.
such model determines the transformations that the system can apply to a program

In the case of high-level programming languages such as Java the memory model determines
\begin{enumerate}
   \item the transformations the compiler may apply to a program when \textbf{producing bytecode}
   \item the transformations a VIrtual-Machine may apply to bytecode when \textbf{producing native code}
   \item the \textbf{optimizations} that hardware may perform on the native
\end{enumerate}
Besides, the model also impacts the programmer,
since such transformations determine the possible outcomes of a program.\\
Without a well defined memory model for a programming language, it is impossible to known what the legal results are for a program in such language.
\note{When we programming \textit{"correctly"} in Java, using \texttt{volatile} keywords and related constructs, we can however {---}in some sense{---} ignore the memory model}

\paragraph{Memory Hierarchy}
In modern architectures memory is stratified ranging from mass memory (hard disks) to \texttt{CPU} registers, passing through different cache levels (\texttt{L1,L2,L3}),
obtaining a \textbf{memory hierarchy};
depending on \texttt{CPU} architectures, cache levels may be shared or not among cores.

\section{Java Memory Model}
For incorretly synchronized programs, the behaviour is \textit{bounded} by a well-defined notion of \textbf{causality},
so the semantics are \textit{not} completely undefined as they were in the early (pre \texttt{Java 5} - $2004$) versions of the memory model.\\
The causality constraints are \textit{strong enough} to \textit{respect} the \textbf{safety} and \textbf{security} \textbf{properties} of java,
and \textit{weak enough} to \textit{allow} standard compiler and hardware \textbf{optimizations}.

\subsection{Runtime Data Areas}
\textbf{Local} {-}\textit{primitive} type{-} \textbf{variables} of methods are allocated on thread stacks,
and \textit{cannot} be accessed by other threads;
\textit{Objects} are instead allocated on the Heap. 
\nl

For what concerns the distribution of data in Java,
it may be spread orthogonally around the memory hierarchy, 
i.e. anything can go anywhere.\\
This leads to two key issues:
\begin{enumerate}
   \item \textbf{Visibility} of variable updates
   \item 
\end{enumerate}

(...)

\section{\texttt{volatile} modifier}
\lstinline|volatile| is a modifier that can only be applied to fields of a class,
and intuitively it declares that a field can be modified by multiple threads\footnote{Clearly incompatible with \lstinline|final|}.
The JMM guarantees that the write of a volatile variable is visible when it is read.
An \textit{implementation} should guarantee that the new value is flushed from the cache to the RAM, if a read happens "after" a write.

\begin{center}
   What does it mean that \textit{"a read happens after a write"}?
\end{center}
This will be discussed later on in Sec \ref{sec:behaviour_actions}, however,
When threads do \textit{not} use any synchronization mechanism,
their behavior is described as the sequence of performed read/write \textbf{actions},
along with the results of the read operations;
such sequence has a \textbf{partial ordering} whose \textit{legitimacy} is checked by the JMM
which aims to ensure that a read results in the actual last value written in this partial ordering.
\nl

Notice that \lstinline|volatile| doesn't solve \textbf{Data Races}, which need synchronization mechanism;
the typical example is incrementing a shared counter, which actually consists of three operations, \texttt{read}, increment the value read, and \texttt{write} it;
such actions are \textbf{not} performed \textit{atomically}\footnote{So... which operations are atomic and which are not?},
thus a second thread may read the "old" before the first one writes the updated one, 
resulting in only one incrementation instead of two.

\subsubsection{Monitors}
Monitors are the default Java synchronization mechanisms.
Every object has a monitor exposing a lock which can be held only by one thread at a time.
methods and ... with the \lstinline|synchronized| modifier are guarded by the lock.

\section{Describing thread behaviour}
\label{sec:behaviour_actions}
The JMM has no explicit global ordering of all actions by time consistent with each thread's perception of time, and has no global store.\\
Executions are instead described in terms of memory \textbf{actions},
\textbf{partial orders} on these actions, and a \textbf{visibility function} that assigns a write action to each read action.

\labelitemize{
   \textit{Actions}
}{
   \begin{enumerate}
      \item Volatile read
      \item Volatile write
      \item lock
      \item TODO
   \end{enumerate}
   }
\nl

An execution of a \textit{single-threaded} program fixes a total order $\leq_{po}$ on its \textit{actions}, called \textbf{program order};
while for a \textit{multi-threaded} the program order consists in the union the program order of its threads,
so it does not relate actions of different threads.\\
An execution of a \textit{multi-threaded} program is \textbf{sequentially consistent} if there is a total order of its actions consistent with the program order {---}and such that each read has the value of the last write{---}.\\
For \textit{datarace-free}\footnote{Also called \textit{\underline{corretly}} or \textit{\underline{well} synchronized programs}} mt-programs, the JMM guarantees that only \textbf{sequential consistent} executions are legal.

JMM has been designed to guarantee three things:
\begin{enumerate}
   \item Promise for programmers\\
   Sequential consistency must be sacrificed to allow optimizations, 
   but it still holds for datarace-free program
   \item Promise for security\\
   Values should not appear "out of thin air", allowing for information leakage
   \item Promise for compilers\\
   HW and SW optimizations should be applied without violating (both) the first two requirements
\end{enumerate}

TODO

\begin{paracol}{2}
   \begin{lstlisting}
      int r1
   \end{lstlisting}
\end{paracol}
Instr reordering may be performed as long as it guarantees sequential consistency in the \textit{single thread}

TODO

\begin{paracol}{2}
   \begin{lstlisting}[caption={Thread 1}]
      r1 = x;
      y = r1;
   \end{lstlisting}
\switchcolumn
   \begin{lstlisting}[caption={Thread 2}]
      r2 = y;
      x = r2;
   \end{lstlisting}
\end{paracol}
\lstinline|x = y = 0| initially;
can we obtain \lstinline|r1 == r2 == 42| at the end?\\
\textit{"Well \textbf{no}, but actually \textbf{yes}..."}\\
In some situations the Runtime environment may \textit{guess} that, 
at some point, \lstinline|x| evaluates to \lstinline|42|:
we say that \lstinline|42| comes \textit{"out-of-thin-air"}.
Then it checks by looking at the two thread instructions if it may happen that \lstinline|r1 == r2 == 42|:
\begin{center}
   "Yes! So x actually really evaluates to 42! I guessed right! \smiley"
\end{center}
This was an accepted guess before the JMM introduced with Java 5 (2004),
but currently such claims at runtime are forbidden.

\subsection{Synchronization order}
Each execution of a program is associated with a \textbf{synchronization order} $\leq_{so}$ which is a total order over all synchronization actions satisfying:
\begin{enumerate}
   \item Consistency with program order
   \item Read to a volatile variable $v$ returns the value of the write to $v$ that is ordered last before the read by the synchronization order.
\end{enumerate}
TODO

\paragraph{Formally Defining Data Races}
Two accesses $x$ and $y$ form a data race in an execution of a program if they are from different threads, they conflict, and they are not ordered by happens-before.

A program is said to be correctly synchronized or datarace-free if and only if all sequentially consistent executions of the program are free of data races.

The first requirement for the JMM is to ensure sequential consistency for correctly synchronized or datarace free programs\\
Programmers should not worry about code transformations for datarace-free programs.
TODO

\labelitemize{\textit{Executions - Formal Definition}}
{
   $E = (P,A,\leq_{po},\leq_{so},W,V,\leq_{sw},\leq_{hb})$
   \begin{enumerate}
   \item TODO
   TODO
\end{enumerate}}

$E = (P,A,\leq_{po},\leq_{so},W,V,\leq_{sw},\leq_{hb})$ is a \textbf{well-formed execution} if:
\begin{enumerate}
   \item Each read of a var $x$ sees a write to $x$, and all reads and writes of volatile variables are volatile actions
   \item Synchronization order is consistent with program order and mutual exclusion
   \item The execution obeys intra-thread consistency
   \item The execution obeys intra-thread and happens-before consistency on
   TODO

\end{enumerate}

Now, which \textit{well-formed executions} are \textbf{legal}?
\textbf{Legal executions} are built iteratively:
in each iterations, the JMM commits a set of memory actions;
actions can be committed if they occur in some well-behaved...

A well-formed $E = (P,A,\leq_{po},\leq_{so},W,V,\leq_{sw},\leq_{hb})$ is validated by \textit{commiting} actions in A;
if all actions of A are committed, then E is legal.
There must exists a sequence of subsets of a
\[
  C_0 C_1 = A 
\]
and one $\{E_i\}$ of well-formed executions such that each $E_i$ witnesses the actions in $C_i$


TODO many slides have been skipped

\section{Intructions reordering}
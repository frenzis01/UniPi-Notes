\chapter{RUST}
\textbf{Rust} is a general purpose, system programming language
with a focus on safety, especially \textbf{safe concurrency},
supporting both \textit{functional} and \textit{imperative} paradigms.
Its main goal is to \textit{ensure \underline{safety} without penalizing \underline{efficiency}}.\\
\texttt{C/C++} provide more control but less safety, while \texttt{Python/Haskell} provide less control but more safety.
\textbf{Rust} aims to get the best of both worlds, providing both \textbf{control} and \textbf{safety}.

Despite its syntax resemblance to C/C++, in a deeper sense Rust is closer to the ML family languages;
in fact almost every part of a function body is an expression, include \texttt{if-then-else} constructs, which returns a value.

\section{Key Points}
Rust, similarly to C, compilates to \textbf{object code} for bare-metal performance,
but it supports \textbf{memory safety}:
programs can \textit{dereference} only previously allocated pointers that have
not been freed, and \textit{out-of-bound} array accesses not allowed;
Besides, the \textbf{overhead} introduced is very low, since it's the \textit{compiler} which checks that memory safety rules are followed,
and there's \textit{\underline{no} garbage collection}, so zero-cost abstraction in managing memory.\\
This is achieved through and \textbf{advanced type system} and three key concepts to prevent memory corruption:
\begin{enumerate}
   \item Ownership
   \item Borrowing
   \item Lifetime
\end{enumerate}

Again, Rust is designed to be \textbf{memory safe} even in the presence
of concurrency, and guarantees the following properties \textbf{statically}, meaning that
if the program \textit{compiles} it will \textit{never manifest a violation} of these properties: 
\begin{itemize}
   \item No \texttt{null} pointers\\
   $longrightarrow$ accessing a variable which does not hold a value
   \item No dangling pointers\\
   $longrightarrow$ Pointers to invalid memory location
   \begin{itemize}
      \item Pointers to explicitly deallocated objects;
      \item Pointers to locations beyond the end of an array;
      \item Pointers to objects allocated on the stack;
   \end{itemize}
   \item No double frees\\
   $longrightarrow$ A memory location in the heap is reclaimed twice
   \item No data races\\
   $longrightarrow$ unpredictable results in concurrent computations
   \item No iterator invalidation
\end{itemize}

\section{\texttt{null} and Primitive types in Rust}
A \texttt{null} value does \textbf{not} exist in Rust, so in some way it must address the problem of accessing a variable which does not hold a value.\\
Data values can only be initialized through a fixed set of
forms, requiring their inputs to be already initialized, and if 
any branch of code fails to assign a
value to the variable, we get a \textbf{compile time error}.
\note{Static/global variables must be initialized at declaration
time.}
\lstset{language=Rust}
\textit{Nullable} types, are managed with a generic \lstinline|Option<T>|, playing
the role of Haskell’s \texttt{Maybe} or Java’s \texttt{Optional}
\begin{lstlisting}
   enum std::option::Option<T> {
      None,
      Some(T)
   }
\end{lstlisting}

\subsection{Primitive Types}
\begin{lstlisting}[caption={Rust primitive types}]
   // Numeric types:
   i8 / i16 / i32 / i64 / isize
   u8 / u16 / u32 / u64 / usize
   f32 / f64
   
   bool
   char // (4-byte unicode)
\end{lstlisting}
\begin{itemize}
\item \textbf{Type inference} for variables declarations with let
\item \textbf{No overloading} for literals: type annotations to disambiguate
\item \textbf{Tuples} like in Haskell
\item \textbf{Arrays} with fixed length. 
\note{\textit{out-of-bound} access is checked at \textbf{runtime},
but it's just a single comparison, its overhead is negligible}
\end{itemize}

\section{Memory Management}
As usual, Rust uses a \textbf{stack} of activation records, and a \textbf{heap} for dynamically allocated data structures.

The user is forced to be \textit{aware of where} the data are stored: 
there is no \textbf{implicit boxing}\footnote{Act of \textit{boxing} an \texttt{int} in \texttt{Integer}, or extracting an \textit{int} from \textit{Integer}}.

\begin{lstlisting}
   fn main() {
      let x = 3; // 'let' allocates a variable on the stack
      let y = Box::new(3); // y is a reference to 3 on the heap
      println!("x == y is {}", x == *y); // "x == y is true"
      }
\end{lstlisting}

To avoid the overhead of a Garbage collection mechanism and the possible subtle errors introduced a programmer to whom memory management is delegated, Rust provides \textit{deterministic management of
resources}, with very low overhead, using \textbf{RAII} (\textit{\underline{R}esource \underline{A}cquisition \underline{I}s \underline{I}nitialization}).
\nl

By \textit{default}, Rust variables are \textbf{immutable}, and their usage is statically checked by the compiler.
\lstinline|mut| is used to declare a resource as mutable.
\begin{paracol}{2}
\begin{lstlisting}[caption={Compilation \textit{error}}]
   fn main() {
      let a: i32 = 0;
      a = a + 1;
      println!("a == {}", a);
   }
\end{lstlisting}
\switchcolumn
\begin{lstlisting}[caption={Compilation \checkmark}]
   fn main() {
      let mut a: i32 = 0;
      a = a + 1;
      println!("a == {}", a);
   }
\end{lstlisting}
\end{paracol}

The \textit{Resource Acquisition Is Initialization} (\textbf{RAII}) programming idiom states that Resource \textit{allocation} is done during object
\textit{initialization}, by the constructor, while resource \textit{deallocation}
(\textbf{release}) is done during object destruction (specifically
\textbf{finalization}), by the destructor.

This approached is adopted in modern \texttt{C++}: 
small objects are allocated on \textit{stack},
while larger resources are on the \textit{heap} {--}or elsewhere{--} and are \textbf{owned} by an object on the \textit{stack},
who is responsible for \textit{releasing} the resource in its destructor.\\
Each resource has a \textbf{unique owner}.

Rust supports RAII in a \textit{strict} way through an \textbf{ownership system}, based on the concepts of \textit{\underline{ownership}} and \textit{\underline{borrowing}}.
\labelitemize{\textit{Ownership}}{
   \begin{enumerate}[label=\texttt{O\arabic*} - , left=1em]
      \item Every value is \textit{owned} by a variable, identified by a name (possiby a path);
      \item Each value has \textit{at most \underline{one owner} at a \underline{time}};
      \item When the owner goes \textit{out-of-scope}, the
      value is \textit{reclaimed} / destroyed.
   \end{enumerate}
}

By default, an assignment between variables has
a \textbf{\underline{move} semantics}:
the ownership is moved from the RHS to the LHS 
\begin{lstlisting}
   fn main() {
      let x = Box::new(3);
      let _y = x; // underscore to avoid 'unused' warning
      println!("x = {}", x); // error!
      }
\end{lstlisting}
For primitive types and types implementing the \textbf{Copy
trait}, assignment has a \textbf{\underline{copy} semantics};
\begin{center}
   Here \texttt{O2} is satisfied because a new value is created
\end{center}
\begin{paracol}{2}
   \begin{lstlisting}
      fn main() {
         let x = 3;
         let _y = x;
         println!("x = {:?}", x); // OK
      }
   \end{lstlisting}
   
   \switchcolumn

   \begin{lstlisting}
      fn main() {
         let x = Option::Some(3);
         let _y = x;
         println!("x = {:?}", x); // OK
      }
   \end{lstlisting}
\end{paracol}
   
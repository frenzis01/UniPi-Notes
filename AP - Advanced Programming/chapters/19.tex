\chapter{RUST}
\textbf{Rust} is a general purpose, system programming language
with a focus on safety, especially \textbf{safe concurrency},
supporting both \textit{functional} and \textit{imperative} paradigms.
Its main goal is to \textit{ensure \underline{safety} without penalizing \underline{efficiency}}.\\
\texttt{C/C++} provide more control but less safety, while \texttt{Python/Haskell} provide less control but more safety.
\textbf{Rust} aims to get the best of both worlds, providing both \textbf{control} and \textbf{safety}.

Despite its syntax resemblance to C/C++, in a deeper sense Rust is closer to the ML family languages;
in fact almost every part of a function body is an expression, include \texttt{if-then-else} constructs, which returns a value.

\section{Key Points}
Rust, similarly to C, compilates to \textbf{object code} for bare-metal performance,
but it supports \textbf{memory safety}:
programs can \textit{dereference} only previously allocated pointers that have
not been freed, and \textit{out-of-bound} array accesses not allowed;
Besides, the \textbf{overhead} introduced is very low, since it's the \textit{compiler} which checks that memory safety rules are followed,
and there's \textit{\underline{no} garbage collection}, so zero-cost abstraction in managing memory.\\
This is achieved through and \textbf{advanced type system} and three key concepts to prevent memory corruption:
\begin{enumerate}
   \item Ownership
   \item Borrowing
   \item Lifetime
\end{enumerate}

Again, Rust is designed to be \textbf{memory safe} even in the presence
of concurrency, and guarantees the following properties \textbf{statically}, meaning that
if the program \textit{compiles} it will \textit{never manifest a violation} of these properties: 
\begin{itemize}
   \item No \texttt{null} pointers\\
   $longrightarrow$ accessing a variable which does not hold a value
   \item No dangling pointers\\
   $longrightarrow$ Pointers to invalid memory location
   \begin{itemize}
      \item Pointers to explicitly deallocated objects;
      \item Pointers to locations beyond the end of an array;
      \item Pointers to objects allocated on the stack;
   \end{itemize}
   \item No double frees\\
   $longrightarrow$ A memory location in the heap is reclaimed twice
   \item No data races\\
   $longrightarrow$ unpredictable results in concurrent computations
   \item No iterator invalidation
\end{itemize}

\section{\texttt{null} and Primitive types in Rust}
A \texttt{null} value does \textbf{not} exist in Rust, so in some way it must address the problem of accessing a variable which does not hold a value.\\
Data values can only be initialized through a fixed set of
forms, requiring their inputs to be already initialized, and if 
any branch of code fails to assign a
value to the variable, we get a \textbf{compile time error}.
\note{Static/global variables must be initialized at declaration
time.}
\lstset{language=Rust}
\textit{Nullable} types, are managed with a generic \lstinline|Option<T>|, playing
the role of Haskell’s \texttt{Maybe} or Java’s \texttt{Optional}
\begin{lstlisting}
   enum std::option::Option<T> {
      None,
      Some(T)
   }
\end{lstlisting}

\subsection{Primitive Types}
\begin{lstlisting}[caption={Rust primitive types}]
   // Numeric types:
   i8 / i16 / i32 / i64 / isize
   u8 / u16 / u32 / u64 / usize
   f32 / f64
   
   bool
   char // (4-byte unicode)
\end{lstlisting}
\begin{itemize}
\item \textbf{Type inference} for variables declarations with let
\item \textbf{No overloading} for literals: type annotations to disambiguate
\item \textbf{Tuples} like in Haskell
\item \textbf{Arrays} with fixed length. 
\note{\textit{out-of-bound} access is checked at \textbf{runtime},
but it's just a single comparison, its overhead is negligible}
\end{itemize}

\section{Memory Management}
As usual, Rust uses a \textbf{stack} of activation records, and a \textbf{heap} for dynamically allocated data structures.

The user is forced to be \textit{aware of where} the data are stored: 
there is no \textbf{implicit boxing}\footnote{Act of \textit{boxing} an \texttt{int} in \texttt{Integer}, or extracting an \textit{int} from \textit{Integer}}.

\begin{lstlisting}
   fn main() {
      let x = 3; // 'let' allocates a variable on the stack
      let y = Box::new(3); // y is a reference to 3 on the heap
      println!("x == y is {}", x == *y); // "x == y is true"
      }
\end{lstlisting}

To avoid the overhead of a Garbage collection mechanism and the possible subtle errors introduced a programmer to whom memory management is delegated, Rust provides \textit{deterministic management of
resources}, with very low overhead, using \textbf{RAII} (\textit{\underline{R}esource \underline{A}cquisition \underline{I}s \underline{I}nitialization}).
\nl

By \textit{default}, Rust variables are \textbf{immutable}, and their usage is statically checked by the compiler.
\lstinline|mut| is used to declare a resource as mutable.
\begin{paracol}{2}
\begin{lstlisting}[caption={Compilation \textit{error}}]
   fn main() {
      let a: i32 = 0;
      a = a + 1;
      println!("a == {}", a);
   }
\end{lstlisting}
\switchcolumn
\begin{lstlisting}[caption={Compilation \checkmark}]
   fn main() {
      let mut a: i32 = 0;
      a = a + 1;
      println!("a == {}", a);
   }
\end{lstlisting}
\end{paracol}

The \textit{Resource Acquisition Is Initialization} (\textbf{RAII}) programming idiom states that Resource \textit{allocation} is done during object
\textit{initialization}, by the constructor, while resource \textit{deallocation}
(\textbf{release}) is done during object destruction (specifically
\textbf{finalization}), by the destructor.

\subsection{Ownership}

This approached is adopted in modern \texttt{C++}: 
small objects are allocated on \textit{stack},
while larger resources are on the \textit{heap} {--}or elsewhere{--} and are \textbf{owned} by an object on the \textit{stack},
who is responsible for \textit{releasing} the resource in its destructor.\\
Each resource has a \textbf{unique owner}.

Rust supports RAII in a \textit{strict} way through an \textbf{ownership system}, based on the concepts of \textit{\underline{ownership}} and \textit{\underline{borrowing}}.
\labelitemize{\textit{Ownership}}{
   \begin{enumerate}[label=\texttt{O\arabic*} - , left=1em]
      \item Every value is \textit{owned} by a variable, identified by a name (possiby a path);
      \item Each value has \textit{at most \underline{one owner} at a \underline{time}};
      \item When the owner goes \textit{out-of-scope}, the
      value is \textit{reclaimed} / destroyed.
   \end{enumerate}
}

By default, an assignment between variables has
a \textbf{\underline{move} semantics}:
the ownership is moved from the RHS to the LHS 
\begin{lstlisting}
   fn main() {
      let x = Box::new(3);
      let _y = x; // underscore to avoid 'unused' warning
      println!("x = {}", x); // error!
      }
\end{lstlisting}
For primitive types and types implementing the \textbf{Copy
trait}, assignment has a \textbf{\underline{copy} semantics};
\begin{center}
   Here \texttt{O2} is satisfied because a new value is created
\end{center}
\begin{paracol}{2}
   \begin{lstlisting}
      fn main() {
         let x = 3;
         let _y = x;
         println!("x = {:?}", x); // OK
      }
   \end{lstlisting}
   
   \switchcolumn

   \begin{lstlisting}
      fn main() {
         let x = Option::Some(3);
         let _y = x;
         println!("x = {:?}", x); // OK
      }
   \end{lstlisting}
\end{paracol}

The same move semantics apply also for parameter passing:
Any value passed to the function will be reclaimed
when it returns, as the formal parameters gets out of
scope;
only returned values can survive.
\note{
   tuples allow to return more
}
\begin{lstlisting}
   struct Dummy { a: i32, b: i32 }
   fn foo() {
      let mut res = Box::new(Dummy {
         a: 0,
         b: 0
   });
   take(res);
   println!("res.a = {}", res.a); \\ compilation error
   }
   fn take(arg: Box<Dummy>) {...}
\end{lstlisting}

When invoking \lstinline|take(res)| the ownership of \lstinline|Dummy| is moved from \lstinline|res| to \lstinline|arg|:
when \texttt{take()} returns \lstinline|arg| goes out of scope, so the resource gets freed automatically, making it no longer usable in \lstinline|println|:
this result in a \textbf{compilation error}.
To use again the resource, we would have to make take return it, i.e. \lstinline|res = take(res)|.

This looks rather limiting, but allows to completely avoid the \textit{Double-free} problem:
memory is freed automatically
when the owner goes out of scope, and by rule \texttt{O2}, each value has only one owner.
\note{Rust does not allow explicit memory allocation}

\subsection{Borrowing}
Since Ownership rules in some case may be too restrictive, \textbf{borrowing} is introduced: a resource can be \textit{borrowed} from its owner via
assignment or parameter passing.
To guarantee memory safety, borrowing rules ensure
that \textit{aliasing}\footnote{Both the owner and the borrower can access the resource.
More generally indicates that there are multiple ways to access a resource on the heap.} and \textit{mutability cannot \textbf{coexist}}.\\
Values can be passed
\begin{enumerate}
   \item by immutable reference $\longrightarrow$ \lstinline|x = &y|
   \item by mutable reference $\longrightarrow$ \lstinline|x = &mut y|
   \item or by value $\longrightarrow$ \lstinline|x = y|
\end{enumerate}

\labelitemize{\textit{Borrowing}}{
\begin{enumerate}[label=\texttt{O\arabic*} - , left=1em]
   \label{enum:borrowing_rules}
   \item[] \note{\qquad About mutable and immutable references:}
   \item At most \textbf{one} \textit{mutable} reference to a resource can exist at any time
   \item If there is a \textit{mutable} reference, \textbf{no} \textit{immutable} references can exist
   \item If there is \textbf{no}\textit{ mutable} reference,\textbf{several}
   \textit{immutable} references to the same resource can exist

   \item[] \note{\qquad During borrowing, ownership is reduced or
   suspended:}
   \item Owner \textit{cannot} free or \textbf{mutate} its resource while it is \textit{immutably borrowed}
   \item Owner \textit{cannot} even \textbf{read} its resource while it is \textit{mutably borrowed}
\end{enumerate}
}

\subsection{Strings}
\labelitemize{\textit{String types}}{
   \begin{enumerate}
      \item \lstinline|String|
      does not require to know the length at compilation
      time, thus allocated on the \textit{heap}.
      \item \lstinline|&str|
      size must be known statically, allocated on the \textit{stack}.
   \end{enumerate}
   }
\note{
   Method \lstinline|String::from()| allocates memory on the heap: it takes an argument of type \lstinline|&str| and returns a \lstinline|String|.
}

A String object has three components:
\begin{enumerate}
   \item a reference to the heap location containing the character sequence
   \item capacity (unsigned integer)
   \item length (unsigned integer)
\end{enumerate}
\lstinline|String| does not implement \lstinline|Copy|, thus assignment is subject to move semantics;
assignment creates a copy of length, capacity and reference,
but not of the char sequence in the heap.

\subsection{Lifetime}
A \textbf{lifetime} is a construct that the borrow checker uses to ensure the validity of the \textit{borrowing rules} \ref{enum:borrowing_rules}.
Lifetimes are associated with each individual ownership
and borrowing: 
a lifetime \textit{begins} when the ownership starts, and \textit{ends}
when it is moved / destroyed, 
while for borrowings, it ends where the borrowed value is
accessed the last time.

Lifetimes are mostly \textit{inferred},
but sometimes they must be made explicit using the same syntax of generics.
Using lifetimes, the compiler checks the validity of the
rules of ownership and borrowing in the expected way;
in particular, it ensures that {--}the \textit{owner} of{--} every
borrowed variable/reference has a lifetime that is longer
than the borrower \texttt{[B4,B5]}.
\nl

Borrowed (reference) formal parameters (arguments, return value) of a function have a
lifetime, and  
if borrowed values are returned, each \textit{must} have a lifetime.\\
The compiled tries to infer output lifetimes according to the following rules, but when not sufficient explicit lifetimes are necessary:
\labelitemize{\textit{Lifetime}}{
   \begin{enumerate}[label=\texttt{R\arabic*} - , left=1em]
      \item The lifetimes of the borrowed paramers are, by default, all \textbf{distinct}
      \item If there is \underline{exactly} \textbf{one input} lifetime, it will be assigned to \textbf{each
      output} lifetime
      \item If a method has \textbf{more than one input} lifetime, \textit{but} \textbf{one} of them is
      \lstinline|&self| or \lstinline|&mut self|, then this lifetime is assigned to \textbf{all output} lifetimes
   \end{enumerate}
}

\begin{lstlisting}
   fn longest(s1: &str, s2: &str) -> &str { //does not compile
      if s1.len() > s2.len() { s1 }
      else { s2 }
   }
\end{lstlisting}
Here the lifetime of the parameters depends on whether \lstinline|s1| or \lstinline|s2| is returned,
so the compiler cannot infer the lifetime of the output parameters;
hence, an \textbf{explicitly named lifetime} for input parameters is requires, as in the following snippet.
\begin{lstlisting}
   fn longest<'a>(s1: &'a str, s2: &'a str) -> &'a str {
      if s1.len() > s2.len() { s1 }
      else { s2 }
\end{lstlisting}
\newpage

\section{More on Types}
\subsection{Enums}
\begin{paracol}{2}
   \begin{lstlisting}
enum RetInt {
      Fail(u32),
      Succ(u32)
   }
fn foo_may_fail(arg: u32) -> RetInt {
      let fail = false;
      let errno: u32;
      let result: u32;
      ...
      if fail {
            RetInt::Fail(errno)
         } else {
            RetInt::Succ(result)
         }
   }
\end{lstlisting}
\switchcolumn
\begin{lstlisting}
#[derive(Debug)] // needed to print
enum Tree<T> {
   Empty,
   Node(T, Box<Tree<T>>, Box<Tree<T>>)
}

fn main() {
   let tree = Tree::Node(
      42,
      Box::new(Tree::Node(
         0,
         Box::new(Tree::Empty),
         Box::new(Tree::Empty)
      )),
      Box::new(Tree::Empty));
   println!("{:?}", tree);
   //>Node(42, Node(0, Empty, Empty), Empty)
}
\end{lstlisting}
   \lstinline|println!("{:?}", tree);| indicates to print \lstinline|tree| in \textit{"debug mode"}.
\end{paracol}

\subsection{Pattern Matching}
\begin{lstlisting}
   let x = 5; // try others...
   match x {
      1              => println!("one"),
      2              => println!("two"),
      3|4            => println!("three or four"),
      5..=10         => println!("five to ten"),
      e @ 11..=20    => println!("{}", e),
      i32::MIN..=0   => println!("less than zero"),
      21..           => println!("large"),
      _              => println!("???"),
   }
\end{lstlisting}


\subsection{Classes}
Rust is \textbf{not} \textit{Object Oriented} and there is \textbf{no inheritance}, instead it pushes for composition over inheritance.

\begin{lstlisting}
#[derive(Debug)]
struct Rectangle { // class
      width: u32, // instance variable
      height: u32,
   }
impl Rectangle { // methods
      fn area(&self) -> u32 { // first argument is this
            self.width * self.height
            // self.width = 20; // <- illegal, self is immutable
         }
   }
fn main() {
      let rect1 = Rectangle {
            width: 30,
            height: 50,
         };
      println!(
      "The area of the rectangle is {} square pixels.", rect1.area()
      );
   }
\end{lstlisting}

\subsection{Traits}
\textbf{Traits} are equivalent to \textit{Type Classes} in Haskell and to \textit{Concepts} in
C++20, similar to Interfaces in Java.
A trait can include \textit{abstract} and \textit{concrete} (default)
\textbf{methods}, but \underline{not} fields or variables.
A struct can implement a trait providing an
implementation for at least its abstract methods
\begin{lstlisting}
   impl <TraitName> for <StructName>{ ... }
\end{lstlisting}

The \lstinline|#[derive]| clause can be used {--}if possible{--} to derive
automatically an implementation of a trait.

Rust supports \textbf{bounded universal explicit polymorphism}
with \textbf{generics}, as in Java, where bounds are one or
more traits. 
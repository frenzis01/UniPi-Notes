\chapter{Polymorphism}
Polymorphism basically means \textit{"many forms"}, where \textit{forms} are \textbf{types}.
Thus there may be \textit{polymorphic} function names, or \textit{polymorphic} types.

There are many "flavors" of polymorphism, many variations.
Two main kinds opposed to each other are \textit{ad hoc} and \textit{universal} polymorphism, which however, may coexist:
\begin{itemize}
    \item \textbf{ad hoc} PM indicates that a single function name denotes different algorithms, determined by the actual types.
    \item \textbf{universal} PM indicates a single algorithm (solution) applicable to objects of different types.
\end{itemize}

When PM is taken into account, it is crucial to consider when happens the \textbf{binding} between a function name and the actual code to be executed:
\begin{itemize}
    \item compile time; \textit{static/early binding}
    \item linking time
    \item execution time; \textit{late/dynamic binding}
\end{itemize}
In general the earlier the binding happens, the better (for debugging reasons).
If the binding spans over more phases (e.g. \textit{overriding} in Java), as a convention we consider the \textbf{binding time} the last phase.

\section{Classification}
\begin{figure}[h]
    \centering
    \includegraphics[width=0.7\textwidth]{images/polymorphism.png}
    \caption{Polymorphism classification}
    \label{fig:polymorphism_classification}
\end{figure}
\subsection{Overloading}
\textbf{Overloading} is present in every language for basic operators $+-*...$,
and sometimes is supported for user-defined functions, 
and in some languages it is even allowed the overloading of primitive operator by user-defined functions.\\
Since this falls under the \textbf{ad hoc} polymorphism family,
the code to be executd is determined by the type of the arguments;
the binding can either happen at \textit{compile} or at \textit{runtime},
depending on the typing of the language,
whether it is static or dynamic.

\begin{lstlisting}
    // C language doesn't allow overloading for user-defined functions
    int sqrInt(int x) { return x * x; }
    double sqrDouble(double x) { return x * x; }
    
    // Overloading in Java & C++
    int sqr(int x) { return x * x; }
    double sqr(double x) { return x * x; }
\end{lstlisting}

\note{Haskell introduces \textbf{type classes} for handling
overloading in presence of type inference}

\section{Coercion}
\textbf{Coercion} is the automatic (implicit) conversion of an object to a different type, opposed to casting which is explicit instead.
Coercion allows a code snippet to be applied of arguments of different (convertible) types.
Sometimes coercion is allowed only if there is no \textbf{information loss}.

\begin{lstlisting}
    double sqrt(double x){...}
    double d = sqrt(5) // applied to int
\end{lstlisting}

´\section{Inclusion Polymorphism}
Inclusion polymorphism is also known as \textit{subtyping polymorphism} or \textit{\textbf{inheritance}}.
It is ensured by \textit{Barbara Liskov}'s \textbf{substitution principle}:
\begin{center}
    \textit{A subtype object can be used in any context where a supertype object is expected}
\end{center}
Methods and fields defined in a superclass may be invoked and accessed by subclasses if not redefined (see \textit{Overriding}).

\section{Overriding}
In Java a method $m$ of a
class $A$ can be redefined
in a subclass $B$ of $A$.

Overriding introduces ad hoc polymorphism in the universal polymorphism of inheritance.
Notice that overriding requires the final binding to happen at runtime:
it happens through the lookup done by \lstinline{invokevirtual} in the JVM.

\section{C++ v Java}
\begin{lstlisting}[language=C++]
class A {
 public:
  virtual void onFoo() {}
  virtual void onFoo(int i) {}
};
class B : public A {
 public:
  virtual void onFoo(int i) {}
};
class C : public B {};
int main() {
  C* c = new C();
  c->onFoo();
  // Compile error - doesn't exist
}
\end{lstlisting}
The equivalent code in Java compiles, because in java invokes the function \lstinline{onFoo()} with no arguments defined in the superclass A.
In C++ instead, the function onFoo(int i) defined in B is found and stops the search, but there is arguments type mismatch, thus it doesn't compile.
This happens because in C++ the method lookup is based on the method \textit{name}, not on its \textit{signature}.

\section{C++ Templates}
They are similar to \textit{Generics} in Java, 
they are used as function and class templates
each concrete instantiation produces a copy of the generic code, specialized for that type:
monomorphization.
In java Generics, instead, \textbf{type erasure} happens at runtime, i.e. type variables \lstinline{T} are replaced by \lstinline{Object} variables.\\
Templates support parametric polymorphism and type parameters can also be primitive types (unlike Java generics)
\begin{lstlisting}[language=C++]
    template <class T> // or <typename T>
    T sqr(T x) { return x * x; }
\end{lstlisting}
Assuming to invoke \lstinline{sqr(T x)} on variables of different types, the compiler will generated a specific code for each type used.
This works even on user-defined types;
check the following code for an example:
\begin{lstlisting}
class Complex {
 public:
  double real;
  double imag;
  Complex(double r, double im) : real(r), imag(im){};
  Complex operator*(Complex y) {  // overloading of *
    return Complex(real * y.real - imag * y.imag,
                   real * y.imag + imag * y.real);
  }
};

Complex cc = sqr(c); // legal and produces a function "Complex sqr(Complex x) {...}"
\end{lstlisting}

It is important to check for type ambiguosity;
in the following example, it is highlighted a case where it's not clear whether it is \lstinline{i} to be converted to \lstinline{long} or \lstinline{m} to \lstinline{i}.
\begin{lstlisting}
template <class T>
T GetMax(T a, T b) {  return (a > b) ? a : b; }
...
n = GetMax(l, m);       // ok: GetMax<long>
// v = GetMax(i, m);    // no: ambiguous
v = GetMax<int>(i, m);  // ok
\end{lstlisting}

\subsection{Macros}
\textbf{Macros} can be exploited to achieve \textit{polymorphism} and can have the same effect of the templates,
but notice that macros are executed by the preprocessor\footnote{Macro expansion can be seen using the option \textit{-E} when compiling} and are only \textbf{textual substition},
there is no parsing, no static analysis checks or whatsoever.
\begin{lstlisting}
#define sqr(x) ((x) * (x))
int a = 2;
int aa = sqr(a++); // int aa = ((a++) * (a++));
// value of aa? aa contains 6  :(

#define fact(n) (n == 0) ? 1 : fact(n-1) * n
// compilation fails because fact is not defined
\end{lstlisting}
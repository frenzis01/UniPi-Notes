\chapter{Type Inference}
\begin{figure}[htbp]
   \centering
   \includegraphics{images/haskell_polymorphism.png}
   \caption{Haskell Polymorphism Recap}
   \label{fig:haskell_polymorphism}
\end{figure}

\section{Overloading}
Haskell allows \textbf{overloading} even of \textbf{primitive types}:
the code to be executed is determined by the type of the arguments,
leading to have \textit{early binding} in \textit{statically} typed languages
or \textit{late binding} in \textit{dynamically} typed languages.

In Haskell we can write the following, but what is the type?
\begin{lstlisting}
   sqr x = x * x
\end{lstlisting}

When considering overloading besides arithmetic, we find that some functions are \textbf{fully polymorphic}:
\begin{lstlisting}
   length :: [w] -> Int
\end{lstlisting}

While others not so much;
for example, \textit{membership} works only for types that support equality,
while \textit{sorting} works only for types which support \textit{ordering}. 
\begin{lstlisting}
   member :: [w] -> w -> Bool
   sort :: [w] -> [w]
\end{lstlisting}

\section{Type Classes}
\textbf{Type Classes} solve many overloading problems concerning arithmetic and equality (and similar properties) support.

The idea is to generalize ML’s eqtypes to arbitrary types
and provide concise types to describe overloaded
functions, so no exponential blow-up (i.e. defining functions for every possible combination of type arguments).\\
Type classes allow users to define functions using overloaded
operations {---}e.g. square, squares, and member{---} and to
declare new collections of
overloaded functions: equality and arithmetic
operators are not privileged built-ins.
Haskell's solutions fits perfectly within type inference framework.

The intuition is that a sorting function may allow to be passed a comparison \lstinline|cmp| operator as argument,
thus making the function parametric.
\begin{lstlisting}
   qsort:: (a -> a -> Bool) -> [a] -> [a]
   qsort cmp [] = []
   qsort cmp (x:xs) = qsort cmp (filter (cmp x) xs) ++ [x] ++
   qsort cmp (filter (not.cmp x) xs)
\end{lstlisting}

Developing this idea, consider rewriting the parabola function to take operators as argument
\begin{lstlisting}
   parabola x = (x * x) + x
   parabola' (plus, times) x = plus (times x x) x
\end{lstlisting}
Here the extra parameter is a \textit{\textbf{dictionary}} that provides implementations for the overloaded ops.
These implies rewriting calls to pass appropriate implementations for plus and times:
\begin{lstlisting}
   y = parabola'(intPlus,intTimes) 10
   z = parabola'(floatPlus, floatTimes) 3.14
\end{lstlisting}

\begin{enumerate}
   \item Type class declarations
   \begin{enumerate}
      \item Define a set of operations, give it a name
      \item Example: \lstinline|Eq a| type class
      • operations \lstinline|==| and \lstinline|\=| with \lstinline|type a -> a -> Bool|
   \end{enumerate}
   \item Type class instance declarations
   \begin{enumerate}
      \item Specify the implementations for a particular type
      \item For \lstinline|Int| instance, \lstinline|==| is defined to be integer equality
   \end{enumerate}
   \item Qualified types (or Type Constraints)
   Concisely express the operations required on otherwise polymorphic type
   \lstinline|member:: Eq w => w -> [w] -> Bool|
\end{enumerate}

\labelitemize{
   \textit{implementation summary}
}{
   \begin{enumerate}
      \item Each overloaded symbol has to be introduced in at least one type class
      \item The compiler translates each function that uses an overloaded symbol into a function with an extra parameter: the dictionary.
      \item References to overloaded symbols are rewritten by the compiler to lookup the symbol in the dictionary.
      \item The compiler converts each type class declaration into a dictionary type declaration and a set of selector functions.
      \item The compiler converts each instance declaration into a dictionary of the appropriate type.
      \item The compiler rewrites calls to overloaded functions to pass a dictionary. It uses the static, qualified type of the function to select the dictionary.
   \end{enumerate}
}

\subsection{Compositionality}
\begin{lstlisting}
   class Eq a where
   (==) :: a -> a -> Bool
   instance Eq Int where
   (==) = intEq -- intEq primitive equality
   instance (Eq a, Eq b) => Eq(a,b) where
   (u,v) == (x,y) = (u == x) && (v == y)
   instance Eq a => Eq [a] where
   (==) [] [] = True
   (==) (x:xs) (y:ys) = x==y && xs == ys
   (==) _ _ = False
\end{lstlisting}

\subsection{Compound Translation}
\newpage
\subsection{Subclasses}
\begin{paracol}{2}
   \vspace{\fill}
   
   A subclass declaration expresses this relationship:
   \begin{lstlisting}
      class Eq a => Num a where
      (+) :: a -> a -> a
      (*) :: a -> a -> a
   \end{lstlisting}
• With that declaration, we can simplify the type of the function

\begin{lstlisting}
   memsq :: (Eq a, Num a) => a -> [a] -> Bool
   memsq x xs = member (square x) xs
\end{lstlisting}

\vspace{\fill}
\switchcolumn

\begin{figure}[htbp]
   \centering
   \includegraphics{images/haskell_subclasses.png}
   \caption{Haskell Subclasses relationships}
   \label{fig:haskell_subclasses}
\end{figure}

\end{paracol}

\subsection{Deriving}
For Read, Show, Bounded, Enum, Eq, and Ord, the compiler
can generate instance declarations automatically.
\begin{lstlisting}
   data Color = Red | Green | Blue
      deriving (Show, Read, Eq, Ord)
   
   Main>:t show
   show :: Show a => a -> String
   Main> show Red
   "Red"
   Main> Red < Green
   True
   Main>:t read
   read :: Read a => String -> a
   Main> let c :: Color = read "Red"
   Main> c
   Red
\end{lstlisting}

\subsection{Numeric Literals}
\begin{lstlisting}
   class Num a where
      (+) :: a -> a -> a
      (-) :: a -> a -> a
      fromInteger :: Integer -> a
      -- Even literals are overloaded.
      -- 1 :: (Num a) => a
      ...

   inc :: Num a => a -> a
   inc x = x + 1
\end{lstlisting}

\labelitemize{
   \textit{Advantages}
}{
   Numeric literals can be interpreted as values of any
   appropriate numeric type,
   for example: 1 can be an Integer or a Float or a user-
   defined numeric type.
}
\chapter{Introduction}
% \section*{19 - Settembre}
\subsection*{Info and Contact}
% Ricevimento Lunedì 14:30 - 16:00/16:30

\href{https://pages.di.unipi.it/corradini/Didattica/AP-23/}{Pagina del corso}
% \href{http://didawiki.di.unipi.it/doku.php/magistraleinformaticanetworking/ae/ae2023/start}{Didawiki Corso}
\section{Introduction}
\subsection{Framework}
A software \textbf{framework} is a collection of common code providing generic functionality that can selectively overridden or specialized by user code providing specific functionality.

When using \textit{frameworks} there is an \textbf{inversion of control}.
Differently from what happens when using libraries,
the program-flow is dictated by the framework, not by the caller.

\subsection{Design Patterns}
A \textbf{design pattern} is a general reusable solution to a commonly occurring problem within a given context in software design. 
A design pattern is characterized by:
\begin{itemize}
    \item \textbf{Name}
	\item \textbf{Problem Addressed}
	\item \textbf{Context} - Used to determine applicability
	\item \textbf{Forces} - Constraints or issues that the solution must address
	\item \textbf{Solution} - It must resolve all \textit{forces}
\end{itemize}

% \section*{20 - Settembre}
% Goldbolt is a useful tool, to see preprocessor output, compiling, ecc.
\subsection{Programming Languages}
A \textbf{PL} is defined via \textbf{syntax}, \textbf{semantics} and \textbf{pragmatics}\footnote{the way in which the PL is intended to be used in practice}.
\subsubsection{Syntax}
Used by the compiler for \textit{scanning} and \textit{parsing}.
The \textit{lexical} grammar defines the syntax of token (e.g. "for" blocks, constants)
\subsubsection{Semantics}
Semantics are the meaning of the language constructs, they may be described using natural language, which even if precise, allows ambiguousity.
\textbf{Formal} approches to semantics definition are:
\begin{enumerate}
    \item \textit{Denotational} - Mapping every syntactic entity with a mathematical entity
    \item \textit{Operational} - Defining a computation relation in a form $e \Rightarrow v$, where $e$ is a program
    \item \textit{Axiomatic} - Based on Hoare-triples $\textit{Precondition} \wedge \textit{Program}\Rightarrow \textit{Postcondition}$ 
\end{enumerate}
However, they rarely scale to fully-fledged programming languages.
\subsubsection{Pragmatics}
\textit{Pragmatics} include coding conventions, guidelines for elegant code, etc.

\subsection{Programming Paradigms}
Paradigms belong to languages \textit{pragmatics}, not to the way the language is defined, i.e. not syntax nor semantics.
\begin{enumerate}
    \item \textbf{Imperative}
    \item \textbf{Object-oriented}
    \item \textbf{Concurrent} - Processes, communication, \dots 
    \item \textbf{Functional} - values, expressions,
    functions, higher-order functions, \dots
    \item \textbf{Logic} - Assertions, relations, \textit{strange sorceries}\dots 
\end{enumerate}

Modern PLs, provide constructs and solutions to program in all these paradigms

\subsection{Implementing PLs}
\begin{itemize}
    \item Programs written in \textbf{L} must be \textit{executable}
    \item Every language \textbf{L} implicitly defines and \textit{Abstract Machine} \textbf{$M_L$} having \textbf{L} as a Machine Language
    \item Implementing $M_L$ on an existing host machine $M_O$ via compilation or interpretation (or both) makes programs written in \textbf{L} executables 
\end{itemize}

An \textbf{Abstract Machine} $M_L$ for $L$ is a \ul{collection of data structures and algorithms which can perform the storage and execution of programs written in L.}\\
Viceversa, $M$ defines a language $L_M$ including all programs which can be executed by the interpreter $M$.\\
There is a \ul{bidirectional correspondance} between machines and languages components.\\
\begin{center}
$
\begin{array}{ccc}
    \textit{Primitive data processing} & \longleftrightarrow & \textit{Primitive data types}  \\
    \textit{Sequence control} & \longleftrightarrow & \textit{Control structures}  \\
    \textit{Data transfer control }& \longleftrightarrow & \textit{Parameter passing and value return}  \\
    \textit{Memory management }& \longleftrightarrow & \textit{Memory management}\\
     & 
\end{array}
$   
\end{center}

\begin{figure}[htbp]
    \centering
    \includegraphics{images/interpreter_structure.png}
    \caption{fetch-decode-execute cycle. \textit{``operands''} are the data to be processed} 
    \label{fig:interpreter_structure}
\end{figure}

In Java, the \textit{Java Virtual Machine} is the \textit{Abstract Machine} for Java, and the Java language is the \textit{Machine Language} for the JVM.


In computer science one of the main focuses is \textbf{abstraction}, and this applies also to the implementation of AMs, leading to an onion-like structure, or \textbf{hierarchical} structure,
as can be seen in the scheme in \ref{fig:machines_hierarchy}.

\begin{figure}[htbp]
    \centering
    \includegraphics{images/machines_hierarchy.png}
    \caption{AM Hierarchical Structure}
    \label{fig:machines_hierarchy}
\end{figure}

% A software component is a unit of composition with
% contractually specified interfaces and explicit context
% dependencies only. A software component can be
% deployed independently and is subject to
% composition by third parties.

\subsection*{Implementing PLs - Wrap Up}
\begin{itemize}
    \item $L$ High-level programming language
    \item $M_L$ Abstract machine for $L$
    \item $M_O$ host machine
\end{itemize}

\subsubsection{Pure Interpretation}

\begin{figure}[htbp]
    \centering
    \includegraphics{images/am_pure_int.png}
    \caption{Pure Interpretation}
    \label{fig:am_pure_int.png}
\end{figure}

The source program L is executed by $M_L$.
$M_L$ is not native to $M_O$; it is instead interpreted by $M_O$.
$M_L$ is interpreted over $M_O$, so every instruction of $L$ corresponds into a sequence of instructions in $L_O$. Such decoding is not a proper ``translation'', because the code in $L_O$ corresponding to an instruction of $L$ is directly executed by the interpreter, it does not ``output'' a translated program.\\
It isn't very efficient, mainly because of fetch-decode phases.

\subsubsection{Pure Compilation}

\begin{figure}[htbp]
    \centering
    \includegraphics{images/am_pure_comp.png}
    \caption{Pure Compilation}
    \label{fig:am_pure_comp.png}
\end{figure}

$L$ programs are translated into $L_O$, the machine laguange of $M_O$,
hence, $M_L$ is not realized at all and the programs are directly executed on $M_O$.\\
The Compiler may be executed on a different abstract machine and still produce code for $M_O$, but this ain't mandatory.
\textit{Compilation} is more efficient than \textit{Interpretation}, 
but produced code is larger.

\subsubsection{Both}

\begin{figure}[htbp]
    \centering
    \includegraphics{images/am_both.png}
    \caption{Both}
    \label{fig:am_both}
\end{figure}

All real languages use both \textit{interpretation} and \textit{compilation}, the \textit{``pure''} cases are only limit cases.
So we use an \textit{Intermediate Representation} (IR) to compile the code, and then interpret the IR.
If IR is close to the source language, we say that we have a compiled implementation, otherwise we have an interpreted implementation.

Some languages, e.g. Java, use an intermediate Abstract Machine, called a \textit{Virtual Machine},
which increases \textit{Portability} and \textit{Interoperability}.
Java compilers generate bytecode (\texttt{.class} files) that is interpreted by the Java virtual machine (JVM), meaning that the JVM reads the bytecode and executes the instructions one by one, translating them into native machine code on the fly. The JVM may translate bytecode into machine code by just-in-time (JIT) compilation.

\begin{itemize}
	\item Compilation leads to better performance in general
	    \begin{itemize}
		    \item Allocation of variables without variable lookup at run time
		    \item Aggressive code optimization to exploit hardware features
	    \end{itemize}
	\item Interpretation facilitates interactive debugging and testing
	    \begin{itemize}
		    \item Interpretation leads to better diagnostics of a programming problem
		    \item Procedures can be invoked from command line by a user
		    \item Variable values can be inspected and modified by a user
	    \end{itemize}
\end{itemize}

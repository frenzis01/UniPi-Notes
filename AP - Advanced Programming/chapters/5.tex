\chapter{Java Beans}
\section{3 - Ottobre}
"A \textbf{Java Bean} is a \textit{reusable} software component that can be
manipulated visually in a builder tool."

Typically a Bean has a GUI representation but is not necessary.
What is necessary instead for a class to be recognized as a bean is that it:
\begin{itemize}
    \item has a public constructor with no arguments
    \item implements \lstinline{java.io.Serializable}
    \item is in a \lstinline{jar} file with a \textit{manifest} file that contains:
    \lstinline{Java-Bean: True}
\end{itemize}

Beans can be \textbf{assembled} to build a new bean or application, writing clue code to wire beans together.
\textit{Connection-oriented} programming is based on the \textbf{Observer} or (\textit{Publish-Subscribe}) paradigm.
\textit{Observers} come into play when there is a $1:N$ dependency between objects and one of them changes state, creating the need for the others to be notified and updated.
Beans must be able to run in a \textit{design environment} allowing the user to customize aspect and behaviour.
Beans provide support for some standard features:
\begin{enumerate}
    \item \textbf{Properties} e.g. color. 
    \textbf{Bounded} properties generate an \textit{event} of type \lstinline{PropertyChangeEvent},
    while \textbf{constrained} can only change value if none of the registered \textit{observers} "poses a veto",
    by raising an \textit{exception} when they receive the \lstinline{PropertyChangeEvent} object.
    \item \textbf{Events}: The \textbf{Observer} pattern is based on \textit{Events} and \textit{Events listeners}.
    An \textit{event} is an object created by an \textit{event source} and propagated to the registered \textit{event listeners}.
    Sometimes event \textbf{adaptors} can be placed between source and listener, 
    which might implement queuing mechanism, filter events, demuxing from many sources to a single listener.
    \begin{itemize}
        \item \textbf{Design Patterns for Events}
        \begin{lstlisting}
    public void add<EventListType>(<EventListType> a)
    public void remove<EventListType>(<EventListType> a)
        \end{lstlisting}
    \end{itemize}
    \item \textbf{Customization}
    \item \textbf{Persistence}
    \item \textbf{Introspection}: process of analyzing a bean to determine capabilities.
    There are implicit methods based on \textit{reflection}, \textit{naming conventions} and \textit{design patterns},
    but can be simplified by explicitly defining info for the builder tool in the \lstinline{<BeanName>BeanInfo} class.
    Such class allows exposition of features, specifying customizer class, segregate feats in normal/expert mode, and some other stuff.
    \begin{itemize}
        \item \textbf{Design Patterns for Simple Methods}
        \begin{lstlisting}
    public <PropertyType> get<PropertyName>();
    public void set<PropertyName>(<PropertyType> a);
        \end{lstlisting}
        \item \textbf{Design Patterns for Simple Methods}
        \begin{lstlisting}
    public java.awt.Color getSpectrum (int index);
    public java.awt.Color[] getSpectrum ();
    public void setSpectrum (int index, java.awt.Color color);
    public void setSpectrum (java.awt.Color[] colors);
        \end{lstlisting}
    \end{itemize}
    
\end{enumerate}
